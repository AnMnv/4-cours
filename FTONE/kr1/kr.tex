\documentclass[a4paper,14pt]{extreport}
  \usepackage[left=1.5cm,right=1.5cm,
  top=1.5cm,bottom=2cm,bindingoffset=0cm]{geometry}
  \usepackage{scrextend}
  \usepackage[T1,T2A]{fontenc}
  \usepackage[utf8]{inputenc}
  \usepackage[english,russian,ukrainian]{babel}
  \usepackage{tabularx}
  %\linespread{1.3}
  \usepackage{amssymb}
  \usepackage{color}
  \usepackage{amsmath}
  \usepackage{mathrsfs}
  \usepackage{listings}
  \usepackage{graphicx}
  \usepackage{xcolor}
  \usepackage{hyperref}
  \usepackage{tcolorbox}
  \usepackage{wrapfig}
  \definecolor{electricblue}{rgb}{0.49, 0.98, 1.0}
  

  \newcommand{\decorated}[1]{\tcbox[colback=goldenyellow, colframe=nicered,nobeforeafter]{ #1 }}
   \definecolor{goldenyellow}{rgb}{1.0, 0.87, 0.0}
  \definecolor{nicered}{rgb}{1.0, 0.0, 0.25}
\usepackage{tikz} 


\newcommand{\mybox}[4]{
    \begin{figure}[h]
        \centering
    \begin{tikzpicture}
        \node[anchor=text,text width=\columnwidth-1.2cm, draw, rounded corners, line width=1pt, fill=#3, inner sep=5mm] (big) {\\#4};
        \node[draw, rounded corners, line width=.5pt, fill=#2, anchor=west, xshift=5mm] (small) at (big.north west) {#1};
    \end{tikzpicture}
    \end{figure}
}

\newcommand{\myimg}[4]{\includegraphics[width=#1\linewidth]{#2}\caption{#3}\label{#4}}

\begin{document}
  

\begin{center}\decorated{\textbf{Мнацаканов Антон ДП-82}}\end{center}
Варіант 2 (непарний номер студента у списку групи )
\begin{enumerate}
\item Потенційна яма трикутної форми.
\item Квантово-розмірний Ефект Штарка.
\item Густина станів і квадратичний закон дисперсії.
\item Спін-залежний транспорт носіїв заряду в нанорозмірних структурах.
\item Квантовий ефект Холла. Квант опору.
\item Кулонівська блокада.
\item Класифікація надґраток. Вольт-амперні характеристики надґраток.
\end{enumerate}


\mybox{1. Потенційна яма трикутної форми.}{red!40}{red!5}{
    Для трикутної ями в одиницях $\cfrac{\left(\cfrac{9\pi^2\hbar F^2}{2m}\right)^{\tfrac{1}{3}}}{4}$ різниця між енергіями сусідніх рівнів 
    \begin{equation}\label{1.1}
    \triangle E_{n+1,n}=(4n+7)^{\tfrac{2}{3}}-(4n+3)^{\tfrac{2}{3}} \end{equation} Дивлячись на рівність \ref{1.1} можна побачити, що зі зростанням порядкового номера рівня $n$ відносний внесок постійних доданків у значення виразів у першій і другій дужках зменшується, то різниця між ними, а відповідно й різниця енергій між сусідніми рівнями, як і для гіперболічної ями, у разі збільшення $n$ зменшується також.\\
    Також цікаво порівняти вплив ширини прямокутноїпотенціальноїями на відстань між сусідніми дискретними енергетичними рівнями
    квантовоїчастинки в ямі зі зміною структури енергетичного спектра при переході від прямокутноїдо трикутноїчи гіперболічноїям. У прямокутній ямі ця відстань зменшується, тобто енергетичні рівні згущуються зі збільшенням її ширини а. Аналогічне згущення енергетичних рівнів у разі трикутноїчи гіперболічноїпотенціальних ям спостерігається у напрямку від дна ями до її верху, тобто у тому напрямку, в якому ширина як трикутної, так гіперболічноїям зростають. Отже, можна дійти логічного висновку, що саме збільшення ширини ями знизу вверх у цих ямах є причиною такого згущення дискретних рівнів. А параболічний характер зростання ширини ями є тією границею між ямами різноїформи, де характер слідування енергетичних рівнів один за одним зазнає інверсії – від розрідження до згущення.
}


\mybox{2. Квантово-розмірний Ефект Штарка.}{green!40}{green!5}{
    Під час накладання на кристал однорідного електричного
    поля напруженості $E$ виникає ефект Штарка – зони дозволених
    енергій електрона у кристалі розділяються на сукупність
    еквідистантних рівнів, відстань між якими дорівнює $eEa$, де $a$ – параметр кристалічноїґратки у напрямку $E$. Цю відстань прийнято називати штарківською енергією електрона в кристалі й позначати $\hbar\Omega $. Відповідну їй частоту $\Omega$ називають штарківською частотою
    електрона в кристалі і визначають співвідношенням $\Omega = \dfrac{eEa}{\hbar}$. 
}


\mybox{3. Густина станів і квадратичний закон дисперсії.}{orange!40}{orange!5}{
    Густина станів в 3D системах визначає об’ємну концентрацію носіїв заряду, а її розмірність – еВ$^{-1}$см$^{-3}$. 
    Зі співвідношення \[\rho(E)=\dfrac{(2m)^{\frac32}}{2\pi^2\hbar^2}\cdot \sqrt{E}\] випливає, що у тривимірних системах густина станів вільних електронів є степеневою функцією ефективної маси носіїв заряду та їхньої енергії і зростає зі збільшенням останніх за законом 
    $\rho\sim m^{\frac{3}{2}}E^{\frac{1}{2}}$. 
    \begin{wrapfigure}{r}{0pt}
    \myimg{0.3}{3.1.png}{}{r31}
    \end{wrapfigure}
    Графічно енергетична залежність густини станів вільних тривимірних електронів зоображена на рис. \ref{r31}.
    Те, що густина станів 3D електронів зростає з енергією, є наслідком тривимірності системи, а те, що вона зростає за законом $E^{\frac12}$, є наслідком вибраного нами параболічного наближення для закону дисперсії. Для інших законів дисперсії закон зростання $\rho(E)$
    буде іншим.\\ 
    У 2D системі вільний рух електрона можливий тільки у площині. Для того, щоб розрахувати густину станів для двовимірного електрона, аналогічно до попереднього 
    \begin{wrapfigure}{l}{0pt}
    \myimg{0.3}{3.2.png}{}{r32}
    \end{wrapfigure}

    випадку потрібно знайти площу k-простору $S_k$ між
    двома колами радіусу $k$ i $k+dk$ (рис. \ref{r32}) і поділити її на площу, що припадає на один квантовий стан $S_1$. Густина станів в 2D системах визначає поверхневу концентрацію носіїв заряду. Її розмірність – еВ$^{-1}$см$^{-2}$.
    \begin{equation}\rho(E)=\cfrac{m}{\pi\hbar^2}
    \label{eq1}
    \end{equation}\vspace{0.1cm}

    З (\ref{eq1}) бачимо, що у двовимірних системах з параболічним законом дисперсії густина станів вільних електронів на енергетичному рівні $Е$ від величини енергії не залежить і визначається тільки ефективною масою електрона – $\rho(E)\sim m^1 E^0$.
    Розглянемо тепер 1D кристал, коли вільний рух
    електрона можливий тільки вздовж одного напрямку, наприклад $0Х$. Тоді $k_2 = k_x^2$ , а лінійний інтервал k-простору $l_k$, який відповідає
    інтервалу енергії від Е до $Е+dE$, становить $l_k=2dk$\dots У результаті для одновимірної системи одиничних розмірів з параболічним законом дисперсії для вільних електронів отримуємо \begin{equation}
    \rho(E)=\cfrac{(2m)^{\frac12}}{\pi \hbar}\cdot E^{-\tfrac{1}{2}}
    \end{equation}
    }





    \mybox{3. \dots \texttt{ П Р О Д О В Ж Е Н Н Я}}{orange!40}{orange!5}{
    Процедура розрахунку густини станів вільних електронів і її енергетичної залежності для лінійного закону дисперсії нічим не відрізняється від аналогічної процедури для параболічного закону дисперсії. Якщо зв’язок між $Е$ і $k$ задається дисперсійним
    співвідношенням $E(k)=\tfrac{\hbar k}{\alpha}$, то, як і раніше, фіксованому значенню енергії електрона E у тривимірному $k$-просторі буде відповідати сфера, у двовимірному просторі – коло, а в одновимірному – дві фіксовані точки на осі $0k_x$. Звичайно, для фіксованого значення $Е$ радіус
    \begin{wrapfigure}{l}{0pt}
    \myimg{0.2}{3.3.png}{}{r33}
    \end{wrapfigure}
    сфери чи кола, а також відстань зазначених точок від початку відліку значень хвильових векторів будуть іншими, ніж за параболічного закону дисперсії, як й іншими будуть товщини прошарків k-простору, що лежать між довільними двома “поверхнями” постійної енергії.\\
    \begin{wrapfigure}{l}{0pt}
    \myimg{0.25}{3.4.png}{}{r34}
    \end{wrapfigure} 
    Для закону дисперсії $E(k)=\tfrac{\hbar k}{\alpha}$ дорівнює $dk=\tfrac{\alpha}{\hbar}dE$. Відповідно для 3D електронів об’єм k-простору, що лежить між
    двома поверхнями постійної енергії $Е$ і $E+dE$, буде

    \begin{align}
    V_k=\dfrac{4\pi\alpha^3}{\hbar^3}E^2 dE \quad \text{та} \quad  \rho(E)=\dfrac{\alpha^3E^2}{\pi^2\hbar^3}
    \end{align}
    Графічно залежність густини станів вільних 3D електронів з лінійним законом дисперсії показана на рис. \ref{r33}.\\
    Для 2D електронів, що описуються законом дисперсії\\ 
    $E(k)=\dfrac{\hbar k}{\alpha}$, площа $k$-простору, що лежить між колами постійної енергії $Е$ і $E+dE$,
    становитиме 
    \begin{align}
    S_k=\dfrac{2\pi\alpha^2}{\hbar^2}E dE \quad \text{та} \quad  \rho(E)=\dfrac{\alpha^2E}{\pi\hbar^2}
    \end{align}
    Густина станів 2D електронів з лінійним законом дисперсії лінійно зростає у разі збільшення їхньої енергії Е, графік приведений на рис. \ref{r34}. 
    \begin{wrapfigure}{l}{0pt}
    \myimg{0.3}{3.5.png}{}{r35}
    \end{wrapfigure}

    У разі 1D електронів, що описуються законом дисперсії $E(k)=\dfrac{\hbar k}{\alpha}$, величина
    \begin{align}
    L_k=\dfrac{2\alpha}{\hbar}dE \quad \text{тому} \quad  \rho(E)=\dfrac{2\alpha}{\pi\hbar}
    \end{align}
    Тому, густина станів вільних одновимірних електронів з лінійним законом дисперсії не залежить від енергії і графічно має вигляд, показаний на рис. \ref{r35}.
}







\mybox{4. Спін-залежний трнспорт носіїв заряду в нанорозмірних структурах.}{black!40}{black!5}{
Спін, будучи однією з фундаментальних характеристик електрона, призводить до появи нових особливостей транспорту носіїв заряду в наноструктурах. Спінові ефекти виникають, коли в матеріалі з'являється спіновий дисбаланс заселеності рівня Фермі. Такий дисбаланс зазвичай присутній в феромагнітних матеріалах, у яких щільності вакантних станів для електронів з різними спінами практично ідентичні, проте ці стани сушественно розрізняються по енергії, як схематично показано на рис.\ref{l41} (тут і далі під різними спінами електрона розуміються різні проекції його спіна на вісь квантування). Енергетичний зрушення призводить до заповнення дозволених зон електронами з одним певним спіном і до появи у матеріалу власного магнітного моменту (намагніченості). Заселеність енергетичних зон електронами з різним спіном визначає як спінову поляризацію інжектіруемих з такого матеріалу електронів, так і особливості транспорту носіїв заряду через нього.
Власну спінову поляризацію електронів в матеріалі визначають як відношення різниці концентрацій електронів з різними спінами $ n_{\uparrow}$ та $n_{\downarrow}$ до їх загальної концентрації:
\begin{equation}
P = \frac{n_{\uparrow}-n_{\downarrow}}{n_{\uparrow} + n_{\downarrow}}
\end{equation}
Електричний струм в твердотільних структурах, складених з матеріалів з різною спінової поляризацією, залежить від спінової поляризації носіїв заряду і спину поляризації областей, через які ці носії рухаються. Електрони, інжектовані з певним спіном, можуть зайняти в колекторі тільки вакантні місця з такою ж орієнтацією спіна. Електрон, спочатку спін-поляризований в инжектируются електроді, у міру руху змінює (через розсіювання) як свій імпульс, так і свій спін. Для практичних застосувань важливо знати, як довго електрон «пам'ятає» свою спінову орієнтацію. Як характеристики «спінової пам'яті» використовують середня відстань, яку проходить електроном до зміни свого спіна. Його називають длuной cnunoвoй penaccauuu (spin relaxation length) $ l_{s}. $
\center{\myimg{0.5}{4.1.png}{}{l41}}
}
\mybox{4. \dots \texttt{ П Р О Д О В Ж Е Н Н Я}}{black!40}{black!5}{
У твердих тілах її величина перевищує 100 нм і визначається спін-незалежним середнім вільним пробігом електронів, в якості якого доцільно розглядати середню довжину вільного пробігу при неупругом розсіянні $ l_{ \text {in}} $, оскільки при зміні напрямку спіна розсіяного електрона баланс по імпульсу в системі взаємодіючих частинок не зберігається.\\
Також у спін-поляризованих матеріалах стану з переважним спіном контролюються намагниченностью цих матеріалів. Якщо намагніченість змінюється на протилежну, то переважна орієнтація спінів також змінюється на протилежну. При інжекції спін-поляризованих електронів в матеріал з відмінною від нуля намагниченностью, а отже, і зі спінової поляризацією, контрольованої зовнішнім магнітним полем, цей матеріал може вести себе як провідник або як ізолятор - в залежності від напрямку намагніченості і орієнтації спінів інжектованих електронів. При однаковому напрямку спинив інжектованих електронів і електронів матеріалу забезпечується найвища провідність. Протилежний зміст спинив перешкоджає проходженню електронів через матеріал.
}


\mybox{5. Квантовий ефект Холла. Квант опору.}{yellow!40}{yellow!5}{
    Один з найцікавіших ефектів -- це квантовий ефект Холла (КЕХ), за умов реалізації якого квантується холлівська провідність двовимірної структури, поміщеної за низьких температур у квантово сильне магнітне поле. Головна увага саме до цього ефекту зумовлена тим, що КЕХ є унікальним фізичним явищем, притаманним винятково 2D системам, явищем, яке не спостерігається більше ні в яких системах іншої вимірності. \\
    Ефект Холла належить до групи гальваномагнітних явищ і
    виникає тоді, коли на провідне середовище одночасно діють електричне і магнітне поля. Якщо зразок, в якому протікає електричний струм з густиною $j$, помістити в магнітне поле з індукцією $B$, спрямоване перпендикулярно до напрямку струму, то у ньому в результаті дії сили Лоренца виникає поперечне щодо напрямків струму і магнітного поля електричне поле напруженості $E_H$, яке називається холлівським полем.\\ 
    Напруженість холлівського поля, природно, є тим більшою, чим більша густина електричного струму в зразку та індукція магнітного поля, і зв’язана з ними співвідношенням
    \begin{equation}
    E_H=R_HjB
    \end{equation}
    Значення холлівської провідності і холлівського опору в режимі цілочислового КЕХ можна подати й інакше, ніж (12.3) і (12.4), а саме:
\begin{align}
\sigma_{H} &=\frac{2 \alpha}{\mu_{o} c} i \\
\rho_{H} &=\frac{\mu_{o} c}{2 \alpha i}
\end{align}

де $\alpha=\left(\mu_{o} c / 2\right)\left(e^{2} / \hbar\right)-$ це стала тонкої структури, яка є безрозмірною мірою взаємодії між речовиною та електромагнітним полем і лежить в основі сучасної квантової електродинаміки. Завдяки цьому найпершим практичним застосуванням цілочислового КЕХ було уточнення саме цієї важливої сталої. Піонерська робота фон Клітцінга і співавторів, де вперше повідомлялося про КЕХ і про яку йшлося вище, так і називалася "Новий метод дуже точного визначення сталої тонкої структури, що ґрунтується на вимірюванні квантованого холлівського опору". Цей альтернативний спосіб визначення $\alpha$ за допомогою експерименту принципово нового типу збільшує кількість незалежних вимірів сталої тонкої структури, що повинно дати додаткову інформацію стосовно несуперечливості наших уявлень про фізичну картину світу і може привести до формування нової, точнішої системи фундаментальних констант. 
}

\mybox{5. \dots \texttt{ П Р О Д О В Ж Е Н Н Я}}{yellow!40}{yellow!5}{
Так, міжнародне значення величини $\alpha^{-1}$, прийняте у 1973 році, становить $\alpha^{-1}=137,03604 \pm 0,00011 .$ Використання квантового ефекту Холла приводить до значення 1986 року $\alpha^{-1}=137,035991 \pm 0,000008$. У межах експериментальної точності результат не залежить ні від матеріалу (Si, GaAs, InGaAs), ні від технологічного методу вирощування структур, ні від рухливості електронів. 
На рис. \ref{l51} як приклад, що підтверджує сказане, подано порівняння результатів вимірювань квантового холлівського опору, за яким визначається $\alpha$, на семи гетероструктурах на основі GaAs з різною рухливістю двовимірних носіїв заряду.
\center{\myimg{0.5}{5.1.png}{}{l51}}
Іншим важливим практичним застосуванням цілочислового КЕХ, на можливість якого також одразу вказали фон Клітцінг і співавтори, стало його використання для отримання абсолютного еталона опору, який би залежав лише від фундаментальних констант природи - $\rho_{H}(i)=\hbar / e^{2} i \approx 25812,80 / i$ Ом. 3 одного боку, величина $\rho_{H}$, яку визначають на експерименті, лежить у дуже зручному для вимірювання опору діапазоні. 3 іншого боку, ії значення в експериментах з цілочислового КЕХ відтворюється з точністю до восьми значущих цифр. Зважаючи на це, у 1990 р. квант опору $h/e^{2}$, що визначається з експериментів цілочислового КЕХ, був прийнятий як міжнародний еталон опору.
}


\mybox{6. Кулонівська блокада.}{blue!30}{blue!5}{
    Кулонівською блокадою називають явище відсутності струму в
    електричному колі з тунельним переходом, що містить квантову точку, за наявності різниці потенціалів на тунельному переході, спричинене неможливістю тунелювання електронів унаслідок їхнього кулонівського відштовхування.\\ 
    Фізично кулонівська блокада виникає тоді, коли втрата квантовою точкою, впровадженою у тунельний бар’єр, одного електрона стає енергетично невигідною, тобто коли за такої втрати її електростатична енергія зростає.\\ 
    На основі явища кулонівської блокади можна виготовляти
    принципово нові прилади наноелектроніки – одноелектронні
    транзистори, теоретична межа швидкодії яких становить сотні
    терагерц, а енергоспоживання – кілька десятків нановат.
}

\mybox{7. Класифікація надґраток. Вольт-амперні характеристики надґраток.}{electricblue!40}{electricblue!5}{
    Шари, що формують надґратку, можуть відрізнятися або
    хімічним складом, або характером легування (типом провідності),або тим та іншим одночасно.
    Надґратки, утворені періодичною послідовністю двох напівпровідників різного хімічного складу, називають \textbf{композиційними надґратками}, або надґратками з гетероструктурами. Надґратки, утворені періодичною послідовністю шарів одного і того ж напівпровідника $n-$ i $p$-типу провідності, отриманих легуванням
    різними домішками, з можливими бездомішковими (і) прошарками між ними, називають \textbf{легованими надґратками}, або $n_ip_i$-структурами. Комбінуючи зміни хімічного складу шарів, що контактують між собою, і їхнє легування, отримують \textbf{композиційні леговані надґратки}.\\
    ..................
    Вольт-амперна характеристика (ВАХ) надґратки – це залежність сили струму, який тече через надґратку, від прикладеної до неї напруги $I = f(U)$, або залежність густини струму від напруженості прикладеного електричного поля $j = f(E)$. Як і багато інших фізичних
    характеристик надґраток, їхні ВАХ істотно анізотропні і залежать від того, паралельно чи перпендикулярно до шарів надґратки накладено зовнішнє електричне поле.
}


\end{document}