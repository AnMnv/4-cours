\documentclass[a4paper,14pt]{extreport}
  \usepackage[left=1.5cm,right=1.5cm,
      top=1.5cm,bottom=2cm,bindingoffset=0cm]{geometry}
  \usepackage{scrextend}
  \usepackage[T1,T2A]{fontenc}
  \usepackage[utf8]{inputenc}
  \usepackage[english,russian,ukrainian]{babel}
  \usepackage{tabularx}
  \usepackage{amssymb}
  \usepackage{color}
  \usepackage{amsmath}
  \usepackage{mathrsfs}
  \usepackage{listings}
  \usepackage{graphicx}
  \usepackage{xcolor}
  \usepackage{hyperref}
  \usepackage{tcolorbox}
  \usepackage{tikz}
  \usepackage{ulem}
  \usepackage[framemethod=TikZ]{mdframed}
  \usepackage{wrapfig,boxedminipage,lipsum}
  \usepackage{csvsimple}
  \usepackage{capt-of}

  \usepackage{siunitx}
\usepackage{tikz} % To generate the plot from csv
\usepackage{pgfplots}
\pgfplotsset{compat=newest} % Allows to place the legend below plot
\usepgfplotslibrary{units} % Allows to enter the units nicely



  
  
%}

\newcommand{\img}[4]{\center{\includegraphics[width=#1\linewidth]{#2}}\captionof{figure}{#3}\label{#4}}
\begin{document}
  \pagecolor{white}

  %----------------------------------------1
  \newtcbox{\xmybox}[1][red]{on line, arc=7pt,colback=#1!10!white,colframe=#1!50!black, before upper={\rule[-3pt]{0pt}{10pt}},boxrule=1pt, boxsep=0pt,left=6pt,right=6pt,top=2pt,bottom=2pt}

\begin{titlepage}
    \begin{center}
    \large
    Національний технічний університет України \\ "Київський політехнічний інститут імені Ігоря Сікорського"


    Факультет Електроніки

    Кафедра мікроелектроніки
    \vfill

    \textsc{ЗВІТ}\\

    {\Large Про виконання лабораторної роботи №1\\
    з дисципліни: «Фізичні основи сенсорики»\\[1cm]

    Резистивні сенсори температури


    }
    \bigskip
    \end{center}
    \vfill

    \newlength{\ML}
    \settowidth{\ML}{«\underline{\hspace{0.4cm}}» \underline{\hspace{2cm}}}
    \hfill
    \begin{minipage}{1\textwidth}
    Виконавець:\\
    Студент 4-го курсу \hspace{4cm} $\underset{\text{(підпис)}}{\underline{\hspace{0.2\textwidth}}}$  \hspace{1cm}А.\,С.~Мнацаканов\\
    \vspace{1cm}

    Перевірив: \hspace{6.1cm} $\underset{\text{(підпис)}}{\underline{\hspace{0.2\textwidth}}}$  \hspace{1cm}ас. Коваль В. М.\\

    \end{minipage}

    \vfill

    \begin{center}
    2021
    \end{center}
\end{titlepage}



\newpage
\setcounter{page}{2}
\textbf{Мета роботи} – визначити залежність опору резистивного сенсора
температури від відстані до джерела тепла.



\begin{center}
  \textbf{Короткі теоретичні відомості}
\end{center} 
  Сенсори температури є одними з найпоширеніших класів сенсорів у
  промисловості (приблизно 50\% від загальної кількості первинних
  перетворювачів), що обумовлює існування великої кількості їх різновидів:
  металеві терморезистори, термістори, позистори, болометри, термопарні сенсори,
  термодіоди, термотранзистори, піроелектричні сенсори, теплові та фотонні
  сенсори.\par
  Терморезистори – це резистивні елементи, виготовлені з провідникового або
  напівпровідникового матеріалу, в яких використовується залежність електричного
  опору матеріалу від температури. Відповідно до виду застосовуваного матеріалу
  розрізняють металеві та напівпровідникові терморезистори (термістори,
  позистори).\par
  Температурна залежність питомого опору металу визначається, головним
  чином, довжиною вільного пробігу електронів:
  \begin{equation}
  \rho=\dfrac{h}{K\cdot q^2\cdot n^{2/3}\cdot l_{\text{ср}}},
  \end{equation}
  де $l_{\text{ср}}$ – середня довжина вільного пробігу електрону, n – концентрація електронів,
  q – елементарний заряд, h – постійна Планка, К – стала.\par
  Зі зростанням температури збільшується амплітуда коливання вузлів
  кристалічної гратки, що призводить до зменшення довжини вільного пробігу електронів та зростання опору провідника. Тому для металів температурний коефіцієнт опору (ТКО) завжди приймає додатнє значення. Для виготовлення металевих терморезисторів використовують платину, мідь, нікель, вольфрам, пермалой тощо.\par
  Температурна залежність електропровідності напівпровідників має
  складніший характер і визначається температурними залежностями концентрації та рухливості носіїв заряду:
  \begin{align}
  \sigma &= q(n\mu_n+p\mu_p)\\
  n &= n_0 e^{-\frac{\triangle E_d}{2kT}}\\
  p &= p_0 e^{-\frac{\triangle E_a}{2kT}}
  \end{align}

  де
  $\sigma$ – питома електропровідність,

  q – елементарний заряд,

  n p, – концентрація

  електронів та дірок, $\mu_n,\mu_p$ – рухливість електронів та дірок, $\triangle E_d, \triangle E_a$ – глибина
  залягання донорних та акцепторних домішок, k – стала Больцмана, Т – температура.\par

  В області низьких температур зі збільшенням температури зростає
  концентрація вільних носіїв заряду за рахунок іонізації домішок. При подальшому
  збільшенні температури концентрація вільних носіїв заряду практично не
  змінюється, оскільки всі домішки вже іонізовані, а ймовірність іонізації власних
  атомів ще дуже низька. Ця ділянка називається ділянкою виснаження домішок.
  При подальшому зростанні температури концентрація вільних носіїв заряду
  зростає за рахунок іонізації власних атомів.\par

  Рухливість носіїв заряду обумовлюється процесами розсіяння носіїв заряду
  у напівпровідниках. При цьому розглядають 2 механізми розсіяння: розсіяння на
  теплових коливаннях гратки та на іононізованих домішках:
  \begin{align}
  \mu\approx& T^{-\frac32}\\
  \mu\approx& T^{\frac32}
  \end{align}

  При зростанні температури амплітуда теплових коливань атомів гратки
  зростає, при цьому зростає розсіяння носіїв заряду на них, тому рухливість носіїв
  заряду падає. Даний механізм розсіяння переважає за високих температур, а за
  низьких має нехтовно малий вплив.\par
  За низьких температур домінуючий вплив здійснює розсіяння на іонізованих
  домішках, оскільки навіть за низьких температур більша частина домішкових
  атомів знаходиться в іонізованому стані. Кожний іон створює навколо себе
  електричне поле, яке притягує або відштовхує носії заряду, тобто відхиляє
  траєкторію їх руху. Чим вища температура, тим більша швидкість носія заряду, а,
  отже меншим є час його перебування під впливом відхиляючого поля іонізованих
  домішків, а тому рухливість носіїв заряду зростає.\par
  Однак рухливість носіїв заряду визначається також і концентрацією
  домішків: чим вища концентрація домішків, тим рухливість носіїв заряду менша.
  Слід зауважити, що в області високих температур вплив концентрації є незначним.
  Оскільки температурна залежність рухливості описується більш слабкою
  степеневою функцією, а температурна залежність концентрації –
  експоненціальною функцією, то в загальному підсумку температурна залежність
  питомої електропровідності описується температурною залежністю концентрації,
  а залежність рухливості від температури вносить свій вклад лише в області
  виснаження домішків.\par
  Напівпровідникові терморезистори в залежності від того зростає чи
  зменшується їх опір при нагріванні поділяються відповідно на позистори та
  термістори.\par
  Термістор – це напівпровідниковий терморезистор з від’ємним ТКО.
  Фізичні явища, які лежать в основі роботи термісторів: збільшення концентрації
  носіїв зарядів, збільшення інтенсивності обміну електронами між іонами зі
  змінною валентністю, фазові перетворення напівпровідникового матеріалу.
  Перше явище лежить в основі роботи термісторів, виготовлених з
  монокристалів ковалентних напівпровідників (кремнію, германію, карбіду
  кремнію, з’єднань А3В5). Такі терморезистори працюють в області температур, що
  відповідає домішковій або власній електропровідності напівпровідника.
  Найбільш поширені термістори виготовляються з порошків оксидів Mn, Fe,
  Ni, Cu, Zn та Co. Електропровідність таких матеріалів пов’язана з обміном
  електронів між сусідніми іонами.\par
  В оксидах ванадію за температур фазового переходу (68 та – 110 $^\circ$С)
  спостерігається зменшення питомого опору на декілька порядків, що також
  використовується в термісторах.\par
  Позистор – це напівпровідниковий терморезистор з додатнім ТКО.
  Позистори виготовляють на основі кераміки з титанату барію з домішками La, Ce,
  Ta, Nb, Sb. Крім титаната барія, використовують також кремнієві позистори, які
  працюють в області виснаження домішків.\par
  Напівпровідникові терморезистивні перетворювачі мають такі переваги
  порівняно з металевими терморезисторами: малі габарити, мала інерційність та
  висока чутливість. Однак напівпровідникові перетворювачі поступаються
  металевим в точності.\par
  Термопарні сенсори температури працюють на основі ефекту Зеєбека, який
  полягає у виникненні термоЕРС в колі з двох різнорідних провідників чи
  напівпровідників, які називаються термоелектродами, якщо температура місця
  з’єднання електродів (так званий робочий або гарячий спай) та температура
  вільних (або холодних) кінців є різною. Як правило, термопара складається з двох
  послідовно з’єднаних пайкою або зварюванням металевих різнорідних
  провідників. У поєднанні з електровимірювальними приладами термопара
  утворює термоелектричний термометр, шкала якого градуюється безпосередньо в
  К або $^\circ$С. Принцип дії таких термометрів оснований на вимірюванні термоЕРС
  термопари, один кінець якої термостатований (як правило, при 0$^\circ$С), та
  співставленні цієї величини з табличними даними для даної пари термоелектродів
  з метою визначення досліджуваної температури. Для виготовлення металевих
  термопарних сенсорів використовують наступні матеріали: хромель – алюмель,
  мідь – константан, залізо – константан тощо. Напівпровідникові термопарні
  сенсори температури також набувають значного поширення. Так, широко
  використовується термопара кремній – алюміній. Переваги кремнієвих термопар –
  це застосування традиційної інтегральної технології для виготовлення такого
  сенсору.\par
  ІЧ-фотоприймачі можна поділити на два класи: теплові та фотонні. В
  теплових приймачах ІЧ випромінювання, яке поглинається, викликає нагрівання
  чутливого елементу, що в свою чергу викликає зміну певних характеристик
  детектора. В фотонних приймачах поглинуте ІЧ випромінювання призводить до
  переходів між енергетичними станами кристалу. Теплові приймачі
  характеризуються рівномірною чутливістю у досить широкому діапазоні ІЧ-
  спектру, однак мають невелику чутливість та швидкодію. Для фотонних
  приймачів характерним є селективність по спектру, однак високий рівень
  чутливості в цьому діапазоні та швидкодія.\par
  Принцип дії ІЧ-фотоприймачів базується на основі основних законів
  поглинання та випромінювання твердих тіл, сформульованих для так званого
  абсолютно чорного тіла. Абсолютно чорне тіло (АЧТ) – це тіло, що поглинає все
  падаюче на нього електромагнітне випромінювання. Перший закон
  випромінювання Віна:
  \begin{align}
   u_v = v^3\cdot f \left( \dfrac{v}{T} \right) 
  \end{align}
  де $u_v$ – густина енергії випромінювання, Т –
  температура тіла, що випромінює, f – функція, що залежить лише від частоти та
  температури.\par
  Перший закон Віна є загальною формулою, з якої може бути виведений
  будь-який інший закон випромінювання, наприклад, закон Стефана-Больцмана,
  другий закон Віна, закон Планка, закон Релея-Джинса і т.д.
  Другий закон Віна – частковий випадок першого закону Віна, справедливий
  лише в області високих частот:
  \begin{align}
   u_v = \dfrac{8\cdot \pi\cdot h\cdot v^3}{c^3}\cdot e^{\frac{hv}{kT}},
  \end{align}
  де h – стала Планка, с – швидкість світла.\par
  Закон Релея-Джинса справедливий в області низьких частот:
  \begin{align}
  E(v,T)=\dfrac{2\cdot\pi\cdot k \cdot T \cdot v^3 }{c^3}
  \end{align}
  Цей закон передбачає квадратичну залежність спектральної густини
  випромінювання від частоти. При прямуванні частоти до нуля даний закон
  переходить в закон Планка:
  \begin{align}
  I = \dfrac{2\cdot h \cdot v^3}{c^2}\cdot \dfrac{1}{e^{\frac{hv}{kT}}-1} 
  \end{align}
  де Е (v, Т) – спектральна густина випромінювання, І – інтенсивність випромінювання АЧТ.\par
  Закон Планка визначає спектр випромінювання АЧТ. А закон Релея-Джинса та другий закон Віна є його крайніми випадками, які історично стали відомими раніше.\par
  Загальна енергія теплового випромінювання визначається законом Стефана-Больцмана:
  \begin{align}
  j = \epsilon \cdot \sigma \cdot T^4
  \end{align}
  Із закону Планка шляхом диференціювання випливає закон зміщення Віна, який полягає в тому, що довжина хвилі, за якої енергія випромінювання АЧТ є максимальною, визначається формулою:
  \begin{align}
  \lambda = \dfrac{0,0028999}{T}
  \end{align}
  За цією формулою визначають співвідношення між температурою АЧТ та
  кольором його випромінювання. Оптична пірометрія – це метод вимірювання
  температури, який базується на співвідношенні між температурою тіла та
  оптичним випромінюванням, яке це тіло випромінює. Звідси пірометр – це прилад для безконтактного вимірювання температури тіла, принцип дії якого основується
  на вимірюванні потужності теплового випромінювання досліджуваного об’єкту.
  Болометр – це терморезистор з зачорненою поверхнею, яка здатна ефективно
  поглинати ІЧ-випромінення. Безконтактний метод вимірювання є необхідним для:
  \begin{itemize}
  \item високих вимірюваних температур (більше 2000$^\circ$С),
  \item дуже агресивного оточуючого середовища (хімічна промисловість),
  \item матеріалів, що погано проводять тепло (скло, дерево, пластмаси),
  \item частин, що знаходяться під високою напругою,
  \item рухомих тіл (наприклад, листовий матеріал в прокатному виробництві
  металу).
  \end{itemize}

  Піроелектричний сенсор температури працює на основі піроелектричного
  ефекту, який полягає у зміні поляризованості діелектрика при зміні температури.
  В найпростішому вигляді даний клас сенсорів являє собою конденсатор –
  діелектрична пластина виготовлена з піроелектрику і розміщена між металевими
  обгортками. На одну з металевих обгорток наноситься зачорнений шар. В
  результаті поглинання теплової енергії температура пластини конденсатора
  збільшується і між обгортками з’являється напруга, що реєструється.
  Матеріали піроелектричних датчиків: тригліцинсульфат, титанат барія,
  титанат свинця тощо. Основна відмінність піроелектричних сенсорів температури
  від термодатчиків полягає в тому, що в піродатчиках нескомпенсований
  електричний заряд виникає лише в моменти швидкої появи/зникнення
  випромінювання. При тривалому опроміненні піродатчиків, електричний сигнал
  на виході сенсора буде рівний нулю. До переваг піроелектричних сенсорів
  температури відносять їх високу швидкодію та відсутність нагріву активного
  елементу сенсору.

\begin{center}
\textbf{Порядок виконання роботи}
\end{center}
\begin{enumerate}
  \item Встановити тепловий екран на штатив між джерелом та приймачем ІЧ
випромінювання.
\item  Ввімкнути блок живлення ЛИПС-35, виставити на ньому вихідну напругу 25
В та вихідний струм 1...2 А.
\item  Ввімкнути ампервольтомметр Р386 та виставити на ньому початковий
вимірювальний діапазон опорів до 1МОм.
\item  Коли лампа розжарювання досягне стаціонарного режиму (ІЧ
випромінювання постійної густини), зняти тепловий екран та розпочати
вимірювання.
\item  За допомогою ампервольтомметра Р386 виміряти початковий опір
болометра (рівноважне значення), який знаходиться на відстані 20 см від
лампи розжарювання.
\item  Перемістити джерело ІЧ випромінювання вздовж штативу на 1 см в
напрямку приймача ІЧ випромінювання.
\item  За допомогою ампервольтомметра Р386 виміряти опір болометра
(рівноважне значення), який знаходиться на відстані 19 см від лампи
розжарювання.
\item  Повторювати два попередні пункти до тих пір, поки відстань між лампою
розжарювання та болометром досягне 1 см.
\item Під час проведення вимірювань врахувати, що теплові приймачі є досить
чутливими до конвекційних потоків та характеризуються тепловою
інерцією, тому потребують захисту від протягів і встановленню рівноважних
значень опору при кожному дослідженні.
\item Вимкнути блок живлення ЛИПС-35 та ампервольтомметр Р386.
\end{enumerate}




\begin{center}
\textbf{Обробка результатів вимірювання}
\end{center}
\begin{enumerate}
\item Побудувати графічну залежність опору резистивного сенсора температури
від відстані до джерела теплового випромінювання.
\item Апроксимувати побудовану графічну залежність однією з відомих математичних функцій.
\item  На основі побудованого графіку встановити знак ТКО болометра. При цьому слід врахувати закон обернених квадратів для теплового випромінювання.
\item  Зробити висновок про тип терморезистора та характер залежності опору болометра від відстані до джерела ІЧ випромінювання.
\end{enumerate}
\clearpage
\newpage

\begin{center}
\textbf{Виконання}
\end{center}
Побудуємо графічну залежність опору резистивного сенсора температури
від відстані до джерела теплового випромінювання використовуючи таб \ref{t1}.


\begin{table}[ht]
\begin{minipage}[b]{0.4\linewidth}
\centering
\caption{Залежність опору резистивного сенсора температури від відстані до джерела теплового випромінювання}
  \begin{tabular}{|c|c|}
    \hline
    l,см                      & R, Ом     \\ \hline
    20                        & 0,842 \\ \hline
    19                        & 0,819 \\ \hline
    18                        & 0,8   \\ \hline
    17                        & 0,775 \\ \hline
    16                        & 0,729 \\ \hline
    15                        & 0,713 \\ \hline
    14                        & 0,708 \\ \hline
    13                        & 0,702 \\ \hline
    12                        & 0,689 \\ \hline
    11                        & 0,679 \\ \hline
    10                        & 0,667 \\ \hline
    9                         & 0,643 \\ \hline
    8                         & 0,634 \\ \hline
    7                         & 0,581 \\ \hline
    6                         & 0,579 \\ \hline
    5                         & 0,572 \\ \hline
    4                         & 0,541 \\ \hline
    3                         & 0,527 \\ \hline
    2                         & 0,497 \\ \hline
    1                         & 0,421 \\ \hline
  \end{tabular}  
  \label{t1}
\end{minipage}
\hfill
\begin{minipage}[b]{0.6\linewidth}
\centering
\img{0.9}{11.png}{Залежність опору резистивного сенсора температури від відстані до джерела теплового випромінювання.}{im1}
\end{minipage}
\end{table}




\begin{center}
\textbf{Висновок}
\end{center}

Виходлячи з даних з рис. \ref{im1} видно, що при зменшеннi вiдстанi до джерела випромiнювання і збiльшеннi температури опiр зменшується, тобто можна сказати, що ТКО - вiд’ємний $\Rightarrow$ тип терморезистору - термiстор. 
На рис. \ref{im1} видно деяку нелiнiйнiсть,
проте вона наближається до прямолiнiйної залежностi, про що свiдчить симетричнiсть графiку вiдносно деякої точки перегину, через те, що є паралельно пiд’єднаний опір, номiнал якого рiвний термiстору за кiмнатної температури.

\newpage
\begin{center}
\textbf{Контрольні запитання}
\end{center}

1. В чому полягає принцип дії металевого терморезистора?\\
2. Що собою являє температурна залежність концентрації носіїв заряду в
напівпровідниках?\\
3. Якою є температурна залежність рухливості носіїв заряду в
напівпровідниках?\\
4. Поясніть принцип дії позистора та термістора на основі температурної
залежності електропровідності в напівпровідниках.\\
5. Порівняйте напівпровідникові та металеві терморезистори.\\
6. Що таке ефект Зеєбека та його застосування в термометрії?\\
7. Перелічіть сенсори, які використовуються для контактного та
безконтактного вимірювання температури.\\
8. Що собою являє болометр?\\
9. Які особливості теплових та фотонних ІЧ-приймачів?
10.Які закони поглинання та випромінювання, сформульовані для АЧТ, Ви
знаєте?\\
11.Поясність фізичну суть роботи піроелектричних сенсорів температури.\\
12.Наведіть особливості піроелектричних сенсорів температури.\\

\end{document}