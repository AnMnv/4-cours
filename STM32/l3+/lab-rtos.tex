\documentclass[a4paper,14pt]{extreport}
  \usepackage[left=1.5cm,right=1.5cm,
      top=1.5cm,bottom=2cm,bindingoffset=0cm]{geometry}
  \usepackage{scrextend}
  \usepackage[T1,T2A]{fontenc}
  \usepackage[utf8]{inputenc}
  \usepackage[english,russian,ukrainian]{babel}
  \usepackage{tabularx}
  \usepackage{amssymb}
  \usepackage{color}
  \usepackage{amsmath}
  \usepackage{mathrsfs}
  \usepackage{listings}
  \usepackage{graphicx}
  \usepackage{xcolor}
  \usepackage{hyperref}

  



  
  
%}

\newcommand{\img}[4]{\center{\includegraphics[width=#1\linewidth]{#2}}\captionof{figure}{#3}\label{#4}}
\begin{document}
  \pagecolor{white}

  %----------------------------------------1
\begin{titlepage}
    \begin{center}
    \large
    Національний технічний університет України \\ "Київський політехнічний інститут імені Ігоря Сікорського"


    Факультет Електроніки

    Кафедра мікроелектроніки
    \vfill

    \textsc{ЗВІТ}\\

    {\Large Про виконання лабораторної роботи\\
    з дисципліни: «Мікропроцесори та мікроконтролери»\\[1cm]

    %Резистивні сенсори температури


    }
    \bigskip
    \end{center}
    \vfill

    \newlength{\ML}
    \settowidth{\ML}{«\underline{\hspace{0.4cm}}» \underline{\hspace{2cm}}}
    \hfill
    \begin{minipage}{1\textwidth}
    Виконавець:\\
    Студент 4-го курсу \hspace{4cm} $\underset{\text{(підпис)}}{\underline{\hspace{0.2\textwidth}}}$  \hspace{1cm}А.\,С.~Мнацаканов\\
    \vspace{1cm}

    Перевірив: \hspace{6.1cm} $\underset{\text{(підпис)}}{\underline{\hspace{0.2\textwidth}}}$  \hspace{1cm} Татарчук Д. Д.\\

    \end{minipage}

    \vfill

    \begin{center}
    2021
    \end{center}
\end{titlepage}



\newpage
\setcounter{page}{2}


\begin{verbatim}
 /* USER CODE END Header_StartGreen */
 void StartGreen ( void const * argument )
 {
     /* USER CODE BEGIN StartGreen */
     /* Infinite loop */
     for (;;)
         {
         HAL_GPIO_TogglePin ( GPIOC , GPIO_PIN_5 ) ;
         osDelay (100) ;
         }
 }
 
 
 /* *
 
 * @brief Function implementing the BlueTask thread .
 
 * @param argument : Not used
 
 * @retval None
 
 */
 
 /* USER CODE END Header_StartBlue */
 
 void StartBlue ( void const * argument )
 
 {
    for (;;)
         {
         HAL_GPIO_TogglePin ( GPIOB , GPIO_PIN_4 ) ;
         osDelay (300) ;
         } 
 }
\end{verbatim}



------------------------------------------------------------------------------------------------------------------------



\begin{verbatim}

int but_prev = 0;
int but_cur = 0;
int a = 0;

while (1)
    {
        
        but_cur = HAL_GPIO_ReadPin(GPIOA, GPIO_PIN_0);
        

        if ( (but_prev == 0) && (but_cur != 0) )
            {    

                a=a+1;
                switch (a)
                    {
                        case 1:
                            HAL_GPIO_TogglePin(GPIOB, GPIO_PIN_4);
                            HAL_Delay(2000);
                            break;

                        case 2:
                            HAL_GPIO_TogglePin(GPIOB, GPIO_PIN_4);
                            HAL_Delay(3000);
                            break;

                        case 3:
                            a=0;
                            break;
                    }



            }


     
        else
            {
                
                switch (a)
                    {
                        case 0:
                            HAL_GPIO_TogglePin(GPIOB, GPIO_PIN_5);
                            HAL_Delay(1000);
                            break;

                        case 1:
                            HAL_GPIO_TogglePin(GPIOB, GPIO_PIN_5);
                            HAL_Delay(2000);
                            break;

                        case 2:
                            HAL_GPIO_TogglePin(GPIOB, GPIO_PIN_5);
                            HAL_Delay(3000);                            
                            break;
                    }



            }

                but_prev = but_cur;
        



    }
\end{verbatim}
\begin{center}
\textbf{Спостереження}
\end{center}

За допомогою написаної програми та заздалегідь підключеним світлодіодам до пінів 5 та 4, можна спостерігати як вони  будуть блимати з рiзним перiодом: один 300 мс а інший 100 мс і при цьому використовувалась операцiйна система реального часу RTOS.
\end{document}