\documentclass[a4paper,14pt]{extreport}
  \usepackage[left=1.5cm,right=1.5cm,
      top=1.5cm,bottom=2cm,bindingoffset=0cm]{geometry}
  \usepackage{scrextend}
  \usepackage[T1,T2A]{fontenc}
  \usepackage[utf8]{inputenc}
  \usepackage[english,russian,ukrainian]{babel}
  \usepackage{tabularx}
  \usepackage{amssymb}
  \usepackage{color}
  \usepackage{amsmath}
  \usepackage{mathrsfs}
  \usepackage{listings}
  \usepackage{graphicx}
  \usepackage{xcolor}
  \usepackage{hyperref}

  \usepackage{minted}
  \usemintedstyle{monokai}



  
  
%}

\newcommand{\img}[4]{\center{\includegraphics[width=#1\linewidth]{#2}}\captionof{figure}{#3}\label{#4}}
\begin{document}
  \pagecolor{white}

  %----------------------------------------1
\begin{titlepage}
    \begin{center}
    \large
    Національний технічний університет України \\ "Київський політехнічний інститут імені Ігоря Сікорського"


    Факультет Електроніки

    Кафедра мікроелектроніки
    \vfill

    \textsc{ЗВІТ}\\

    {\Large Про виконання лабораторної роботи №1\\
    з дисципліни: «Мікропроцесори та мікроконтролери»\\[1cm]

    %Резистивні сенсори температури


    }
    \bigskip
    \end{center}
    \vfill

    \newlength{\ML}
    \settowidth{\ML}{«\underline{\hspace{0.4cm}}» \underline{\hspace{2cm}}}
    \hfill
    \begin{minipage}{1\textwidth}
    Виконавець:\\
    Студент 4-го курсу \hspace{4cm} $\underset{\text{(підпис)}}{\underline{\hspace{0.2\textwidth}}}$  \hspace{1cm}А.\,С.~Мнацаканов\\
    \vspace{1cm}

    Перевірив: \hspace{6.1cm} $\underset{\text{(підпис)}}{\underline{\hspace{0.2\textwidth}}}$  \hspace{1cm} Татарчук Д. Д.\\

    \end{minipage}

    \vfill

    \begin{center}
    2021
    \end{center}
\end{titlepage}



\newpage
\setcounter{page}{2}


\begin{minted}[bgcolor=black]{c}
import numpy as np
    
def incmatrix(genl1,genl2):
    m = len(genl1)
    n = len(genl2)
    M = None #to become the incidence matrix
    VT = np.zeros((n*m,1), int)  #dummy variable
    
    #compute the bitwise xor matrix
    M1 = bitxormatrix(genl1)
    M2 = np.triu(bitxormatrix(genl2),1) 

    for i in range(m-1):
        for j in range(i+1, m):
            [r,c] = np.where(M2 == M1[i,j])
            for k in range(len(r)):
                VT[(i)*n + r[k]] = 1;
                VT[(i)*n + c[k]] = 1;
                VT[(j)*n + r[k]] = 1;
                VT[(j)*n + c[k]] = 1;
                
                if M is None:
                    M = np.copy(VT)
                else:
                    M = np.concatenate((M, VT), 1)
                
                VT = np.zeros((n*m,1), int)
    
    return M
\end{minted}




\begin{center}
\textbf{Висновок}
\end{center}

Виходлячи з даних з рис. \ref{im1} видно, що при зменшеннi вiдстанi до джерела випромiнювання і збiльшеннi температури опiр зменшується, тобто можна сказати, що ТКО - вiд’ємний $\Rightarrow$ тип терморезистору - термiстор. 
На рис. \ref{im1} видно деяку нелiнiйнiсть,
проте вона наближається до прямолiнiйної залежностi, про що свiдчить симетричнiсть графiку вiдносно деякої точки перегину, через те, що є паралельно пiд’єднаний опір, номiнал якого рiвний термiстору за кiмнатної температури.




\end{document}