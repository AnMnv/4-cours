\documentclass[a4paper,14pt]{extreport}
\usepackage[left=1.5cm,right=1.5cm,
    top=1.5cm,bottom=2cm,bindingoffset=0cm]{geometry}
\usepackage{scrextend}
\usepackage[T1,T2A]{fontenc}
\usepackage[utf8]{inputenc}
\usepackage[english,russian,ukrainian]{babel}
\usepackage{tabularx}
\usepackage{amssymb}
\usepackage{color}
\usepackage{amsmath}
\usepackage{mathrsfs}
\usepackage{listings}
\usepackage{graphicx}
\graphicspath{ {./images/} }
\usepackage{lipsum}
\usepackage{xcolor}
\usepackage{hyperref}
\usepackage{tcolorbox}
\usepackage{tikz}
\usepackage[framemethod=TikZ]{mdframed}
\usepackage{wrapfig,boxedminipage,lipsum}
\mdfdefinestyle{MyFrame}{%
linecolor=blue,outerlinewidth=2pt,roundcorner=20pt,innertopmargin=\baselineskip,innerbottommargin=\baselineskip,innerrightmargin=20pt,innerleftmargin=20pt,backgroundcolor=gray!50!white}
 \usepackage{csvsimple}
 \usepackage{supertabular}
\usepackage{pdflscape}
\usepackage{fancyvrb}
%\usepackage{comment}
\definecolor{ggreen}{rgb}{0.4,1,0}
\definecolor{rred}{rgb}{1,0.1,0.1}
\usepackage{array,tabularx}
\usepackage{colortbl}

\usepackage{varwidth}
\tcbuselibrary{skins}
\usepackage{fancybox}




\usepackage{float}
\usepackage{wrapfig}
\usepackage{framed}
%for nice Code{
\lstdefinestyle{customc}{
  belowcaptionskip=1\baselineskip,
  breaklines=true,
  frame=L,
  xleftmargin=\parindent,
  language=C,
  showstringspaces=false,
  basicstyle=\small\ttfamily,
  keywordstyle=\bfseries\color{green!40!black},
  commentstyle=\itshape\color{purple!40!black},
  identifierstyle=\color{blue},
  stringstyle=\color{orange},
}
\lstset{escapechar=@,style=customc}
%}


\begin{document}
\pagecolor{white}
\begin{titlepage}
  \begin{center}
    \large
    Національний технічний університет України \\ "Київський політехнічний інститут імені Ігоря Сікорського"


    Факультет Електроніки

    Кафедра мікроелектроніки
    \vfill

    \textsc{ЗВІТ}\\

    {\Large Про виконання лабораторної роботи №1\\
      з дисципліни: «Алгоритми та структури даних-2»\\[1cm]

        Робота з файлами в С++


    }
  \bigskip
\end{center}
\vfill

\newlength{\ML}
\settowidth{\ML}{«\underline{\hspace{0.4cm}}» \underline{\hspace{2cm}}}
\hfill
\begin{minipage}{1\textwidth}
Виконавець:\\
Студент 3-го курсу \hspace{4cm} $\underset{\text{(підпис)}}{\underline{\hspace{0.2\textwidth}}}$  \hspace{1cm}А.\,С.~Мнацаканов\\
\vspace{1cm}

Перевірив: \hspace{6.1cm} $\underset{\text{(підпис)}}{\underline{\hspace{0.2\textwidth}}}$  \hspace{1cm}Д.\,Д.~Татарчук\\

\end{minipage}

\vfill

\begin{center}
2020
\end{center}
\end{titlepage}
%--------------------------------1-------------------------------
%\begin{center}\fcolorbox{black}{ggreen}{Варіант № 5}\end{center}
\textbf{Мета роботи} – вивчити особливості роботи з файлами в С++. Підготувати файл для виконання наступних робіт.\\

\textbf{Завдання}\\
Написати програму, що виконує наступні дії:\\

1) Обрати з таблиці 1 функцію згідно варіанту.\\

2) Обчислити значення функції f(x) на інтервалі $x\in \left[0, \dfrac{\pi}{4}\right]$ з кроком $\dfrac{\pi}{40}$.\\

3) Обчислені значення зберегти у вигляді файлу, в якому кожній точці відповідає пара чисел x f(x). Ім’я файлу сформувати наступним чином. Перші чотири символи – назва групи латинськими літерами, наступні два символи – варіант. Наприклад DP6102 – група ДП61, другий варіант.\\

4) Отриманий файл зберегти для виконання наступних робіт.\\

\begin{center}
\begin{tabular}{|c|c|}
\hline
Варіант & Функція \\
\hline
5 &  sin(|x|)\\
\hline

\end{tabular}
\end{center}

\vspace{0.3cm}
\begin{center}\textbf{Виконання роботи}\end{center}
\textbf{Код на С++}\\

\lstinputlisting[language=C++]{Laba-1.cpp}

\begin{lstlisting}
*dim,m,table,2,1, ,time
m(1,0,1) = 0.0001
m(1,1,1) = 0
m(2,0,1) = time
m(2,1,1) = time*speed

d,p51x, , %m% , , , ,ux,, , , ,

/solve
\end{lstlisting}

















\end{document}
