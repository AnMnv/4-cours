\documentclass[a4paper,14pt]{extreport}
\usepackage[left=1.5cm,right=1.5cm,
    top=1.5cm,bottom=2cm,bindingoffset=0cm]{geometry}
\usepackage{scrextend}
\usepackage[T1,T2A]{fontenc}
\usepackage[utf8]{inputenc}
\usepackage[english,russian,ukrainian]{babel}
\usepackage{tabularx}
\usepackage{amssymb}
\usepackage{color}
\usepackage{amsmath}
\usepackage{mathrsfs}
\usepackage{listings}
\usepackage{graphicx}
\graphicspath{ {./images/} }
\usepackage{lipsum}
\usepackage{xcolor}
\usepackage{hyperref}
\usepackage{tcolorbox}
\usepackage{tikz}
\usepackage[framemethod=TikZ]{mdframed}
\usepackage{wrapfig,boxedminipage,lipsum}
\mdfdefinestyle{MyFrame}{%
linecolor=blue,outerlinewidth=2pt,roundcorner=20pt,innertopmargin=\baselineskip,innerbottommargin=\baselineskip,innerrightmargin=20pt,innerleftmargin=20pt,backgroundcolor=gray!50!white}
 \usepackage{csvsimple}
 \usepackage{supertabular}
\usepackage{pdflscape}
\usepackage{fancyvrb}
%\usepackage{comment}
\definecolor{ggreen}{rgb}{0.4,1,0}
\definecolor{rred}{rgb}{1,0.1,0.1}
\usepackage{array,tabularx}
\usepackage{colortbl}

\usepackage{varwidth}
\tcbuselibrary{skins}
\usepackage{fancybox}


\usepackage[framemethod=TikZ]{mdframed}
\usetikzlibrary{calc}
\makeatletter
\newlength{\mylength}
\xdef\CircleFactor{1.1}
\setlength\mylength{\dimexpr\f@size pt}
\newsavebox{\mybox}
\newcommand*\circled[2][draw=blue]{\savebox\mybox{\vbox{\vphantom{WL1/}#1}}\setlength\mylength{\dimexpr\CircleFactor\dimexpr\ht\mybox+\dp\mybox\relax\relax}\tikzset{mystyle/.style={circle,#1,minimum height={\mylength}}}
\tikz[baseline=(char.base)]
\node[mystyle] (char) {#2};}
\makeatother

\definecolor{amber}{rgb}{1.0, 0.75, 0.0}
\definecolor{babyblue}{rgb}{0.54, 0.81, 0.94}

\usepackage{float}
\usepackage{wrapfig}
\usepackage{framed}
%for nice Code{
\lstdefinestyle{customc}{
  belowcaptionskip=1\baselineskip,
  breaklines=true,
  frame=L,
  xleftmargin=\parindent,
  language=C,
  showstringspaces=false,
  basicstyle=\small\ttfamily,
  keywordstyle=\bfseries\color{green!40!black},
  commentstyle=\itshape\color{purple!40!black},
  identifierstyle=\color{blue},
  stringstyle=\color{orange},
}
\lstset{escapechar=@,style=customc}
%}
\usepackage{longtable}
\usepackage{hhline}

\begin{document}
\pagecolor{white}

%----------------------------------------1
\newtcbox{\xmybox}[1][red]{on line,arc=7pt,colback=#1!10!white, colframe=#1!50!black, before upper={\rule[-3pt]{0pt}{10pt}},boxrule=1pt, boxsep=0pt,left=6pt,right=6pt,top=2pt,bottom=2pt}

\begin{center}\xmybox[amber]{Мнацаканов Антон Станіславович} \xmybox[amber]{ДП-82} \xmybox[amber]{Варіант №5} \end{center}

\begin{enumerate}
 \item відстань від цеху до місця аваріі (вибуху) -- 0.7 км
 \item тип вибуховоі речовини -- тротил
 \item маса вибуховоі речовини -- 200 т
 \item будівля (1-2-х поверхова) эі збірного залізобетону
 \item несучих стін -- 2,5
 \item несучих перегородок -- 0,25
 \item верстати -- легкі
 \item трубопроводи -- на естакадах
 \item кабельні лінії -- наземні
 \item контрольно-вимірювальна апаратура -- в наявності
 \item категорія виробництва з пожежноі безпеки -- 6
 \item щільнІсть эабудови -- 10 \%
\end{enumerate}


\begin{landscape}




 %Зона руйнування & Елементи цеху & Ступінь руйнування & Пожежна обстановка  & Ступінь ураженна людей \\
%\text{\hline
 %$\triangle P_{\text{ф}} = 14$ кПа& будівля, верстати, трубопроводи, кабельні лінії, \\ контрольно-вимірювальна апаратура & 1. слабкі (Руйнування заповнень дверних та віконних прорізей, зриваиня покрівлі даху) 2. середні (Пошкодження деформація основних деталей, електропроводки, приладів автоматики, тріщини в трубопроводах) 3. --  4. слабкі (Пошкодження окемих елемвнтів обладнання, важелів управління, вимірювальних приладів) 5. середні  & ІІ ступінь -- Окремі пожежі, що швидко перетворюються в суцільні та супроводжуються вибухами, руйнуванням виробничого устаткування та ін. & Легкі травми та пошкоджвння уламками зруйнованих конструкцій (обладнання)}




\begin{center}
  \begin{longtable}{|p{100pt}|p{130pt}|p{170pt}|p{140pt}|p{110pt}|}
   \hline
   Зона руйнування & Елементи цеху & Ступінь руйнування & Пожежна обстановка  & Ступінь ураженна людей\\
   \hline

   $\triangle P_{\text{ф}} = 14$ кПа & будівля, верстати, трубопроводи, кабельні лінії, контрольно-вимірювальна апаратура & 1. слабкі (Руйнування заповнень дверних та віконних прорізей, зриваиня покрівлі даху) 2. середні (Пошкодження деформація основних деталей, електропроводки, приладів автоматики, тріщини в трубопроводах) 3. --  4. слабкі (Пошкодження окемих елемвнтів обладнання, важелів управління, вимірювальних приладів) 5. середні  & ІІ ступінь -- Окремі пожежі, що швидко перетворюються в суцільні та супроводжуються вибухами, руйнуванням виробничого устаткування та ін. & Легкі травми та пошкоджвння уламками зруйнованих конструкцій (обладнання)\\
   \hline

   \end{longtable}
   \end{center}

\end{landscape}


Висновок: необхідно забезпечити всіх працівників захисним обладнанням (окулярами, рукавичками, шлемами, тощо) та перевіряти надійність закриплення та справність обраднання.

\end{document}
