\documentclass[a4paper,14pt]{extreport}
\usepackage[left=1.5cm,right=1.5cm,
    top=1.5cm,bottom=2cm,bindingoffset=0cm]{geometry}
\usepackage{scrextend}
\usepackage[T1,T2A]{fontenc}
\usepackage[utf8]{inputenc}
\usepackage[english,russian,ukrainian]{babel}
\usepackage{tabularx}
\usepackage{amssymb}
\usepackage{color}
\usepackage{amsmath}
\usepackage{mathrsfs}
\usepackage{listings}
\usepackage{graphicx}
\graphicspath{ {./images/} }
\usepackage{lipsum}
\usepackage{xcolor}
\usepackage{hyperref}
\usepackage{tcolorbox}
\usepackage{tikz}
\usepackage[framemethod=TikZ]{mdframed}
\usepackage{wrapfig,boxedminipage,lipsum}
\mdfdefinestyle{MyFrame}{%
linecolor=blue,outerlinewidth=2pt,roundcorner=20pt,innertopmargin=\baselineskip,innerbottommargin=\baselineskip,innerrightmargin=20pt,innerleftmargin=20pt,backgroundcolor=gray!50!white}
 \usepackage{csvsimple}
 \usepackage{supertabular}
\usepackage{pdflscape}
\usepackage{fancyvrb}
%\usepackage{comment}
\usepackage{array,tabularx}
\usepackage{colortbl}

\usepackage{varwidth}
\tcbuselibrary{skins}
\usepackage{fancybox}


\usepackage{tikz}
\usepackage[framemethod=TikZ]{mdframed}
\usepackage{xcolor}
\usetikzlibrary{calc}
\makeatletter
\newlength{\mylength}
\xdef\CircleFactor{1.1}
\setlength\mylength{\dimexpr\f@size pt}
\newsavebox{\mybox}
\newcommand*\circled[2][draw=blue]{\savebox\mybox{\vbox{\vphantom{WL1/}#1}}\setlength\mylength{\dimexpr\CircleFactor\dimexpr\ht\mybox+\dp\mybox\relax\relax}\tikzset{mystyle/.style={circle,#1,minimum height={\mylength}}}
\tikz[baseline=(char.base)]
\node[mystyle] (char) {#2};}
\makeatother

\definecolor{ggreen}{rgb}{0.4,1,0}
\definecolor{rred}{rgb}{1,0.1,0.1}
\definecolor{amber}{rgb}{1.0, 0.75, 0.0}
\definecolor{babyblue}{rgb}{0.54, 0.81, 0.94}
\definecolor{asparagus}{rgb}{0.53, 0.66, 0.42}
\definecolor{chartreuse}{rgb}{0.5, 1.0, 0.0}
\definecolor{darkorchid}{rgb}{0.6, 0.2, 0.8}

\usepackage{float}
\usepackage{wrapfig}
\usepackage{framed}
%for nice Code{
\lstdefinestyle{customc}{
  belowcaptionskip=1\baselineskip,
  breaklines=true,
  frame=L,
  xleftmargin=\parindent,
  language=C,
  showstringspaces=false,
  basicstyle=\small\ttfamily,
  keywordstyle=\bfseries\color{green!40!black},
  commentstyle=\itshape\color{purple!40!black},
  identifierstyle=\color{blue},
  stringstyle=\color{orange},
}
\lstset{escapechar=@,style=customc}
%}


\begin{document}
\pagecolor{white}

%----------------------------------------1
\newtcbox{\xmybox}[1][red]{on line, arc=7pt,colback=#1!10!white,colframe=#1!50!black, before upper={\rule[3pt] {0pt}{10pt}},boxrule=1pt,boxsep=0pt,left=6pt,right=6pt,top=2pt,bottom=2pt}

\begin{center}\xmybox[amber]{Mnatsakanov Anton} \xmybox[amber]{DP-82} \xmybox[amber]{Variant №5}
\vspace{1cm}

\xmybox[darkorchid]{ФОТОНИ}
\end{center}
\vspace{0.2cm}


In addition to electrons, magnons and phonons, quanta of the electromagnetic field-photons can be excited and propagate in solid dielectrics.\\
A classic example of a boson is the photon, an electromagnetic wave that can propagate both in a vacuum and in a dielectric crystal. The photon, like the electron, demonstrates dualism, exhibiting at times the properties of a particle. In quantum systems the distribution is expressed as a function of energy, multiplicity of degeneracy and number of particles in the system. For particles whose number in any state is limited, there is a special case of statistics - this is the Bose-Einstein distribution (particles are called bosons). The corpuscular properties of the photon are characterized by the momentum, while the wave properties are characterized by the wave vector. They are related by de Broglie's relation: p = $\hbar$k, and this relation can also be read backwards: $\hbar$k = p.
The photon's spin is an integer, it is equal to one: so a photon can only be in two spin states: +1 and -1. The two spin states of the photon signify the right and left circular polarizations of the wave, respectively; this fact is important for understanding the electro-optical and magneto-optical effects. Also a photon inside a crystal is a quasi-particle because it depends on interaction with matter. In particular, the photon-quasiparticle has a modified relation between energy and momentum (dispersion), which is described by the refractive index of the material.\\

It is also interesting that the polariton is a special form of the photon in the crystal, brightly manifested near its resonance with the lattice vibrational optical mode in the crystal. The polariton can also arise as a superposition of an exciton and a photon.



\end{document}
