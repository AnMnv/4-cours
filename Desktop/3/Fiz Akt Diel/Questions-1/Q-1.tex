\documentclass[a4paper,14pt]{extreport}
\usepackage[left=1.5cm,right=1.5cm,
    top=1.5cm,bottom=1.5cm,bindingoffset=0cm]{geometry}
\usepackage{scrextend}
\usepackage[T1,T2A]{fontenc}
\usepackage[utf8]{inputenc}
\usepackage[english,russian,ukrainian]{babel}
\usepackage{tabularx}
\usepackage{amssymb}
\usepackage{color}
\usepackage{amsmath}
\usepackage{mathrsfs}
\usepackage{listings}
\usepackage{graphicx}
\graphicspath{ {./images/} }
\usepackage{lipsum}
\usepackage{xcolor}
\usepackage{hyperref}
\usepackage{tcolorbox}
\usepackage{tikz}
\usepackage[framemethod=TikZ]{mdframed}
\usepackage{wrapfig,boxedminipage,lipsum}
\mdfdefinestyle{MyFrame}{%
linecolor=blue,outerlinewidth=2pt,roundcorner=20pt,innertopmargin=\baselineskip,innerbottommargin=\baselineskip,innerrightmargin=20pt,innerleftmargin=20pt,backgroundcolor=gray!50!white}
 \usepackage{csvsimple}
 \usepackage{supertabular}
\usepackage{pdflscape}
\usepackage{fancyvrb}
%\usepackage{comment}
\definecolor{ggreen}{rgb}{0.4,1,0}
\definecolor{amber}{rgb}{1.0, 0.75, 0.0}
\definecolor{babyblue}{rgb}{0.54, 0.81, 0.94}
\usepackage{array,tabularx}
\usepackage{colortbl}

\usepackage{varwidth}
\tcbuselibrary{skins}
\usepackage{fancybox}

\usetikzlibrary{calc}
\makeatletter
\newlength{\mylength}
\xdef\CircleFactor{1.1}
\setlength\mylength{\dimexpr\f@size pt}
\newsavebox{\mybox}
\newcommand*\circled[2][draw=blue]{\savebox\mybox{\vbox{\vphantom{WL1/}#1}}\setlength\mylength{\dimexpr\CircleFactor\dimexpr\ht\mybox+\dp\mybox\relax\relax}\tikzset{mystyle/.style={circle,#1,minimum height={\mylength}}}
\tikz[baseline=(char.base)]
\node[mystyle] (char) {#2};}
\makeatother

\usepackage{float}
\usepackage{wrapfig}
\usepackage{framed}


\begin{document}
\pagecolor{white}

\fcolorbox{black}{ggreen}{Mnatsakanov Anton DP-82\hspace{2cm} Variant №5}\par
\vspace{1cm}

\fcolorbox{black}{amber}{1} Які явища виникають під впливом на діелектрик  зміни його температури?\par
\fcolorbox{black}{babyblue}{2} Як вибирати комірку Браве?\\
\vspace{1cm}


\begin{center}\circled[fill=amber,draw=black]{1}\end{center}
Some types of dielectrics, when exposed to temperature, can start to conduct electricity, in fact, thus becoming a conductor and often when reaching a peak temperature there is a \underline{breakdown} of the dielectric itself (note that the electrical conductivity increases very sharply up to 10 orders of magnitude). \\
It can also be argued that the \underline{conductivity} of a dielectric is entirely dependent on temperature, because as it increases, the thermal motion of atoms and molecules leads to the activation of new free carriers.\\
Also when a dielectric is heated or cooled asymmetrically the following phenomenon occurs which is called \underline{thermal conductivity} i.e. the transfer of heat through a given material and as a consequence (or not) there may be a phenomenon called \underline{thermal} \underline{deformation} which occurs due to asymmetric vibrations of atoms and ions.\\
In spontaneously polarized dielectric crystals, when heated, an interesting phenomenon can arise -- an \underline{electric voltage} which polarity changes depending on the heating or cooling of the crystal.


\begin{center}\circled[fill=babyblue,draw=black]{2}\end{center}
In general, the primitive Bravais cells (cell with a minimum volume) are those basic cells that allow us to classify crystals with crystallographic syngonies.\\
The three\footnote{The significance of the conditions is plotted in descending order.} main concepts that guided Bravais in selecting the cell:\\
\begin{itemize}
  \item symmetry of unit cell must correspond to highest symmetry of crystal;
  \item unit cell should have largest possible number of identical angles,
        or corners and edges;
  \item unit cell should have the minimal volume.
\end{itemize}
As I understand, from tables and literature the ideal Bravais cell consist of Cubic crystal system (a = b = c and $\alpha = \beta = \gamma = 90^{\circ}$).



\end{document}
