\documentclass[a4paper,14pt]{extreport}
\usepackage[left=1.5cm,right=1.5cm,
    top=1.5cm,bottom=2cm,bindingoffset=0cm]{geometry}
\usepackage{scrextend}
\usepackage[T1,T2A]{fontenc}
\usepackage[utf8]{inputenc}
\usepackage[english,russian,ukrainian]{babel}
\linespread{1.5}
\usepackage{tabularx}
\usepackage{amssymb}
\usepackage{color}
\usepackage{amsmath}
\usepackage{mathrsfs}
\usepackage{listings}
\usepackage{graphicx}
\graphicspath{ {./images/} }
\usepackage{lipsum}
\usepackage{xcolor}
\usepackage{hyperref}
\usepackage{tcolorbox}
\usepackage{tikz}
\usepackage[framemethod=TikZ]{mdframed}
\usepackage{wrapfig,boxedminipage,lipsum}
\mdfdefinestyle{MyFrame}{%
linecolor=blue,outerlinewidth=2pt,roundcorner=20pt,innertopmargin=\baselineskip,innerbottommargin=\baselineskip,innerrightmargin=20pt,innerleftmargin=20pt,backgroundcolor=gray!50!white}
 \usepackage{csvsimple}
 \usepackage{supertabular}
\usepackage{pdflscape}
\usepackage{fancyvrb}
%\usepackage{comment}
\usepackage{array,tabularx}
\usepackage{colortbl}

\usepackage{varwidth}
\tcbuselibrary{skins}
\usepackage{fancybox}


\usepackage{tikz}
\usepackage[framemethod=TikZ]{mdframed}
\usepackage{xcolor}
\usetikzlibrary{calc}
\makeatletter
\newlength{\mylength}
\xdef\CircleFactor{1.1}
\setlength\mylength{\dimexpr\f@size pt}
\newsavebox{\mybox}
\newcommand*\circled[2][draw=blue]{\savebox\mybox{\vbox{\vphantom{WL1/}#1}}\setlength\mylength{\dimexpr\CircleFactor\dimexpr\ht\mybox+\dp\mybox\relax\relax}\tikzset{mystyle/.style={circle,#1,minimum height={\mylength}}}
\tikz[baseline=(char.base)]
\node[mystyle] (char) {#2};}
\makeatother

\definecolor{ggreen}{rgb}{0.4,1,0}
\definecolor{rred}{rgb}{1,0.1,0.1}
\definecolor{amber}{rgb}{1.0, 0.75, 0.0}
\definecolor{babyblue}{rgb}{0.54, 0.81, 0.94}
\definecolor{amethyst}{rgb}{0.6, 0.4, 0.8}

\usepackage{float}
\usepackage{wrapfig}
\usepackage{framed}
%for nice Code{
\lstdefinestyle{customc}{
  belowcaptionskip=1\baselineskip,
  breaklines=true,
  frame=L,
  xleftmargin=\parindent,
  language=C,
  showstringspaces=false,
  basicstyle=\small\ttfamily,
  keywordstyle=\bfseries\color{green!40!black},
  commentstyle=\itshape\color{purple!40!black},
  identifierstyle=\color{blue},
  stringstyle=\color{orange},
}
\lstset{escapechar=@,style=customc}
%}


\begin{document}
\pagecolor{white}


%----------------------------------------1
\begin{titlepage}
  \begin{center}
    \large
    Національний технічний університет України \\ "Київський політехнічний інститут імені Ігоря Сікорського"


    Факультет Електроніки

    Кафедра мікроелектроніки
    \vfill

    \textsc{ЗВІТ}\\

    {\Large Про виконання курсової роботи \\
      з дисципліни: «Твердотільна електроніка-2»\\[1cm]

        Варіант №22


    }
  \bigskip
\end{center}
\vfill

\newlength{\ML}
\settowidth{\ML}{«\underline{\hspace{0.4cm}}» \underline{\hspace{2cm}}}
\hfill
\begin{minipage}{1\textwidth}
Виконавець:\\
Студент 3-го курсу \hspace{4cm} $\underset{\text{(підпис)}}{\underline{\hspace{0.2\textwidth}}}$  \hspace{1cm}Б.\,В.~Лищенко\\
\vspace{1cm}

Превірив: \hspace{6.1cm} $\underset{\text{(підпис)}}{\underline{\hspace{0.2\textwidth}}}$  \hspace{1cm}Л.\,М.~Королевич\\

\end{minipage}

\vfill

\begin{center}
2021
\end{center}
\end{titlepage}



\newpage
\setcounter{page}{2}
\begin{center}
    \textbf{Завдання}
\end{center}
Розрахувати порогові напруги транзисторів мікросхеми


\begin{center}
  \textbf{Виконання завдання}
\end{center}

Треба записати формулу для пошуку порогової напруги. За варіантом  у мене КЕФ, тому формула буде наступною:
\begin{equation}
U_{n o p}^{0}=\phi_{M S}-\dfrac{q \cdot N_{S S}}{C_{o x}}-2 \cdot \phi_{F}-\dfrac{\sqrt{2 \cdot q \cdot \varepsilon_{0} \cdot \varepsilon_{S} \cdot N_{B}}}{C_{o x}} \cdot \sqrt{\left|2 \cdot \phi_{F}+U_{n}\right|}
\end{equation}

У цій формулі
дано майже все, а точніше: $\quad N_{S S}=5,6 \cdot 10^{11} \text{см}^{-3}$
$\varepsilon_{0}=8,85 \cdot 10^{-14} \text{ }\Phi / \text{см}$
$q=1,6 \cdot 10^{-19}\text{Кл}$
$k_{B}=1,38 \cdot 10^{-23}$ Дж/К $, T=300 K,$ $ n_{i}=1,45 \cdot 10^{10} \\
\text{см}^{-3},$ $ \varepsilon_{S}=11.8,$ $ \rho = 3$  Ом$\cdot$м, $U_0 = -0,6 $ B, $U_1 = -0,6 $ B, $\mu_n = 1500 \text{ }\ddfrac{\text{см}^2}{B \cdot c}$


Питома ємність шукається як
\begin{equation}
C_{o x}=\varepsilon_{0} \cdot \varepsilon_{o x} / d_{o x}=\dfrac{8,85 \cdot 10^{-14} \cdot 3,9}{0,5\cdot 10^{-5}}=6,903 \cdot 10^{-8}\text{ } \dfrac{\Phi}{\text{см}^{2}}
\end{equation}

Рівень Фермі у об'ємі кремнію:
\begin{equation}
\phi_{F}=\left(\dfrac{k_{B} \cdot T}{q}\right) \cdot \ln \left(\dfrac{N_{B}}{n_{i}}\right)
\end{equation}




\vspace{0.5 cm}
$\sigma=\dfrac{1}{\rho}=q \cdot  N_{B} \cdot \mu_{n} \Rightarrow  N_{B}=\dfrac{1}{\rho \cdot q \cdot \mu_{n}}=\dfrac{1}{3 \cdot 1,6 \cdot 10^{-19} \cdot 1500} = 1,39 \cdot 10^{15} \text{ }$ см $^{-3}$
\vspace{0.5 cm}


Рівень Фермі тоді буде:
$$\phi_{F}=\left(\dfrac{k_{B} \cdot T}{q}\right) \cdot \ln \left(\dfrac{N_{B}}{n_{i}}\right)=\dfrac{1,38 \cdot 10^{-23} \cdot 300}{1,6 \cdot 10^{-19}} \cdot \ln \left(\dfrac{1,39 \cdot 10^{15}}{1,45 \cdot 10^{10}}\right)=0,297\text{ } B$$\\

Напруги між витоком і підкладкою для кожного транзистора, маємо за умовою,
що $U_0 = -0,6 $ B $U_1 = -6 $ B
3а умовою з +, але так як підкладка КЕФ, то беремо з мінусом.\\


Для $Т_1, T_2, T_3, T_6, T_8: U_{n}=0;\text{ }U_{\text {nop }}=-2,14 $ В\\
Для  $T_4, T_5, T_7: U_{n}=-0,6 B;\text{ } U_{\text {nop }} = -2,19$ В\\



Далі порахуємо «ідеальну» порогову напругу:
$$
U_{\text {\text{ідеал} nop }}=\left(U^{1}+U^{0}\right) / 2=(-6-0,6) / 2=-3,3 \mathrm{~B}
$$

Шукаємо абсолютні похибки:\\
$U_{n}=0$\\
$\Delta U_{n o p}=-3,3+2,41=-0,89 \mathrm{~B}$\\
$\delta=100 \cdot|0,89 / 2,41|=37 \%$\\
$U_{n}=-0,6$\\
$\Delta U_{n o p}=--3,3+2,19=-1,1 \mathrm{~B}$\\
$\delta=100 \cdot|-1,11 / 2,19|=50 \%$\\




Підлеговування треба, тому шукаємо дозу легування за ф-ю $D=\Delta U_{n o p} \cdot C_{o x}$\\
$U_{n}=0$\\
$D=0,89 \cdot 6,903 \cdot 10^{-8} \approx 0,06 \text{ } \text{мкКл} / \text{см}^{2}$\\
$U_{n}=-0,6$\\
$D=1,11 \cdot 6,903 \cdot 10^{-8} \approx 0,08 \text{ } \text{мкКл}  / \text{см}^{2}$\\


Ну і далі підлеговуємо. Для цього додаємо до обрахованої порогової доданок:\\
$U_{n}=0$\\
$U_{\text {nop }}^{\prime}=U_{\text {nop }}+\dfrac{D}{C_{o x}}=-2,41-\dfrac{0,06}{6,9 \cdot 10^{-8}}=-3,28 \mathrm{~B}$\\
$U_{n}=-0,6$\\
$U_{\text {nop }}^{\prime}=U_{\text {nop }}+\dfrac{D}{C_{o x}}=-2,19-\dfrac{0,08}{6,9 \cdot 10^{-8}}=-3,35 \mathrm{~B}$\\


Для того аби зекономити на процесі виготовлення, замість того аби робити два підлегування (з 0.06 і 0.08), можемо зробити одне, для чого візьмемо дозу 0.07, і знову порахуємо напруги (якщо похибка буде менше 10\%, то тоді так і залишаємо, якщо більше, то тоді робимо два підлегування).

$
\begin{array}{l}
U_{n}=0: \\
U_{n o p}^{\prime}=U_{n o p}+\dfrac{D_{c e p}}{C_{o x}}=-2,41-\dfrac{0,07}{6,903 \cdot 10^{-8}}=-3,43 \mathrm{~B} ; \\
\delta=100 \cdot|(-3,3+3,43) /(-3,43)| \approx 3,7 \% ; \\
U_{n}=-0,6: \\
U_{n o p}^{\prime}=U_{n o p}+\dfrac{D_{c e p} .}{C_{o x}}=-2,19-\dfrac{0,07}{6,903 \cdot 10^{-8}}=-3,21 \mathrm{~B} ; \\
\delta=100 \cdot|(-3,3+3,21) /(-3,21)| \approx 2,9 \% .
\end{array}
$


Похибка менше $10 \%$ для всіх трьох напруг, тобто достатньо і одного підлегування, що значно спростить технологію виготовлення.

\begin{center}
  \textbf{Висновок}
\end{center}
Стосовно легування, то доза легування не може бути від’ємною, але знак напруги визначатиметься від того, якою домішкою я буду підлеговувати. Тобто, у даннму випадку напруги були менші за «ідеальну» порогову напругу, тобто вони були недостатньо «електронні», якщо так можна сказати. Якби у мене порогова напруга була менша за ту, яка вийшла, тоді я мав би підлеговувати акцепторними домішками (p-тип), а оскільки навпаки, то треба n-тип. Поширеними є фосфор і мишьяк, але в даннму випадку обираю фосфор, оскільки він більш поширений.



  \begin{table}[h]
  \begin{center}
    \begin{tabular}{|c|c|c|}
    \hline
    Транзистор &Порогова напруга, \text{[B]} 	& D(фосфор),	\text{мкКл} / \text{см}$^{2}$ \\ \hline
    T1            & -3,43               	& 0,07	\\ \hline
    T2            & -3,43               	& 0,07	\\ \hline
    T3            & -3,43               	& 0,07	\\ \hline
    T4            & -3,21               	& 0,07	\\ \hline
    T5            & -3,21               	& 0,07	\\ \hline
    T6            & -3,43               	& 0,07	\\ \hline
    T7            & -3,21               	& 0,07	\\ \hline
    T8            & -3,43               	& 0,07	\\ \hline
    \end{tabular}
    \end{center}
  \end{table}
















\end{document}
