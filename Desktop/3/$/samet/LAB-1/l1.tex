\documentclass[a4paper,14pt]{extreport}
\usepackage[left=1.5cm,right=1.5cm,
    top=1.5cm,bottom=2cm,bindingoffset=0cm]{geometry}
\usepackage{scrextend}
\usepackage[T1,T2A]{fontenc}
\usepackage[utf8]{inputenc}
\usepackage[english,russian,ukrainian]{babel}
\usepackage{tabularx}
\usepackage{amssymb}
\usepackage{color}
\usepackage{amsmath}
\usepackage{mathrsfs}
\usepackage{listings}
\usepackage{graphicx}
\graphicspath{ {./images/} }
\usepackage{lipsum}
\usepackage{xcolor}
\usepackage{hyperref}
\usepackage{tcolorbox}
\usepackage{tikz}
\usepackage[framemethod=TikZ]{mdframed}
\usepackage{wrapfig,boxedminipage,lipsum}
\mdfdefinestyle{MyFrame}{%
linecolor=blue,outerlinewidth=2pt,roundcorner=20pt,innertopmargin=\baselineskip,innerbottommargin=\baselineskip,innerrightmargin=20pt,innerleftmargin=20pt,backgroundcolor=gray!50!white}
 \usepackage{csvsimple}
 \usepackage{supertabular}
\usepackage{pdflscape}
\usepackage{fancyvrb}
%\usepackage{comment}
\usepackage{array,tabularx}
\usepackage{colortbl}

\usepackage{varwidth}
\tcbuselibrary{skins}
\usepackage{fancybox}
\usepackage{multirow}


\usepackage{tikz}
\usepackage[framemethod=TikZ]{mdframed}
\usepackage{xcolor}
\usetikzlibrary{calc}
\makeatletter
\newlength{\mylength}
\xdef\CircleFactor{1.1}
\setlength\mylength{\dimexpr\f@size pt}
\newsavebox{\mybox}
\newcommand*\circled[2][draw=blue]{\savebox\mybox{\vbox{\vphantom{WL1/}#1}}\setlength\mylength{\dimexpr\CircleFactor\dimexpr\ht\mybox+\dp\mybox\relax\relax}\tikzset{mystyle/.style={circle,#1,minimum height={\mylength}}}
\tikz[baseline=(char.base)]
\node[mystyle] (char) {#2};}
\makeatother

\definecolor{ggreen}{rgb}{0.4,1,0}
\definecolor{rred}{rgb}{1,0.1,0.1}
\definecolor{amber}{rgb}{1.0, 0.75, 0.0}
\definecolor{babyblue}{rgb}{0.54, 0.81, 0.94}
\definecolor{amethyst}{rgb}{0.6, 0.4, 0.8}

\usepackage{float}
\usepackage{wrapfig}
\usepackage{framed}
%for nice Code{
\lstdefinestyle{customc}{
  belowcaptionskip=1\baselineskip,
  breaklines=true,
  frame=L,
  xleftmargin=\parindent,
  language=C,
  showstringspaces=false,
  basicstyle=\small\ttfamily,
  keywordstyle=\bfseries\color{green!40!black},
  commentstyle=\itshape\color{purple!40!black},
  identifierstyle=\color{blue},
  stringstyle=\color{orange},
}
\lstset{escapechar=@,style=customc}
%}


\begin{document}
\pagecolor{white}

%----------------------------------------1
\newtcbox{\xmybox}[1][red]{on line, arc=7pt,colback=#1!10!white,colframe=#1!50!black, before upper={\rule[-3pt]{0pt}{10pt}},boxrule=1pt, boxsep=0pt,left=6pt,right=6pt,top=2pt,bottom=2pt}

\begin{titlepage}
  \begin{center}
    \large
    Національний технічний університет України \\ "Київський політехнічний інститут імені Ігоря Сікорського"


    Факультет Електроніки

    Кафедра мікроелектроніки
    \vfill

    \textsc{ЗВІТ}\\

    {\Large Про виконання лабораторної роботи №1\\
      з дисципліни: «Охорона праці та цивільний захист»\\[1cm]

        «ДОСЛІДЖЕННЯ ПАРАМЕТРІВ ВИРОБНИЧОГО ШУМУ ТА ВИЗНАЧЕННЯ ЕФЕКТИВНОСТІ ЗВУКОІЗОЛЯЦІЇ»


    }
  \bigskip
\end{center}
\vfill

\newlength{\ML}
\settowidth{\ML}{«\underline{\hspace{0.4cm}}» \underline{\hspace{2cm}}}
\hfill
\begin{minipage}{1\textwidth}
Виконавець:\\
Студент 3-го курсу \hspace{4cm} $\underset{\text{(підпис)}}{\underline{\hspace{0.2\textwidth}}}$  \hspace{1cm}Х.\,С.~Язиджи\\
\vspace{1cm}

Перевірив: \hspace{6.1cm} $\underset{\text{(підпис)}}{\underline{\hspace{0.2\textwidth}}}$  \hspace{1cm}В.\,В.~Калінчик\\

\end{minipage}

\vfill

\begin{center}
2021
\end{center}
\end{titlepage}


\textbf{Мета роботи}: засвоїти методику вимірювання основних параметрів виробничого шуму, набути навичок і компетенції оцінювання виробничого шуму з точки зору санітарно-гігієнічних умов, ризиків і рівня безпеки праці; використовуючи положення законодавчих актів та нормативно-правових документів.\\


\textbf{Методика вимірювання та оцінювання шуму на робочих місцях та звукоізолюючих властивостей захисних засобів.}\\
Суть вимірювання шуму полягає у визначенні рівня звуку LА та рівнів звукових тисків LР у фіксованих смугах частот (звичайно, октавних) нормованого діапазону (20.. .10000 Гц).\\

Основний прилад для вимірювання шуму - шумовимірювач, датчиком якого є мікрофон. Звуковий тиск, що сприймається мембраною мікрофона, перетворюється в пропорційну йому змінну напругу і далі трансформується в значення.\\
Шум на робочих місцях вимірюється під час вмикання не менше ніж 2/3 діючих у приміщенні джерел шуму, які повинні працювати в нормальному режимі, характерному для даного приміщення. При проведенні вимірювань мікрофон слід розташовувати на висоті 1,5 м над рівнем підлоги чи робочого майданчика (якщо робота виконується стоячи) чи на висоті і відстані 15 см від вуха людини, на яку діє шум (якщо робота виконується сидячи чи лежачи). Мікрофон повинен бути зорієнтований у напрямку максимального рівня шуму та віддалений не менш ніж на 0,5 м від оператора, який проводить вимірювання. Якщо робоче місце не зафіксовано, то шум вимірюється в кількох характерних точках (не менше трьох).\\


\begin{center}\textbf{Хід роботи}\end{center}
\par
1.1. Підготувати джерела шуму.\\

1.2.Увімкнути перше джерело шуму.\\

1.3.Виміряти створюваний джерелом шуму $\mathrm{L}_{1}$ рівень звуку. Для цього на робочому місці на рівні вуха людини направити шумовимірювач В сторону джерела шуму. Отримані результати записати до таблиці Д2.1 результатів лабораторної роботи (додаток 2). Вимкнути джерело шуму $\mathrm{L}_{1} .$\\

1.4. Аналогічно п.п. $1.2-1.3$ виміряти рівень звуку, який створюється
джерелом шуму $L_{2},$ а потім $-L_{3} .$ Результати вимірювань занести в таблицю Д2.1 (додаток 2). \\

1.5. Також виміряти рівень звуку, який створюється одночасно такими комбінаціями джерел шуму $L_{1}+L_{2}, L_{1}+L_{3}, L_{2}+L_{3}, L_{1}+L_{2}+L_{3} .$ Результати вимірювання занести до таблиці Д2.1 (додаток 2).\\

1.6. Розрахувати сумарний рівень звуку методом енергетичного підсумовування результатів вимірювань рівнів звуку, який створюється кожним джерелом окремо (значення $L_{1}, L_{2}, L_{3},$) за допомогою номограми, яка дана у вигляді таблиці для спрощеного розрахунку суми рівня джерел:
\vspace{1cm}

\begin{tabular}{|c|c|c|c|c|c|c|c|c|c|c|c|c|c|}
\hline$L_{1}-L_{2}$ & 0 & 1 & 2 & 3 & 4 & 5 & 6 & 7 & 8 & 9 & 10 & 15 & 20 \\
\hline$\Delta L$ & 3,0 & 2,5 & 2,0 & 1,8 & 1,5 & 1,2 & 1,0 & 0,8 & 0,6 & 0,5 & 0,4 & 0,2 & 0 \\
\hline
\end{tabular}
\vspace{1cm}

1.7. 3 таблиці Д1 (додаток 1 ) обрати допустимі рівні звуків для робочих місць обраного виду трудової діяльності (наприклад, пов'язаний з майбутньою професією, або навчанням) та занести значення в табл. Д2.1.\\

1.8. Зробити висновок про відповідність результатів вимірювання рівнів звуку $L_{1}, L_{2}, L_{3}, L_{1+2}, L_{1+3}, L_{2+3}, L_{1+2+3}$ допустимим значенням згідно $Д \mathrm{CH}$ 3.3.6.037-99 «Санітарні норми виробничого шуму, ультразвуку та інфразвуку».\\

1.9. Обчислити абсолютну та відносну похибку розрахункових та виміряних значень сумарних рівнів звуку. Результати занести до таблиці Д2.1. Зробити висновок про точність методу енергетичного підсумовування рівнів звуку, що створюються різними джерелами.








\begin{center}
  Формули для обрахунку похибок
\end{center}

Абсолютна похибка $\Delta$:
$$
\Delta = \text{ } \mid X_{\text {вим }}-X_{\text {дійсне }} \mid
$$
Відносна похибка $\sigma$:
$$
\sigma=\pm \frac{\Delta}{X_{\text {дійсне }}} \cdot 100 \%
$$









\newpage
\begin{landscape}
\begin{table}[h]
\begin{tabular}{|c|c|c|c|c|c|c|c|c|}
\hline
\multirow{2}{*}{}      & \multirow{2}{*}{\begin{tabular}[c]{@{}c@{}}Джерело\\ шуму\end{tabular}} & \multicolumn{4}{c|}{Рівень звуку, дБА}                   & \multicolumn{2}{c|}{Похибка}
& \multirow{2}{*}{\begin{tabular}[c]{@{}c@{}}Висновок про\\ точність  методу \\ вимірювання\end{tabular}} \\ \cline{3-8}
                       &                                                                         & Експеримент & Розрахунок &  \begin{tabular}[c]{@{}c@{}}Допустиме\\ значення\end{tabular} & Висновок & абсолютна, дБА & відносна, \% &                         \\ \hline
п.2.5                  & $L_1$                                                                       &      40       & X          & \multirow{7}{*}{20 - 120}  &   *      & X              & X            & X                                  \\ \cline{1-4} \cline{6-9}
\multirow{2}{*}{п.2.6} & $L_2$                                                                       &      45       & X          &                   &      *     & X              & X            & X                                        \\ \cline{2-4} \cline{6-9}
                       & $L_3$                                                                       &      62       & X          &                   &     **     & X              & X            & X                                   \\ \cline{1-4} \cline{6-9}
\multirow{4}{*}{п.2.7} & \multirow{4}{*}{\begin{tabular}[c]{@{}c@{}}$L_1+L_2$\\ $L_1+L_3$\\ $L_2+L_3$\\ $L_1+L_2+L_3$\end{tabular}}  &   54  & 41,2  &  &     *       &       0,8         &          1,5    &            ***       \\ \cline{3-4} \cline{6-9}
                       &                                                                               &     54      &    22        &                    &   *     &       0,4         &             0,7 &                                   ***                    \\ \cline{3-4} \cline{6-9}
                       &                                                                              &     59     &      17      &                    &      *     &    0,9            &             1,5 &                                   ***                            \\ \cline{3-4} \cline{6-9}
                       &                                                                              &     63     &      20,8      &                    &   **   &       1,2         &             1,9 &                                    ***                        \\ \hline
\end{tabular}
\end{table}
\par
*  -- допустиме значення для наукової діяльністі, конструювання, викладання, проектно-конструкторські бюро, програмування ЕОМ.\\

** -- допустиме значення для висококваліфікованої роботи, вимірювальна та аналітична робота в лабораторіях.\\

*** -- виходячи з того, що похибна не перевищує 2\% можна казати, що похибка є незначною.\\



\end{landscape}

\end{document}
