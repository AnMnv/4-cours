\documentclass[a4paper,14pt]{extreport}
\usepackage[left=1.5cm,right=1.5cm,
    top=1.5cm,bottom=2cm,bindingoffset=0cm]{geometry}
\usepackage{scrextend}
\usepackage[T1,T2A]{fontenc}
\usepackage[utf8]{inputenc}
\usepackage[english,russian,ukrainian]{babel}
\usepackage{tabularx}
\usepackage{amssymb}
\usepackage{color}
\usepackage{amsmath}
\usepackage{mathrsfs}
\usepackage{listings}
\usepackage{graphicx}
\graphicspath{ {./images/} }
\usepackage{lipsum}
\usepackage{xcolor}
\usepackage{hyperref}
\usepackage{tcolorbox}
\usepackage{tikz}
\usepackage[framemethod=TikZ]{mdframed}
\usepackage{wrapfig,boxedminipage,lipsum}
\mdfdefinestyle{MyFrame}{%
linecolor=blue,outerlinewidth=2pt,roundcorner=20pt,innertopmargin=\baselineskip,innerbottommargin=\baselineskip,innerrightmargin=20pt,innerleftmargin=20pt,backgroundcolor=gray!50!white}
 \usepackage{csvsimple}
 \usepackage{supertabular}
\usepackage{pdflscape}
\usepackage{fancyvrb}
%\usepackage{comment}
\definecolor{ggreen}{rgb}{0.4,1,0}
\definecolor{rred}{rgb}{1,0.1,0.1}
\usepackage{array,tabularx}
\usepackage{colortbl}

\usepackage{varwidth}
\tcbuselibrary{skins}
\usepackage{fancybox}

\definecolor{ggreen}{rgb}{0.4,1,0}
\definecolor{rred}{rgb}{1,0.1,0.1}
\definecolor{amber}{rgb}{1.0, 0.75, 0.0}
\definecolor{babyblue}{rgb}{0.54, 0.81, 0.94}
\definecolor{asparagus}{rgb}{0.53, 0.66, 0.42}
\definecolor{chartreuse}{rgb}{0.5, 1.0, 0.0}
\definecolor{darkorchid}{rgb}{0.6, 0.2, 0.8}
\usepackage{fp}

\usepackage{float}
\usepackage{wrapfig}
\usepackage{framed}
%for nice Code{
\lstdefinestyle{customc}{
  belowcaptionskip=1\baselineskip,
  breaklines=true,
  frame=L,
  xleftmargin=\parindent,
  language=C,
  showstringspaces=false,
  basicstyle=\small\ttfamily,
  keywordstyle=\bfseries\color{green!40!black},
  commentstyle=\itshape\color{purple!40!black},
  identifierstyle=\color{blue},
  stringstyle=\color{orange},
}
\lstset{escapechar=@,style=customc}
%}


\begin{document}

\newtcbox{\xmybox}[1][red]{on line, arc=7pt,colback=#1!10!white,colframe=#1!50!black, before upper={\rule[3pt] {0pt}{10pt}},boxrule=1pt,boxsep=0pt,left=6pt,right=6pt,top=2pt,bottom=2pt}


\pagecolor{white}
\begin{titlepage}
    \begin{center}
      \large
      Національний технічний університет України \\ "Київський політехнічний інститут імені Ігоря Сікорського"


      Факультет Електроніки

      Кафедра мікроелектроніки
      \vfill

      \textsc{ЗВІТ}\\

      {\Large Про виконання ІКР №4\\
        з дисципліни: «Твердотільна електроніка-2»\\[1cm]

        Пасивні елементи напівпровідникових
          ІМС. Дифузійні резистори.

      }
    \bigskip
  \end{center}
  \vfill

  \newlength{\ML}
  \settowidth{\ML}{«\underline{\hspace{0.4cm}}» \underline{\hspace{2cm}}}
  \hfill
  \begin{minipage}{1\textwidth}
  Виконавець:\\
  Студент 3-го курсу \hspace{4cm} $\underset{\text{(підпис)}}{\underline{\hspace{0.2\textwidth}}}$  \hspace{1cm}А.\,С.~Мнацаканов\\
  \vspace{1cm}

  Перевірив: \hspace{6.1cm} $\underset{\text{(підпис)}}{\underline{\hspace{0.2\textwidth}}}$  \hspace{1cm}О.\,В.~Мачулянський\\

  \end{minipage}

  \vfill

  \begin{center}
  2021
  \end{center}
\end{titlepage}


\begin{center}ЗАВДАННЯ\end{center}

\begin{enumerate}
  \item Ознайомитись з методикою розрахунку інтегральних ДР (методичні
  матеріали щодо вивчення теми «Пасивні елементи напівпровідникових
  ІМС. Напівпровідникові резистори»). Використовуючи вихідні дані
  наведені в таблиці, визначити геометричні розміри ДР (планарно-епітаксіальна технологія).
  \item Провести аналіз результатів оцінки параметрів ДР.
\end{enumerate}


\FPeval\R {600}
\FPeval\deltaR {20} %
\FPeval\ros {200}
\FPeval\P {3*10^{-3}}
\FPeval\Po {2}
\FPeval\dtrav {round((0.2*10^{-6}):8)}
\FPeval\dy {round((1.3*10^{-6}):8)} 

\FPeval\Nb {0}
\FPeval\kk {0.3}
\FPeval\deltab {0.1}
\FPeval\deltal {0.1}
\FPeval\deltaKK {0.14}
\FPeval\n {2}


\begin{center}
  \xmybox[green]{Визначення ширини резистора}
\end{center}

Відносна похибка коефіцієнта форми резистора, визначається як\\
 \FPeval\Kphi{round(\R/\ros:3)}
\begin{equation}
  K_{\phi} = \dfrac{R}{\rho_s} =   \FPprint\Kphi 
\end{equation}

Відносна похибка коефіцієнта форми резистора, яка визначається як
\begin{equation}
  \dfrac{\triangle K_{\phi}}{K_{\phi}} = \dfrac{\triangle R}{R} -
  \dfrac{\triangle \rho_s}{\rho_s} - \alpha_R \cdot \triangle T,
\end{equation}
де $\dfrac{\triangle \rho_s}{\rho_s} = 0,05$ -- похибка поверхневого опору, обумовлена можливими відхиленнями глибини базового шару і концентрації домішки в цьому шарі;\\
$\alpha_R = 10^{-3}$ -- температурний коефіцієнт опору резистора;
$\triangle T = -50^{\circ}$ -- діапазон температур. 


$$\dfrac{\triangle K_{\phi}}{K_{\phi}} = 0,2 - 0,05-0,01 = 0,14$$


Терпер знайдемо мінімальну ширину резистора, при якій 
забезпечується задана похибка геометричних розмірів\\
\begin{equation}
  b_{\text{точн}} = \left( \triangle b + \dfrac{\triangle l}{K_{\phi}}\right) \cdot \dfrac{K_{\phi}}{\triangle K_{\phi}}
\end{equation}

\FPeval\btochn{ round (\deltab + (\deltab/\Kphi)*(1/\deltaKK):3)}
$$b_{\text{точн}} = \left( \deltab + \dfrac{\deltab}{\Kphi}\right) \cdot \dfrac{1}{\deltaKK} = \btochn $$ мкм

Розрахуємо $b_{P}$ за формулою
\begin{equation}
  b_{P} = \sqrt{\dfrac{P}{P_0 \cdot K_{\phi}}},
\end{equation}
де $P_0$ -- максимально допустима питома потужність розсіювання.

\FPeval\bP{round( root(2,(\P/(\Po*\Kphi))) :3)}
$$b_{\text{P}} =  \sqrt{\dfrac{\P}{\Po \cdot \Kphi}} = \bP = 220 \text{ мкм}$$


Проміжне значення ширини резистора на фотошаблонах:
\begin{equation}
  b_{\text{пром}} = b_{\text{розр}} - 2\cdot (\triangle_{\text{трав}} + \triangle y)
\end{equation}

\FPeval\bP{round(220*10^{-6}:6)}

\FPeval\bprom{round( ((\bP)-(2*(\dtrav+\dy))) :8)}
$$ b_{\text{пром}} = \bP  - 2 \cdot (\dtrav +\dy)  = \bprom = 217 \text{мкм}$$

Оскільки $b_{\text{пром}} \ge b_{\text{техн}}$, то $b_{\text{топ}} = 220 $ мкм.\\

Реальна ширина резистора на кристалі визначається виразом:
\begin{equation}
  b = b_{\text{топ}} + 2\cdot (\triangle_{\text{трав}} + \triangle y)
\end{equation}
\FPeval\bprom{round( ((\bP)+(2*(\dtrav+\dy))) :6)}
$$ b = \bP  + 2 \cdot (\dtrav +\dy)  = \bprom = 223 \text{ мкм}$$


\begin{center}
  \xmybox[green]{Визначення довжини резистора}
\end{center}
Значення довжини дифузійного резистора визначається наступним чином:
\begin{equation}
  L_{\text{розр}} = b \cdot \left( \dfrac{R}{\rho_s} - n_1k_1 - n_2k_2 - 0,55\cdot N_B \right),
\end{equation}
де $n_1, n_2$ -- число контактних майданчиків.

\FPeval\lrozp{round(\bprom* (\R/\ros - 2*\n*\kk - 0.55*\Nb):7 )}
$$ L_{\text{розп}} = \bprom \cdot \left( \dfrac{\R}{\ros} - \n \kk - \n \kk - 0,55\cdot \Nb \right) = \lrozp  = 401 \text{ мкм}$$ 

Реальна довжина резистора визначається як:
\begin{equation}
  L =  L_{\text{топ}} - 2 \cdot (\triangle_{\text{трав}} + \triangle y),
\end{equation}
\FPeval\L{ round( \lrozp - 2*(\dtrav+\dy):8) }
$$ L = \lrozp - 2\cdot (\dtrav+\dy) = \L = 398 \text{ мкм}$$

Варто зазначити, що дані значення геометричних розмірів підлягають для 
виготовлення резистора, але сам технологічний процес вимагає достатньо високої технології виробництва, тобто $\dfrac{\rho_s}{\rho} = 0.05$.

%\FPeval\aaa{\P*\L*\bprom} 
%\aaa







%\begin{center}\fbox{\fcolorbox{black}{amber}{1}}\end{center}




%\clearpage
%\begin{center}\fbox{\fcolorbox{black}{babyblue}{2}}\end{center}




























\end{document}
