\documentclass[a4paper,14pt]{extreport}
\usepackage[left=1.5cm,right=1.5cm,
    top=1.5cm,bottom=2cm,bindingoffset=0cm]{geometry}
\usepackage{scrextend}
\usepackage[T1,T2A]{fontenc}
\usepackage[utf8]{inputenc}
\usepackage[english,russian,ukrainian]{babel}
\usepackage{tabularx}
\usepackage{amssymb}
\usepackage{color}
\usepackage{amsmath}
\usepackage{mathrsfs}
\usepackage{listings}
\usepackage{graphicx}
\graphicspath{ {./images/} }
\usepackage{lipsum}
\usepackage{xcolor}
\usepackage{hyperref}
\usepackage{tcolorbox}
\usepackage{tikz}
\usepackage[framemethod=TikZ]{mdframed}
\usepackage{wrapfig,boxedminipage,lipsum}
\mdfdefinestyle{MyFrame}{%
linecolor=blue,outerlinewidth=2pt,roundcorner=20pt,innertopmargin=\baselineskip,innerbottommargin=\baselineskip,innerrightmargin=20pt,innerleftmargin=20pt,backgroundcolor=gray!50!white}
 \usepackage{csvsimple}
 \usepackage{supertabular}
\usepackage{pdflscape}
\usepackage{fancyvrb}
%\usepackage{comment}
\usepackage{array,tabularx}
\usepackage{colortbl}

\usepackage{varwidth}
\tcbuselibrary{skins}
\usepackage{fancybox}


\usepackage{tikz}
\usepackage[framemethod=TikZ]{mdframed}
\usepackage{xcolor}
\usetikzlibrary{calc}
\makeatletter
\newlength{\mylength}
\xdef\CircleFactor{1.1}
\setlength\mylength{\dimexpr\f@size pt}
\newsavebox{\mybox}
\newcommand*\circled[2][draw=blue]{\savebox\mybox{\vbox{\vphantom{WL1/}#1}}\setlength\mylength{\dimexpr\CircleFactor\dimexpr\ht\mybox+\dp\mybox\relax\relax}\tikzset{mystyle/.style={circle,#1,minimum height={\mylength}}}
\tikz[baseline=(char.base)]
\node[mystyle] (char) {#2};}
\makeatother

\definecolor{ggreen}{rgb}{0.4,1,0}
\definecolor{rred}{rgb}{1,0.1,0.1}
\definecolor{amber}{rgb}{1.0, 0.75, 0.0}
\definecolor{babyblue}{rgb}{0.54, 0.81, 0.94}
\definecolor{amethyst}{rgb}{0.6, 0.4, 0.8}
\usepackage{graphicx}
\usepackage{float}
\usepackage{wrapfig}
\usepackage{framed}
%for nice Code{
\lstdefinestyle{customc}{
  belowcaptionskip=1\baselineskip,
  breaklines=true,
  frame=L,
  xleftmargin=\parindent,
  language=C,
  showstringspaces=false,
  basicstyle=\small\ttfamily,
  keywordstyle=\bfseries\color{green!40!black},
  commentstyle=\itshape\color{purple!40!black},
  identifierstyle=\color{blue},
  stringstyle=\color{orange},
}
\lstset{escapechar=@,style=customc}
%}


\begin{document}
\pagecolor{white}

%----------------------------------------1
\newtcbox{\xmybox}[1][red]{on line, arc=7pt,colback=#1!10!white,colframe=#1!50!black, before upper={\rule[-3pt]{0pt}{10pt}},boxrule=1pt, boxsep=0pt,left=6pt,right=6pt,top=2pt,bottom=2pt}

\begin{titlepage}
  \begin{center}
    \large
    Національний технічний університет України \\ "Київський політехнічний інститут імені Ігоря Сікорського"


    Факультет Електроніки

    Кафедра мікроелектроніки
    \vfill

    \textsc{ЗВІТ}\\

    {\Large Про виконання курсової роботи \\
      з дисципліни: «Твердотільна електроніка-2»\\[1cm]

        Варіант №50


    }
  \bigskip
\end{center}
\vfill

\newlength{\ML}
\settowidth{\ML}{«\underline{\hspace{0.4cm}}» \underline{\hspace{2cm}}}
\hfill
\begin{minipage}{1\textwidth}
Виконавець:\\
Студент 3-го курсу \hspace{4cm} $\underset{\text{(підпис)}}{\underline{\hspace{0.2\textwidth}}}$  \hspace{1cm}А.\,С.~Мнацаканов\\
\vspace{1cm}

Перевірив: \hspace{6.1cm} $\underset{\text{(підпис)}}{\underline{\hspace{0.2\textwidth}}}$  \hspace{1cm}Л.\,М.~Королевич\\

\end{minipage}

\vfill

\begin{center}
2021
\end{center}
\end{titlepage}

\begin{center}
    \textbf{Завдання}
\end{center}
Опис технології виготовлення


\begin{center}
  \textbf{Виконання завдання}
\end{center}

В своїй роботі маю КЕФ-5, що позначає пластини кремнію монокристалічного електронного типу провідності з додаванням фосфором і питомим опором 5 Ом$\cdot$см.\\

\begin{enumerate}
 \item Проведення підготовки: пластини кремнію шліфують до заданої товщини, потім полірують, піддають травленню і промивають. 



\item Перша фотолітографія дозволяє розкрити вікна в оксиді для локальної дифузії, в результаті якої формуються області витоку та стоку. Дифузія проводиться в дві
стадії на глибину 0,5 мкм.\\

\item Друга фотолітографія проводиться для розкриття вікон під тонкий оксид. Тонкий оксид вирощується на поверхні кремнію в сухому кисні при температурі 1150...1200$^{\circ}$ C.\\

\item Витравлювання оксиду до кремнію, де будуть знаходитися затвори транзистора.\\

\item Формування підзатворного діелектрика, розгонка.\\

\item Третя фотолітографія, тобто формування вікон для майбутніх контвктів.\\

\item Металізація, нанесення слою алюмінію за допомогою електровакуумного напилення, там де є області алюмінию та кремнію треба ці місця пролегувати $n^+ $ типом, тому що утвориться діод Шотткі.\\

\item Четверта фотолітогорафія, формування з'єднань на ІМС, формування стоку, витоку, затвору.\\

\item Хіміко-механічна планеризація, тобто видалення зайвих нерівностей та полірування.\\

\item Пасивація, утворення тонкого шару алюмінию для захисту від корозії\\

\item Остання фотолітографія -- відкриття контактних площадок.


\end{enumerate}
















\end{document}
