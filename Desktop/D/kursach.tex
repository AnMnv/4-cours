\documentclass[14pt,a4paper]{scrartcl}
\usepackage[utf8]{inputenc}
\usepackage[english,russian,ukrainian]{babel}
\usepackage{indentfirst}
\usepackage{misccorr}
\usepackage{geometry} \geometry{verbose,a4paper,tmargin=1cm,bmargin=2.5 cm,lmargin=2cm,rmargin=1cm}
\usepackage{graphicx}
\usepackage{xcolor}
\usepackage{cmap}
\usepackage{amsmath,amsfonts,amssymb,amsthm,mathtools}
\usepackage{icomma}
\usepackage{euscript}
\usepackage{mathrsfs}
\usepackage{mathtext}
\usepackage{graphicx}
\setlength{\parindent}{5ex}

 \begin{document}
\pagestyle{plain}
\linespread{1.3}
\renewcommand{\baselinestretch}{1.5}

\pagecolor{white}
\begin{titlepage}
  \begin{center}
    \large
    Національний технічний університет України \\ "Київський політехнічний інститут імені Ігоря Сікорського"
     
       
    Факультет Електроніки
     
    Кафедра мікроелектроніки
    \vfill
      
    
     
    {\Large КУРСОВА РОБОТА\\
      з дисципліни: «Теорія сигналів»\\[0.5 cm]
      Тема роботи:\\
    }
    {\bf Методи та застосування спектрального аналізу крові}\\
    
    
  \bigskip
\end{center}
\vfill
 
\newlength{\ML}
\settowidth{\ML}{«\underline{\hspace{0.4cm}}» \underline{\hspace{2cm}}}
\hfill
\begin{minipage}{1\textwidth}
Виконав:\\
Студент 3-го курсу \hspace{4cm} $\underset{\text{(підпис)}}{\underline{\hspace{0.2\textwidth}}}$  \hspace{1cm}Кузьмінський О.Р.\\
\vspace{1cm}

Перевірив: \hspace{6.1cm} $\underset{\text{(підпис)}}{\underline{\hspace{0.2\textwidth}}}$  \hspace{1 cm} Попов А.О.\\

\end{minipage}

\vfill

\begin{center}
2020
\end{center}
\end{titlepage}
%\includegraphics[width=0.5\textwidth]{lablub}

\tableofcontents

\newpage
\section*{1.Вступ}
\addcontentsline{toc}{section}{1.Вступ}
На сьогоднішній день ведення розмов про найрозповсюдженішу рідину в нашому тілі---кров- видається на рідкість звичайною справою. Переважна більшість людей на побутовому рівні знає її колір,смак, основну функцію яку вона виконує, усвідомлює її важливість для організму і те до чого може привести її нестача. Вченими та лікарями відкриті методи аналізу крові на предмет її групи та резус-фактору. В найзагальнішому випадку можна подумати що цим все і обмежується... Але незатухаюче періодичне погіршення якості життя та здоров'я населення змушує шукати нові методи якісного аналізу проблеми для якнайефективнішого її усунення. Звичайно це потребує удосконалення діагностики стану здоров'я недужих, і простим тестом на  групу крові тут не обмежитись.\\

 Для цього потрібно перейти мікромасштаб і зрозуміти молекулярну структуру речовини (крові): які клітини, речовини, хімічні сполуки входять у її склад; збільшена концентрація одних речовин покаже можливу боротьбу з певним типом недуги, що дасть змогу поставити правильний діагноз та призвести до швидкого та безболісного одужання хворого.\\

Однак постає питання, а як саме виявити ті чи інші сполуки крові? Відповідь на це питання було отримано в неявному вигляді ще в XVII столітті, коли було відкрито поняття {\bf спектру}- величини, що показує залежність інтенсивності світлового потоку від довжини хвилі. Пізніше ця концепція була розвинута в періоди астрономічних відкриттів, коли виявилося, що хімічні елементи таблиці Менделеєва мають свій особливий неповторний спектр, що дало змогу визначити хімічний склад інших планет і взагалі ввело поняття {\bf\itshape спектрального аналізу}- методу визначення хімічного складу речовини за відомими спектрами атомів, з яких вона складається.\\

То чому б не застосувати цей самий принцип для аналізу хімічного складу крові? Адже визначивши за спектрами атоми, можна визначити сполуки, що складаються з цих атомів і змоделювати якісну картину будови крові а в перспективі- спрогнозувати можливі захворювання ще до їх появи.\\
\newpage

\section*{2.Біологічні компоненти крові}
\addcontentsline{toc}{section}{2.Біологічні компоненти крові}
З шкільного курсу біології відомо, що кров складається з плазми-міжклітинної сполучної тканини, та власне ''крові'' --- так званих формених елементів- еритроцитів, лейкоцитів, тромбоцитів.

Один з головних носіїв інформації про кров- еритроцити, структурним елементом яких є відомий білок-{\bf гемоглобін} (надалі будемо позначати його як Hb), що зв'язується з киснем, що поступає з легеней і транспортує їх у всі віддалені куточки нашого організму. Також він виводить відпрацьований вуглекислий газ і регулює кислотно-основний стан крові.\\

Hb являє собою складний білок, білковий компонент якого представлений глобіном, небілковий- простетичною групою.\cite{l1} Простетична група в Hb представлена 4 однаковими залізопорфіриновими сполуками, які утворюють гем. Молекула гема складається з протопорфірина, що зв'язаний з залізом двома ковалентними та двома координаційними зв'язками через 4 атоми азоту. Атом заліза розташований в центрі гема (рис. 1) і надає крові характерний червоний відтінок.
\begin{center}
\small
\includegraphics[width=0.3\textwidth]{blood}\\[0.3 cm]
Рис.1. Будова гема\cite{l1}
\end{center}
Можна сказати, що Hb є лише всього однією з можливих комбінацій розташування глобіну. Існують також і інші. В тілі людини міститься мінімум 4 глобінні модифікації, основні з яких це:
оксигемоглобін $HbO_2$, карбоксигемоглобін  $HbCO$, метгемоглобін $HbMet$, фетальний гемоглобін $HbFetal$ та інші.\\

Як бачимо, кров- доволі складна, багатокомпонентна речовина,що потребує більш складного, скрупульозного аналізу, оскільки додаткові непотрібні нам для розгляду компоненти, що містяться в ній, можуть спотворити вихідну картину, тому розглянемо способи якісної оцінки компонентів крові з мінімальними похибками.
\newpage
\section*{3.Методи спектрального аналізу крові}
\addcontentsline{toc}{section}{3.Методи спектрального аналізу крові}
\subsection*{3.1. Метод Дервіза-Воробйова}
\addcontentsline{toc}{subsection}{3.1. Метод Дервіза-Воробйова}

Простим, швидким, економічним та безпечним методом визначення концентрації загального гемоглобіну є засвічення крові в слабкому розчині аміаку на фотометрі типу ГФ-Ц-04. Метод також називається {\bf оксигемоглобіновий}, оскільки в результаті взаємодії аміаку з кров'ю основною похідною гемоглобіну є оксигемоглобін $HbO_2$. \cite{l2} Час перебування в такому розчину складає декілька секунд. Засвічення фотометром відбувається в широкій спектральній смузі з максимумом пропускання по довжині хвилі 540 нм.\\

Однак цей метод має доволі високу похибку оскільки не враховує можливий великий вміст в крові метгемоглобіну $HbMet$. Річ у тім, що під час перебування крові в слабкому розчині аміаку не відбувається перетворення $HbMet$ в $HbO_2$, а коефіцієнт поглинання $HbMet$ на даній довжині хвилі набагато нижчий, ніж у $HbO_2$. Це призводить до значних методичних похибок, що досягають 37 \%. Окрім цього, розчин $HbO_2$ є нестійким і тому не може використовуватись для калібровки приладів за даної методики.\\

Тим не менш, в силу простоти методу, створені компактні портативні гемоглобінометри типу  АГФ-ОЗ/523, що широко використовуються в діагностичних та польових лабораторіях, клініках реанімації та машинах швидкої допомоги. Наведемо конструкцію та принцип роботи такого гемоглобінометра нижче.\\

Основними функціональними вузлами пристрою є оптичний блок та електрична плата управління та вимірювання. \cite{l3} Джерелом світла є напівпровідниковий світлодіод зеленого кольору з вузькою смугою випромінювання. Світловий пучок падає на оптичну кювету з біопробою, пройшовши яку, падає на світлофільтр, максимум спектральної кривої пропускання якого на довжині хвилі складає 523 нм. Нарешті світло потрапляє у фотоприймач (напівпровідниковий фотодіод), де світло перетворюється в електричний сигнал.\\

Електронна плата містить аналогову схему підсилення і перетворення фотоелектричного сигнала фотоприймача в цифровий сигнал,рідкокристалічний індикатор,а також мікропроцесорну систему управління та вимірювання, яка має власну енергонезалежну електроннну пам'ять. Для перетворення струму фотоприймача в цифрову форму застосовується 12-розрядний АЦП.\\

Конструктивно прилад являє собою малогабаритний блок. На верхній панелі розташовані табло-індикатор та фотометрична комірка. Для підключення джерела живлення на задній панелі є гніздо. Для виводу електричних параметрів приладу при його контролі або примусової корекції параметрів слугують дві кнопки L та R, які теж розташовані на задній панелі, на якій також розміщений пенал для зберігання оптичних кювет та контрольний світлофільтр.

\begin{center}
\small
\includegraphics[width=0.3\textwidth]{blood1}\\[0.3 cm]
Рис.2. Зовнішній вигляд гемоглобінометру  АГФ-ОЗ/523 \cite{l4}
\end{center}
\subsection*{3.2. Геміглобінціанідний метод}
\addcontentsline{toc}{subsection}{3.2. Геміглобінціанідний метод}
Принцип цього методу полягає в перетворенні усіх форм гемоглобіну $Hb$ в одну- геміглобіціанід $HbCN$. \cite{l1} Перетворення здійснюється при взаємодії $Hb$ з трансформуючим розчином, що у свою чергу містить наступні компоненти: фериціанід калію, ціанід калію, дигідрофосфат калію та неіонний детергент. Дигідрофосфат калію підтримує рівень pH, за якого реакція проходить за 3-5 хвилин. Детергент підсилює гемоліз (руйнування) еритроцитів і запобігає появи мутності, що пов'язана з білками плазми. Фериціанід калію окислює усі форми $Hb$ в метгемоглобін, який разом з ціаністим калієм утворює геміглобіціанід $HbCN$, що має червонуватий відтінок, інтенсивність якого прямо пропорційна концентрації $Hb$ в пробі.\\

Геміглобінціанідний метод був схвалений Міжнародним Комітетом по стандартизації в гематології (ICSH) в 1963 р. Міжнародний калібровочний розчин готується від імені ICSH з відносною похибкою до 0,2 \% і використовується для атестації комерційних калібровочних розчинів.\\

Експериментально цим методом користуються в дві стадії- спочатку готується {\itshape зразкова проба}, потім {\itshape холоста проба}. \cite{l5} \\

{\itshape Зразкова проба}:0,02 мл крові приливають до 5 мл трансформуючого розчину (розведення крові в 251 раз), добре перемішують, залишають на 10 хв, після чого суміш вимірюють на фотоелектроколориметрі при довжині хвилі 500-560 нм в кюветі з товщиною шару 1 см проти холостої проби.\\

{\itshape Холоста проба}: трансформуючий розчин або вода.\\

Стандартний розчин вимірюють за таких самих умовах, що і зразкову пробу.
Розраховують вміст гемоглобіну (в грам-відсотках) за калібровочним графіком, або за формулою:
\begin{equation}
Hb=\dfrac{E_{\text{зр}}}{E_{\text{ст}}}\times{C}\times{K}\times{0,001},
\end{equation}
де $E_{\text{зр}}$-екстинкція зразкової проби; $E_{\text{ст}}$-екстинкція стандартного розчину;\\
$C$-концентрація геміглобінціаніду в стандартному розчині в міліграм-відсотках; $K$-коефіцієнт розведення крові;\\
0,001-коефіцієнт для переведення міліграм-відсотки в грам-відсотки.\\

Основні переваги геміглобінціанідного методу \cite{l1} :
\begin{itemize}
\item $HbCN$ є стабільною похідною $Hb$, і усі форми $Hb$, окрім сульфогемоглобіну $HbS$ можуть бути легко і кількісно перетворені в $HbCN$;
\item спектр поглинання  $HbCN$ має максимум при $\lambda=540$ нм, тому достатня точність аналізу можлива при вимірюванні оптичної густини на фотометрах навіть з неінтерференціонними світлофільтрами;
\item розчини $HbCN$ строго підпорядковані закону Ламберта-Бера при $\lambda=540$ нм в широкому діапазоні концентрацій;
\item калібровочний розчини $HbCN$ зберігає свої властивості впродовж декількох місяців.
\end{itemize}

До недоліків методу можна віднести використання високотоксичних ціанідів і інших отруйних речовин.
\newpage
\subsection*{3.3. Геміхромний метод}
\addcontentsline{toc}{subsection}{3.3. Геміхромний метод}
На сьогоднішній час цей метод набуває стрімкої популярності, оскільки він має усі якості, притаманні геміглобінціанідному методу, але в той час не містить у своєму складі отруйних ціаністих сполук.\\

Принцип дії геміхромного методу заснований на перетворенні усіх форм $Hb$ в одну- геміхром $HbChr$. \cite{l1} При взаємодії $Hb$ з трансформуючим розчином відбувається його перетворення в окислену низькоспінову форму-геміхром, що має червонуватий відтінок, інтенсивність за довжини хвилі 540 нм якого прямо пропорційна концентрації $Hb$ в пробі.\\

Для перетворення $Hb$ в $HbChr$ використовують декілька трансформуючих реагентів:
\begin{itemize}
\item четвертинні амонієві сполуки з ПАР, загальна формула яких:
\begin{center}
\includegraphics[width=0.3\textwidth]{blood2} \cite{l1},
\end{center}
де: $R_1$-$R_3$-короткі алкільні групи, що містять 1-4 атоми вуглецю;\\[0.1 cm]
$R_4$-алкільна група, що містить 12-16 атомів вуглецю;\\[0.1 cm]
$X$-галоген.\\[0.3 cm]
\item солі жирних кислот з ферицитатом калію та EDTA(етилендіамінтетраоцетна кислота) в тріс-буфері (буферному розчині-речовині, що підтримує pH на сталому рівні);
\item натрію додецилсульфат (SDS);
\item інші реагенти, що не містять ціанідів.\\
\end{itemize}

З перерахованих вище реагентів найкращим є SDS. Максимальна тривалість перетворення $Hb$ в $HbChr$ при використанні цього реагенту складає до 5 хв. Висока швидкість реакції дозволяє використовувати даний метод в якості експрес-метода.
\newpage
Використання методу відбувається в декілька етапів. \cite{l5} \\

{\bf\itshape 1.Побудова калібровочних графіків}.\\

Перш за все готуються та вимірюються калібровочні розчини, по яким будуть працювати спектрофотометри. Для цього спочатку проводять ретельну чистку вимірювальної кювети, добиваючись аби її оптична густина з дистильованою водою відносно холостої кювети з дистильованою водою при $\lambda=540$ нм й оптичному шляху 10 мм дорівнювала нулю.\\

Далі, в змочену зразковим розчином кювету фотометра( з довжиною оптичного шляху 10 мм) вносять вміст ампул з калібровочним розчином. Оптичну густину розчинів вимірюють проти дистильованої води за $\lambda=540$ (520-560) нм за температури $18-25^{\circ}C$. На міліметрівці будують калібровочний графік залежності оптичної густини від концентрації $Hb$ по 4 точкам, що відповідають певній концентрації $Hb$. Калібровочний графік являє собою пряму лінію, що починається в точці (0;0). Калібровочний графік будують не рідше одного разу на тиждень.\\

{\bf\itshape 2.Проведення аналізу}.\\
Визначення концентрації гемоглобіну проводять за такою схемою:
\begin{center}
Табл.1.Схема визначення концентрації $Hb$
\begin{tabular}{|p{6cm}|p{2.5cm}|p{2.5cm}|p{3cm}|}
\hline
Розчини &Холоста проба, мкл& Зразкова проба, мкл &Контрольна проба, мкл\\
\hline
1.Кров&---&20&---\\
\hline
2. Контрольний розчин $Hb$&---&---&20\\
\hline
3.Трансформуючий розчин&5000&5000&5000\\
\hline
\end{tabular}
\end{center}
Зразки ретельно перемішують при досягненні гомогенного розчину й залишать розчин в спокої на 5 хв. Далі вимірюють оптичну густину зразкової проби на фотометрі в кюветі з довжиною оптичного шляху 10 мм за довжини хвилі 540 нм проти холостої проби. Забарвлення розчину стійке протягом 5 год.\\

{\bf\itshape 3.Розрахунок результатів}.\\

Концентрацію $Hb(C)$ в г/л розраховують або за калібровочним графіком або по фактору. Розглянемо спосіб розрахунку за графіком. На осі ординат знаходять точку, якій відповідає значення оптичної густини виміряної проби. З неї проводять перпендикуляр на калібровочний графік. З точки перетину на вісь абсцис опускають перпендикуляр, що відтинатиме значення, яке відповідатиме концентрації $Hb$ в пробі.
\newpage
 Основні переваги геміхромного методу \cite{l1} :
\begin{itemize}
\item Геміхром-стабільна похідна гемоглобіну, й усі форми $Hb$ можуть бути швидко і якісно перетворені в $HbChr$;
\item  максимум спектру поглинання $HbChr$ спостерігається при $\lambda=540$ нм, тому при вимірюванні оптичної густини на фотоелектроколориметрах точність не постраждає;
\item розчини $HbChr$ строго підпорядковані закону Ламберта-Бера при $\lambda=540$ нм в широкому діапазоні концентрацій;
\item калібровочний розчин $HbChr$ стійкий впродовж 1,5 роки;
\item трансформуючий реагент не є отруйним: в своєму складі він не має ціаністих й інших отруйних речовин.
\end{itemize}

Як бачимо, геміхромний метод є більш ефективним і головне безпечним, ніж геміглобінціанідний. Сьогодні він з широким успіхом застосовується в Японії та США.\\

\subsection*{3.4. Метод багатокомпонентного спектрального аналізу}
\addcontentsline{toc}{subsection}{3.4.Метод багатокомпонентного спектрального аналізу}

Досі ми розглядали однохвильові методи спектрального аналізу. Усі специфічні хімічні маніпуляції дозволяли створити умови, аби ми могли визначити за спектром одну кров'яну сполуку-гемоглобін, що не дуже раціонально. Було б добре виконувати спектральний аналіз цільного зразка крові, без складних хімічних перетворень.

Цю ідею уособлює {\bf метод багатокомпонентного спектрального аналізу}\cite{l6}. В його основі лежить закон Бугера-Ламберта-Бера, за яким спектр поглинання суміші являє собою суму спектрів її окремих компонентів. Тобто знаючи спектр крові і спектр поглинання її окремих компонентів (див. розділ 2.Біологічні компоненти крові), можна знайти концентрацію усіх її компонентів, розв'язавши математичну систему рівнянь типу:
\begin{equation}
A^{\lambda}=\varepsilon_1^{\lambda}C_1l+\varepsilon_2^{\lambda}C_2l+.......\varepsilon_n^{\lambda}C_nl,
\end{equation}
де: $A^{\lambda}$-це поглинання на довжині хвилі $\lambda$;\\
$\varepsilon_n^{\lambda}$-мілімолярна поглинальна здатність компоненту $n$ на довжині хвилі $\lambda$;\\
$C_n$-концентрація компоненту $n$. Фактично- це група змінних в системі\\
$l$-довжина світлового шляху.\\

\newpage
На прикладі ця система використовувалась для одночасного визначення гемоглобіну $Hb$, оксигемоглобіну $HbO_2$, карбоксигемоглобіну $HbCO$, метгемоглобіну $HbMet$ та сульфогемоглобіну $HbS$, вимірюючи поглинання на довжинах хвилі  500, 569, 577, 620, та 760 нм ( кожна довжина хвилі-це локальний максимум для однієї з представлених похідних глобіну). Математично, це була точно визначена або квадратна СЛАР, бо кількість рівнянь була рівній кількості невідомих (оскільки кожній похідній відповідало лише одне значення максимуму), тому неважко було знайти її розв'язок.\\

На практиці ж виявилось, що мілімолярні поглинальні здатності похідних на  500, 569, 577 нм і на 620, 760 нм виявились дуже різними, тому довелося використовувати дві різні довжини світлового шляху для збереження значення оптичної густини в межах певного діапазона, в якому вони могли бути виміряні з достатньою точністю. Окрім незручності, пов'язанної з використанням двох різних кювет обидві довжини повинні були відомі до чотирьох знаків після коми, навіть якщо треба було визначити пропорції, не кажучи вже про концентрації. Хоча це все ж не дуже серйозні труднощі, які вдалося обійти і навіть створити прилад {\bf IL282 CO-Oximeter} для автоматичного визначення в крові $Hb$, $HbO_2$, $HbCO$ та $HbMet$.\\

Як було сказано раніше, в квадратній системі рівнянь кожній похідній глобіну відповідало лише одне значення максимуму. Проте це доволі грубий спосіб оцінки концентрації кров'яних компонентів. Для отримання більш точних результатів, потрібно задавати декілька довжин хвиль для одної похідної. Звичайно чим більш дрібне буде це градуювання, тим краще. Але тут виникає проблема- поява "перевизначеної" СЛАР. Коли ми  збільшуємо і градуюємо діапазон значень довжин хвиль, ми тим сами збільшуєм кількість рівнянь, в той час як шукані параметри- похідні глобіну, вони ж-невідомі залишаються незмінними. І тому виходить, що кількість рівнянь набагато більша ніж кількість невідомих. Але на сьогоднішній день обчислювальна техніка дозволяє швидко і точно розв'язувати такого роду рівняння.\\

Машиною, що поєднувала в собі конструктивні особливості спектрофотометра і відповідне спеціальне програмне забезпечення для розв'язку складних математичних задач став спектрофотометр {\bf HP8450 A UV/Vis}, що здатний визначати в крові з великою точністю $Hb$, $HbO_2$, $HbCO$, $HbMet$ та $HbS$. Поглинання спектрів зразків відбувається в діапазоні 480-650 нм, з кроком 2 нм за одної довжини світлового шляху.
\newpage
Спектрофотометр {\bf HP8450 A UV/Vis} відрізняється від звичайного тим, що використовує обернену оптичну геометрію. Світло з комбінації дейтерієвої або вольфрамової лампи усе одразу проходить крізь кювет, а не розсіюється на монохроматорі, щоб потім бути пропущеним через зразок по одній довжині хвилі за один раз. Світло, що пройшло крізь кювет, розсіюється спектрографом і потрапляє до масиву паралельних детекторів, де вимірювання поглинання відбувається одночасно по всій спектральній області від 200 до 800 нм. В якості детекорів використовуються світлочутливі фотодіоди, які розміщуються поза спектрографом таким чином, щоб в залежності від розміщення діода, на нього потрапляла одна  хвиля певної довжини. Таким чином, спектральне розширення визначається відстанню між діодами. Шкала довжин хвиль {\bf HP8450 A UV/Vis} як було сказано раніше сягає від 200 до 800 нм. Його розширенння в 1 нм для діапазону 200-400 нм, і 2 нм для діапазону 400-800 нм стало можливим завдяки 400 фотодіодам.
\begin{center}
\small
\includegraphics[width=0.5\textwidth]{blood3}\\[0.3 cm]
Рис.3. Схематичне зображення звичайного спектрофотометра (a) та спектрофотометра з оберненою оптичною геометрією (b). \cite{l6}\\
L-джерело світла; M-монохроматор; S-спектрограф; C-кювета; D-детектор.
\end{center} 
Спектрофотометр має вбудований мікрокомп'ютер з оперативною пам'яттю 16 КБ, що здатна зберігати 50 спектрів в своєму робочому просторі. Для постійного зберігання спектри можна записувати на магнітну стрічку. Окрім цього, комп'ютер здатний виконувати математичні операції над спектрами й зберігати калібровочні лінії для подальшого розрахунку концентрацій по оптичній густині.\\

Спектрофотометр може визначити концентрації до 12 компонентів за одну маніпуляцію, якщо були внесені чисті спектри компонентів в базу мікрокомп'ютера. Спектри окремих компонентів порівнюються зі спектром вимірюваної суміші з допомогою матричної процедури розв'язку для перевизначеної системи. Концентрація кожного компонента, спектр якої якнайкраще збігається з еталонним, відображається на вбудованій електро-променевій трубці за декілька секунд.
\newpage
\begin{center}
\small
\includegraphics[width=0.35\textwidth]{blood4}\\[0.3 cm]
Рис.4. Експериментальний зразок спектрофотометра для багатокомпонентного аналізу, розроблений компанією "ТЕХНОМЕДИКА": внутрішня електронна начинка(a); в корпусі (б). \cite{l7}\\
\end{center} 

Хоча метод багатокомпонентного спектрального аналізу складний в технічному плані, але його головна перевага над усіма ішими- безреагентність та універсальність, що дозволяють уникнути помилок в преаналітичні та аналітичні фазах, й досягти високої точності в обчисленнях.
\newpage
\section*{4. Висновок}
\addcontentsline{toc}{section}{4.Висновок}
В роботі був проведений аналіз кожного з методів спектрального аналізу крові. Основне на що слід звернути увагу- дані методи в першу чергу призначені для визначення одного компонента еритроцитів крові- гемоглобіну і його похідних. Дійсно на сьогоднішній день це---найважливіший покажчик (разом з підрахунком кількості еритроцитів) для оцінки анемічних станів в клінічній практиці.\\

Не зважаючи на те,що деякі методи є застарілими, як наприклад метод Дервіза-Воробйова, в силу своєї простоти вони досі зберігають релевантність й використовуються як такі собі швидкі, "портативні" способи визначення гемоглобіну. Поряд з цим, більш прогресивний метод такий як геміглобінціанідний забезпечує високу якість вимірюваних результатів, але закономірним чином витісняється геміхромним, який має не тільки стабільніші реагенти, що забезпечують більшу ефективність, але ще й є абсолютно безпечним.\\

Розвиток обчислювальної техніки дозволив автоматизувати більшість процесів, й вінцем усіх досліджень та розробок, що поєднував у собі як теоричну базу в області спектрального аналізу й удосконалені математичні алгоритми став метод багатокомпонентного аналізу, який є набагато раціональнішим ніж усі вище перераховані методи, математично спрощена методика виміру якого й безреагентність дозволяють максимізувати точність вимірів й не покладатись на непередбачувані фактори, які могли б виникнути при взаємодії крові зі специфічними хімічними сполуками. Хоча технічна реалізація й може викликати певні труднощі, але розрахунок усіх компоненів крові за один раз, так би мовити, "не відходячи від каси" того варте...

\clearpage
\newpage
\begin{thebibliography}{9}

\bibitem{l1} Проф.Тогузов.Р.Т. Гемихромный метод определения гемоглобина в крови. Протокол № ЛД/7, 29.05.2002 г.


\bibitem{l2} Антонов В.С., Давыдов В.М., Ованесов Е.Н., Сецко И.В. Оптический метод определения концентрации гемоглобина с учётом его производных. НПП "Техномедика", г. Москва.
\bibitem{l3} Гемоглобинометры фотометрические портативные АГФ-03/523-"МИНИГЕМ". Описание типа СИ.

\bibitem{l4} ЗАО НПП "ТЕХНОМЕДИКА". Гемоглобинометр фотометрический потративный для измерения общего гемоглобина в крови, модернизированным методом Дервиза-Воробьева АГФ-03/523-"МИНИГЕМ". Руководство по эксплуатации.
\bibitem{l5} Проф. Кост.Е.А. Справочник по клиническим лабораторным методам исследования. Издание 2.

\bibitem{l6} Zwart A, Buursma A, van Kampen EJ, Zijlstra WG.Multicomponent analysis of hemoglobin derivatives with reversed-optics spectrophotometer. Clin Chem. 1984 Mar;30(3):373-9.

\bibitem{l7} Костюков Д.В., Лагутина Н. К., Павлушкина Л. В., Сецко И.В., Терешков В.П. Спектральные исследования плазмы и крови новорожденных.
\end{thebibliography}





\end{document}