\documentclass[14pt,a4paper]{scrartcl}
\usepackage[left=1.5cm,right=1.5cm,
    top=1.5cm,bottom=1cm,bindingoffset=0cm]{geometry}

\usepackage[T1,T2A]{fontenc}
\usepackage[utf8]{inputenc}
\usepackage[english,russian,ukrainian]{babel}
\usepackage{tabularx}
\usepackage{amssymb}
\usepackage{color}
\usepackage{amsmath}
\usepackage{mathrsfs}
\usepackage{listings}
\usepackage{graphicx}
\graphicspath{ {./images/} }
%\usepackage{draftwatermark} не будет лезть на картинки
\usepackage[printwatermark]{xwatermark}%будет лезть на картинки
\usepackage{lipsum}
\usepackage{xcolor}
\usepackage{tikz}

 \usepackage{csvsimple}
 \usepackage{supertabular}
\usepackage{pdflscape}
\usepackage{fancyvrb}
\usepackage{comment}



\begin{document}
\pagecolor{white}
\begin{titlepage}
  \begin{center}
    \large
    Національний технічний університет України \\ "Київський політехнічний інститут імені Ігоря Сікорського"
     
       
    Факультет Електроніки
     
    Кафедра мікроелектроніки
    \vfill
      
    \textsc{ЗВІТ}\\
     
    {\Large Про виконання Модульної контрольної роботи №1\\
      з дисципліни: «Алгоритми та структура даних-1»\\[1cm]
    
    }
  \bigskip
\end{center}
\vfill
 
\newlength{\ML}
\settowidth{\ML}{«\underline{\hspace{0.4cm}}» \underline{\hspace{2cm}}}
\hfill
\begin{minipage}{1\textwidth}
Виконавець:\\
Студент 3-го курсу \hspace{4cm} $\underset{\text{(підпис)}}{\underline{\hspace{0.2\textwidth}}}$  \hspace{1cm}А.\,С.~Мнацаканов\\
\vspace{1cm}

Превірив: \hspace{6.1cm} $\underset{\text{(підпис)}}{\underline{\hspace{0.2\textwidth}}}$  \hspace{1cm}Д.\,Д.~Татарчук\\

\end{minipage}

\vfill

\begin{center}
2020
\end{center}
\end{titlepage}

%---------------------------------------------------------------------------------------------------------------------------------------------------------------------------------
\textbf{6. Як виконується сортування за допомогою вибору?}\\

Сортування – це процес перегрупування даних у деякому заданому
порядку. Основна мета сортування – полегшити пошук потрібної інформації у
заданій послідовності даних.\\

Сортування за допомогою прямого вибору базується на нижчеперелічених
операціях:\\

1. Обирається елемент з найменшим значенням.\\

2. Цей елемент обмінюється місцями з першим елементом.\\

3. Потім п.1-2 повторюються з елементами від 2-го до n-го, потім від 3-го до n-го і т. д.\\

\textbf{6. Як описати змінну строкового типу?}\\

Опис змінної строкового типу має вигляд:
\begin{verbatim}
<змінна>:string[довжина строки];\\
\end{verbatim}

Довжина строки – це число, що вказує, яку максимальну кількість
символів можна буде зберігати у вказаній змінній. Це число не може
перевищувати 255. При описі змінної строкового типу довжину строки можна
не вказувати. В такому випадку довжина строки буде 255 символів. Як уже
відмічалось довжина строки вказує яку максимальну кількість символів можна
буде зберігати у заданій змінній. Якщо спробувати записати більшу кількість
символів, то всі зайві символи буде відкинуто без всякого попередження, тому
програміст повинен сам контролювати у яку змінну строкового типу скільки
елементів можна записати, також для полегшення роботи із даними строкового типу існує набір стандартних
процедур та функцій. Розглянемо основні з них:
concat(s1 [,s2,…,sn]) – функція типу string, що повертає строкову змінну,
яка містить у собі послідовно строки s1…sn; copy(st, index, count) – функція типу string, що повертає строкову змінну,
яка містить count послідовних символів строки st, починаючи з елемента зномером index та інші. \\


\textbf{6. Що таке поле запису?}\\

Записи з даними йдуть за заголовком (байти розташовуються послідовно) і містять у собі фактичний вміст полів. Довжина запису (у байтах) визначається підсумовуванням зазначених довжин усіх полів. Числа в даному файлі розміщуються в зворотному порядку.\\

\textbf{6. Що таке файл?}\\

Файл у Паскалі – це або поіменована область зовнішньої пам’яті
ПК(диску, дискети , “віртуального” диску), або пристрій – носій інформації.\\

\textbf{6. Які є у Паскалі процедури та функції для роботи з нетипізованими файлами?}\\

Нетипізовані файли – це файли, до яких інформація записується без урахування типу. Фактично – це канали вводу-виводу інформації нижнього рівня, що можуть використовуватись для прямого доступу до будь-якого файлу без урахування його типу та структури.\\

При роботі з нетипізованими файлами можуть бути використані всі
процедури і функції, що використовуються при роботі з типізованими файлами,
крім read та write. Замість них при роботі з нетипізованими файлами
використовуються процедури blockread та blockwrite:\\
blockread(<файлова змінна>, <буфер>, <кількість записів> [,<результат>]) –
процедура призначена для зчитуання інформації з нетипізованого файлу.\\
blockwrite(<файлова змінна>, <буфер>, <кількість записів> [,<результат>]) –
процедура призначена для запису інформації до нетипізованого файлу.\\











\end{document}