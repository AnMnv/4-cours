% Copyright (c) 2020 CAEmate S.r.l.
% All rights reserved.
% ------------------------------------
%
% NOTICE:  All information contained herein is, and remains the property of CAEmate S.r.l.
% The intellectual and technical concepts contained herein are proprietary to CAEmate S.r.l.
% and are protected by trade secret or copyright law.
% Dissemination of this information or reproduction of this material is strictly forbidden
% unless prior written permission is obtained from CAEmate S.r.l.

The materials used in the present project are described in the following paragraphs.
For more specific information about the material mechanical properties, please see the dedicated paragraph.

\FloatBarrier
\needspace{.5\textheight}
\label{section_ConcreteDescription}
\subsection{Concrete}
{{westatix: materials.description.concrete}}
\FloatBarrier
\subsubsection{Specific notes regarding the current project}
The model takes into account nonlinear effects due to concrete creep and shrinkage, depending on the different age of the single structural elements. The material parameters may vary, as a result of continous calibration of the digital twin.\par
\FloatBarrier
\subsubsection{Concrete properties}
{{westatix: materials.table.concrete}}


\FloatBarrier
\newpage
\label{section_ReinforcementSteelDescription}
\subsection{Reinforcement Steel}
{{westatix: materials.description.reinforcementsteel}}
\FloatBarrier
\subsubsection{Reinforcement steel properties}
{{westatix: materials.table.reinforcementsteel}}


\FloatBarrier
\newpage
\label{section_PrestressingSteelDescription}
\subsection{Prestressing Steel}
{{westatix: materials.description.prestressingsteel}}
\subsubsection{Prestressing steel properties}
{{westatix: materials.table.prestressingsteel}}

