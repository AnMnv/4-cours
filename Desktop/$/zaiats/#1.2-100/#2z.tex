\documentclass[a4paper,12pt]{article}
\usepackage[left=1.5cm,right=1.5cm,
    top=1.5cm,bottom=1.5cm,bindingoffset=0cm]{geometry}
    
\usepackage[warn]{mathtext}
\usepackage[T1,T2A]{fontenc}
\usepackage[utf8]{inputenc}
\usepackage[english,russian,ukrainian]{babel}
\usepackage{tabularx}
\usepackage{amssymb}
\usepackage{color}
\usepackage{amsmath}
\usepackage{mathrsfs}
\usepackage{listings}
\usepackage{graphicx}
\graphicspath{ {./images/} }
%\usepackage{draftwatermark} не будет лезть на картинки
\usepackage[printwatermark]{xwatermark}%будет лезть на картинки
\usepackage{lipsum}
\usepackage{xcolor}
\usepackage{tikz}


\definecolor{lgreen}{rgb}{0.5,1,1}
\definecolor{n}{rgb}{1,0.5,0.5}
\definecolor{n1}{rgb}{1,1,0.5}
\definecolor{n3}{rgb}{1,0.7,0.9}




\begin{document}
\pagecolor{white}
\begin{center}
\begin{Large}
Заєць Дар'я ДМ-81\\
Варіант №4\\

\vspace{0.2cm}
Практична робота №1\\
\end{Large}
\end{center}

\large
\vspace{0.3cm}



\textbf{Завдання:}\\

Розрахувати провідність кремнію. Парні варіанти (номер за списком) концентрація донорної домішки = №варіанту*$10^{14} \text{см}^{-3}$.\\
\vspace{0.3cm}

\textbf{Вихідні дані:}\\

Концентрація донорної домішки: $N^+_d =4\cdot10^{14} (\text{см$^{-3}$})$ ,\\

рухливість носіїв заряду (в кремнії): $\mu_n = 1500\left(\dfrac{\text{см$^2$}}{B\cdot c}\right), \mu_p = 450 \left(\dfrac{\text{см$^2$}}{B\cdot c}\right)$\\

одиничний заряд: $q = 1.6\cdot 10^{-19}$(Кл)\\

Спочатку виведемо рівняння  електронейтральності, використовуючи умову електронейтральності, яка полягає в тому, що сумарний заряд у напівпровіднику має дорівнювати нулю:
\begin{equation}
\sum\limits_{i=1}^n q^-_{i}+ \sum\limits_{j=1}^m q^+_{j} = 0
\label{eq:ref}
\end{equation}

Запишемо (1) вираз через об’ємний заряд Q :
\begin{equation}
Q^-+ Q^+ = 0
\label{eq:ref}
\end{equation}

Також об’ємний заряд Q можна визначити наступним чином:
\begin{equation}
Q^+ = V\cdot \rho^+,
\end{equation}
\begin{equation}
Q^- = V\cdot \rho^-,
\end{equation}
де V --- об’єм, що займають носії заряду;$\rho$ --- об’ємна густина заряду.\\

Підставимо (3) і (4) у (2) та отримаємо:
\begin{equation}
V\cdot \rho^- + V\cdot \rho^+ = 0 \Rightarrow \rho^- +  \rho^+ =0\\
\end{equation}

Об’ємну густину заряду $\rho$ можна визначити наступним чином:
\begin{equation}
\rho^- = -q\cdot N^-
\end{equation}
\begin{equation}
\rho^+ = q\cdot N^+
\end{equation}

Підставляючи (6) і (7) у (5) вираз отримаємо наступне:
\begin{equation}
q\cdot N^- = q\cdot N^+ 
\end{equation}

Оскільки в умові дано напівпровідник n-типу, тоді концентрація від’ємних носіїв заряду визначається основними носіями заряду – електронами, а концентрація додатніх носіїв заряду визначається неосновними носіями заряду – дірками та позитивно зарядженими іонами домішок, тому:
\begin{equation}
N^- = n
\end{equation}
\begin{equation}
N^+ = p + N^+,
\end{equation}
де $N_D$ - концентрація позитивно заряджених іонів донорних домішок.\\

Підставляючи (9) і (10) у (8) отримаємо:
\begin{equation}
n = p + N_D,
\end{equation}


Для того щоб вивести формулу для знаходження провідності кремнію легованого донорними домішкою необхідно врахувати саму електронейтральність речовини при легуванні, а також закон діючих мас:
\begin{equation}
q\cdot n  = q\cdot p,
\label{eq:ref}
\end{equation}

а закон діючіх мас має такий вигляд:
\begin{equation}
n\cdot p = n_i^2
\label{eq:ref}
\end{equation}

Повний заряд основних носіїв (електронів) дорівнює сумі заряду неосновних носіїв (дірок) і заряду іонів донорів тому можна записати такий вираз:\\
\begin{equation}
 q\cdot n = q\cdot p + q\cdot N_D^+ \Rightarrow n = p + N_D^+
\label{eq:ref}
\end{equation}


Підставивши рівність (14) в (13) маємо:
\begin{equation}
 n_i^2 = p^2 + p\cdot N_D^+\\
\label{eq:ref}
\end{equation}

Маємо квадратне рівняння, отримаємо його корені, обчисливши дискримінант, \textbf{але} знаючи що концентрація не може бути від'ємнимною, тоді від'ємним коренем одразу ж нехтуємо та отримуємо:\par
 
 \begin{equation}
 p^2 + p\cdot N_D^+ - n_i^2 = 0
\label{eq:ref}
\end{equation}
\begin{center}
$D = (N_D^+)^2 + 4\cdot n_i^2$\\
\vspace{0.3cm}
$p = \dfrac {-N_D^+ \sqrt{(N_D^+)^2 + 4\cdot n_i^2}}{2}$
\end{center}

Підставляємо у (3) вираз та отримуємо:\\
\begin{equation}
p = \dfrac {N_D^+ \sqrt{(N_D^+)^2 + 4\cdot n_i^2}}{2}
 \end{equation}
 %-----------------------------------------------------------------------------------------------------------------------------------------------------------------------------------------------------
 
 Отримали формулу для розрахунку електропровідності провідника з донорними домішками:\\
 \begin{equation}
\sigma =  q\cdot \dfrac {-N_D^+ +\sqrt{(N_D^+)^2 + 4\cdot n_i^2}}{2} \cdot \mu_n + q\cdot \dfrac {N_D^+ +\sqrt{(N_D^+)^2 + 4\cdot n_i^2}}{2} \cdot \mu_p,
 \end{equation}
 де $N_D^+ = 6\cdot 10^{14}\text{$\text{см}^{-1}$}$ --- концентрація донорної домішки; 
 $n_i = 1.45\cdot 10^{10}\text{$\text{см}^{-3}$}$ --- концентрація власних носіїв; 
 $\mu_n = 1500 \dfrac{\text{$\text{см}^{-3}$}}{B\cdot c}$ --- рухливість електронів; 
 $\mu_p = 450\dfrac{\text{$\text{см}^{-3}$}}{B\cdot c}$ --- рухливість дірок; 
 $q=1.6\cdot10^{-19} \text{Кл}$ --- заряд електрона. 
 \vspace{1cm}

Також додатково я проаналізувала розмірність отриманого виразу, тому з впевненістю можу констатувати той факт що аналітичний вираз для провідності кремнію вірний:
 \begin{equation}
 \text{Кл} \cdot \left( \text{см$^{-3}$} + \sqrt{(\text{см$^{-3}$})^2 + (\text{см$^{-3}$})^2}\right) \cdot \left(\dfrac{\text{см$^2$}}{B\cdot c}\right)\Rightarrow
 \text{Кл} \cdot \text{см$^{-3}$}\cdot \dfrac{\text{см$^2$}}{B\cdot c} = \dfrac{ \text{См}} {\text{см}}
 \end{equation}

Підставляючи вихідні дані у вираз (7), отримаємо:\\
\begin{center}
$\sigma = 1.6\cdot10^{-19}\cdot \dfrac {-4\cdot 10^{14} +\sqrt{(4\cdot 10^{14})^2 + 4\cdot n_i^2}}{2} \cdot 1500 + $ \\
\vspace{0.3cm}
$+ 1.6\cdot10^{-19}\cdot \dfrac {4\cdot 10^{14} +\sqrt{(4\cdot 10^{14})^2 + 4\cdot n_i^2}}{2} \cdot 450 =0.096$
\end{center}

\textbf{Відповідь:} 
$\sigma =  0.096\dfrac{ \text{См}} {\text{см}}$ 




















\end{document}