\documentclass[a4paper,12pt]{article}
\usepackage[left=1.5cm,right=1.5cm,
    top=1cm,bottom=1.5cm,bindingoffset=0cm]{geometry}
    
\usepackage[warn]{mathtext}
\usepackage[T1,T2A]{fontenc}
\usepackage[utf8]{inputenc}
\usepackage[english,russian,ukrainian]{babel}
\usepackage{tabularx}
\usepackage{amssymb}
\usepackage{color}
\usepackage{amsmath}
\usepackage{mathrsfs}
\usepackage{listings}
\usepackage{graphicx}
\graphicspath{ {./images/} }
%\usepackage{draftwatermark} не будет лезть на картинки
\usepackage[printwatermark]{xwatermark}%будет лезть на картинки
\usepackage{lipsum}
\usepackage{xcolor}
\usepackage{tikz}


\definecolor{lgreen}{rgb}{0.5,1,1}
\definecolor{n}{rgb}{1,0.5,0.5}
\definecolor{n1}{rgb}{1,1,0.5}
\definecolor{n3}{rgb}{1,0.7,0.9}




\begin{document}
\pagecolor{white}
Фіцай Б.П. ДП-81\\

Варіант №9\\

Виведення формули для знаходження електропровідності провідника з акцепторними домішками:\\

\large
Закон діючих мас:

\begin{equation}
n\cdot p = n_i^2
\label{eq:ref}
\end{equation}

Виразимо чому дорівнюе p:
\begin{equation}
p = \dfrac{n_i^2}{n}
\label{eq:ref}
\end{equation}

З іншого боку:
\begin{equation}
p = n+N_A
\label{eq:ref}
\end{equation}


Підставимо (3) в (2)\\

Отримаємо:
\begin{equation}
\dfrac{n_i^2}{n} = n+N_A
\label{eq:ref}
\end{equation}


Виконавши деякі нескладні математичні перетворення матимемо:
\begin{equation}
n^2 + nN_A = n_i^2
\label{eq:ref}
\end{equation}
\begin{center}
$ \Downarrow $
\end{center}
\begin{equation}
n^2 + nN_A -n_i^2=0
\label{eq:ref}
\end{equation}

Розв'яжемо квадратне рівняння та знайдемо його корені: так як концентрація не може бути від'ємною, тому
від'ємний корінь рівняння одразу ж відкидаємо:\\
\begin{center}
$D=N_A^2 + 4n_i^2$\\
\vspace{0.3cm}
$n= \dfrac{-N_A+ \sqrt{N_A^2 + 4n_i^2}}{2}$\\
\end{center}

Підставивши у вираз (3), отримаємо:
\begin{equation}
p= \dfrac{-N_A+ \sqrt{N_A^2 + 4n_i^2}}{2} +N_A=\dfrac{N_A+ \sqrt{N_A^2 + 4n_i^2}}{2}
\label{eq:ref}
\end{equation}

Отже,  ми отримали формулу для роздахунку електропровідності провідника з акцепторними домішками:
\begin{equation}
\sigma=q \dfrac{-N_A+ \sqrt{N_A^2 + 4n_i^2}}{2}\cdot \mu_n=q \dfrac{N_A+ \sqrt{N_A^2 + 4n_i^2}}{2}\cdot \mu_p,
\label{eq:ref}
\end{equation}

 де $N_A = 9\cdot 10^{14}\text{$\text{см}^{-1}$}$ --- концентрація акцепторної домішки; 
 $n_i = 1.45\cdot 10^{10}\text{$\text{см}^{-3}$}$ --- концентрація власних носіїв; 
 $\mu_n = 1500 \frac{\text{$\text{ см}^{-3}$}}{B\cdot c}$ --- рухливість електронів; 
 $\mu_p = 450\frac{\text{$\text{ см}^{-3}$}}{B\cdot c}$ --- рухливість дірок; 
 $q=1.6\cdot10^{-19} \text{ Кл}$ --- заряд електрона. 
 \vspace{1cm}

Підставивши всі дані у формулу отримаємо:\\
\begin{center}
$\sigma = 1.6\cdot10^{-19}\cdot \dfrac {-9\cdot 10^{14} +\sqrt{(9\cdot 10^{14})^2 + 4\cdot n_i^2}}{2} \cdot 1500 + $ \\
\vspace{0.3cm}
$+ 1.6\cdot10^{-19}\cdot \dfrac {9\cdot 10^{14} +\sqrt{(9\cdot 10^{14})^2 + 4\cdot n_i^2}}{2} \cdot 450 = 0.0648$
\end{center}

\textbf{Відповідь:} 
$\sigma = 0.0648 \dfrac{ \text{См}} {\text{см}}. $




















\end{document}