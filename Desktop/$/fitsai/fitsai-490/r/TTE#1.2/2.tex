\documentclass[a4paper,12pt]{article}
\usepackage[left=1.5cm,right=1.5cm,
    top=1cm,bottom=1.5cm,bindingoffset=0cm]{geometry}
    
\usepackage[warn]{mathtext}
\usepackage[T1,T2A]{fontenc}
\usepackage[utf8]{inputenc}
\usepackage[english,russian,ukrainian]{babel}
\usepackage{tabularx}
\usepackage{amssymb}
\usepackage{color}
\usepackage{amsmath}
\usepackage{mathrsfs}
\usepackage{listings}
\usepackage{graphicx}
\graphicspath{ {./images/} }
%\usepackage{draftwatermark} не будет лезть на картинки
\usepackage[printwatermark]{xwatermark}%будет лезть на картинки
\usepackage{lipsum}
\usepackage{xcolor}
\usepackage{tikz}


\definecolor{lgreen}{rgb}{0.5,1,1}
\definecolor{n}{rgb}{1,0.5,0.5}
\definecolor{n1}{rgb}{1,1,0.5}
\definecolor{n3}{rgb}{1,0.7,0.9}




\begin{document}
\pagecolor{white}
Фіцай Р.П. ДП-81\\

Варіант №10\\

Виведення формули для знаходження електропровідності провідника з донорними домішками:\\

\large
Запишемо закон діючих мас:

\begin{equation}
n\cdot p = n_i^2
\label{eq:ref}
\end{equation}

Повний заряд електронів дорівнює сумі заряду дірок та заряду іонів донорів:
\begin{equation}
 q\cdot n = q\cdot p + q\cdot N_D^+ 
\label{eq:ref}
\end{equation}
\begin{center}
$ \Downarrow$\\
\end{center}
\begin{equation}
n = p + N_D^+
\label{eq:ref}
\end{equation}


Підставивши вираз (3) у вираз (1) отримаємо:
\begin{align}
 n_i^2 = p^2 + p\cdot N_D^+\\
 p^2 + p\cdot N_D^+ -n_i^2 = 0
\end{align}


Розв'яжемо квадратне рівняння та знайдемо його корені.\\

Від'ємним корінем знехтуємо:
\begin{equation}
p = \dfrac {-N_D^+ \sqrt{(N_D^+)^2 + 4\cdot n_i^2}}{2}\text{   } D = (N_D^+)^2 + 4\cdot n_i^2\\
\label{eq:ref}
\end{equation}

Підставляємо у (3) вираз та отримуємо:\\
\begin{equation}
p = \dfrac {N_D^+ \sqrt{(N_D^+)^2 + 4\cdot n_i^2}}{2}
 \end{equation}
 %-----------------------------------------------------------------------------------------------------------------------------------------------------------------------------------------------------
 
 Отримали формулу для розрахунку електропровідності провідника з донорними домішками:\\
 \begin{equation}
\sigma =  q\cdot \dfrac {-N_D^+ +\sqrt{(N_D^+)^2 + 4\cdot n_i^2}}{2} \cdot \mu_n + q\cdot \dfrac {N_D^+ +\sqrt{(N_D^+)^2 + 4\cdot n_i^2}}{2} \cdot \mu_p,
 \end{equation}
 де $N_D^+ = 10\cdot 10^{14}\text{$\text{см}^{-1}$}$ --- концентрація донорної домішки; 
 $n_i = 1.45\cdot 10^{10}\text{$\text{см}^{-3}$}$ --- концентрація власних носіїв; 
 $\mu_n = 1500 \frac{\text{$\text{ см}^{-3}$}}{B\cdot c}$ --- рухливість електронів; 
 $\mu_p = 450\frac{\text{$\text{ см}^{-3}$}}{B\cdot c}$ --- рухливість дірок; 
 $q=1.6\cdot10^{-19} \text{ Кл}$ --- заряд електрона. 
 \vspace{1cm}

Підставляємо дані, у формулу провідності та отримаємо:\\
\begin{center}
$\sigma = 1.6\cdot10^{-19}\cdot \dfrac {-10\cdot 10^{14} +\sqrt{(10\cdot 10^{14})^2 + 4\cdot n_i^2}}{2} \cdot 1500 + $ \\
\vspace{0.3cm}
$+ 1.6\cdot10^{-19}\cdot \dfrac {10\cdot 10^{14} +\sqrt{(10\cdot 10^{14})^2 + 4\cdot n_i^2}}{2} \cdot 450 = 0.072$
\end{center}

\textbf{Відповідь:} 
$\sigma = 0.072 \dfrac{ \text{См}} {\text{см}}. $



















\end{document}