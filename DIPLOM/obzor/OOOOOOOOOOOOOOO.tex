\title{\textbf{Negative thermal expansion nature and application}}
\author{Mnatsakanov Anton}
\newcommand{\abstractText}{\noindent Abstract goes here.}

\documentclass[12pt, a4paper, twocolumn]{article}
\usepackage[left=1.6cm,right=1.8cm,top=2.0cm,bottom=2.8cm,bindingoffset=0cm]{geometry}
\usepackage[russian,ukrainian,english]{babel}
\input{packages.tex}
\usepackage{xurl}
\usepackage[super,comma,sort&compress]{natbib}
\usepackage{abstract}
\renewcommand{\abstractnamefont}{\normalfont\bfseries}
\renewcommand{\abstracttextfont}{\normalfont\small\itshape}
\usepackage{lipsum}

%%%%%%%%%%%%%%
% References %
%%%%%%%%%%%%%%

 
\usepackage{hyperref}
\hypersetup{colorlinks=true, urlcolor=blue, linkcolor=blue, citecolor=blue}

\begin{document}

%%%%%%%%%%%%%%%%%%%%%%%%%%%%%%%%%%%%%%%%%%%%
\twocolumn[
  \begin{@twocolumnfalse}
    \maketitle
    \begin{abstract}
      Negative thermal expansion coefficient (which at lower temperatures is typical for any polar crystal)\\ 


      \newline
      \newline
    \end{abstract}
  \end{@twocolumnfalse}
]
%%%%%%%%%%%%%%%%%%%%%%%%%%%%%%%%%%%%%%%%%%%%
%%%%%%%%%%%%%%%%%%%%%%%%%%%%%%%%%%%%%%%%%%%% 
\section{Introduction }


\section{Invar and Covar effects}
Эффект исчезновения теплового расширения материала возникает в связи с тем, что магнитострикция точно компенсирует тепловое расширение.\\ 

Тепловое расширение является одним из фундаментальных свойств всех твердых тел.
Но уже нередко, некоторые материалы проявлябт свойства сжиматься  при нагревании и под постоянным давлением. Это материалы обладающие отрицательным  тепловым расширением (NTE).( На самом деле NTE не может быть объяснена нормальная схема описана выше и является предметом исследования в сам.) NTE материалы  имеют большое практическое значение, поскольку позволяет
регулировать  тепловое расширения материала до
какого-то конкретного значение, обычно этого достигают  путем формирования композитов.

...


%%%%%%%%%%%%%%%%%%%%%%%%%%%%%%%%%%%%%%%%%%%%
%%%%%%%%%%%%%%%%%%%%%%%%%%%%%%%%%%%%%%%%%%%% References 
\nocite{*}
\bibliographystyle{plain}
\bibliography{test}

\end{document}


      

  