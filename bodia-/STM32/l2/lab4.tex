\documentclass[a4paper,14pt]{extreport}
  \usepackage[left=1.5cm,right=1.5cm,
      top=1.5cm,bottom=2cm,bindingoffset=0cm]{geometry}
  \usepackage{scrextend}
  \usepackage[T1,T2A]{fontenc}
  \usepackage[utf8]{inputenc}
  \usepackage[english,russian,ukrainian]{babel}
  \usepackage{tabularx}
  \usepackage{amssymb}
  \usepackage{color}
  \usepackage{amsmath}
  \usepackage{mathrsfs}
  \usepackage{listings}
  \usepackage{graphicx}
  \usepackage{xcolor}
  \usepackage{hyperref}
 



  
  
%}

\newcommand{\img}[4]{\center{\includegraphics[width=#1\linewidth]{#2}}\captionof{figure}{#3}\label{#4}}
\begin{document}
  \pagecolor{white}

  %----------------------------------------1
\begin{titlepage}
    \begin{center}
    \large
    Національний технічний університет України \\ "Київський політехнічний інститут імені Ігоря Сікорського"


    Факультет Електроніки

    Кафедра мікроелектроніки
    \vfill

    \textsc{ЗВІТ}\\

    {\Large Про виконання лабораторної роботи №4\\
    з дисципліни: «Мікропроцесори та мікроконтролери»\\[1cm]

    %Резистивні сенсори температури


    }
    \bigskip
    \end{center}
    \vfill

    \newlength{\ML}
    \settowidth{\ML}{«\underline{\hspace{0.4cm}}» \underline{\hspace{2cm}}}
    \hfill
    \begin{minipage}{1\textwidth}
    Виконавець:\\
    Студент 4-го курсу \hspace{4cm} $\underset{\text{(підпис)}}{\underline{\hspace{0.2\textwidth}}}$  \hspace{1cm}А.\,С.~Мнацаканов\\
    \vspace{1cm}

    Перевірив: \hspace{6.1cm} $\underset{\text{(підпис)}}{\underline{\hspace{0.2\textwidth}}}$  \hspace{1cm} Татарчук Д. Д.\\

    \end{minipage}

    \vfill

    \begin{center}
    2021
    \end{center}
\end{titlepage}



\newpage
\setcounter{page}{2}


\begin{verbatim}
 int but_prev = 0;
 
 int but_cur = 0;
 
 int a = 0;
 
  while (1)
 
 {
 
 but_cur = HAL_GPIO_ReadPin (GPIOA , GPIO_PIN_0 ) ;
 
 if ( ( b u t _ p r e v == 0) && ( but_cur != 0) )
 
 {
 
 a=a + 1 ;
 
 switch (a)
  {
 
 case 1:
 
 HAL_GPIO_TogglePin ( GPIOC , GPIO_PIN_8 ) ; 
 HAL_Delay ( 2 0 0 0 ) ;
 
 break ;
 
 case 2:
 
 HAL_GPIO_TogglePin ( GPIOC , GPIO_PIN_8 ) ;
 
 HAL_Delay ( 3 0 0 0 ) ;
 
 break ;
 
 case 3:
  a =0;
  break ;
 
 }
 
 }

 else
{
switch{(a)
case 0:
HAL_GPIO_TogglePin ( GPIOC ,HAL_Delay ( 1 0 0 0 ) ;
break ;
GPIO_PIN_8 ) ;
case  1:
HAL_GPIO_TogglePin ( GPIOC ,HAL_Delay ( 2 0 0 0 ) ;
break ;
GPIO_PIN_8 ) ;
}
case
 2:
HAL_GPIO_TogglePin ( GPIOC ,HAL_Delay ( 3 0 0 0 ) ;
break ;
GPIO_PIN_8 ) ;
}
but_prev = but_cur ;}
\end{verbatim}




\begin{center}
\textbf{Висновок}
\end{center}

Написана програма, в якiй за натисканням кнопки змiнюється перiод
включення-виключення свiтлодiоду: 
 При запуску перiод має складати 1 секунду.
 
 Пiсля першого натискання 2 секунди.
 
 Пiсля другого натискання 3 секунди.
 
 Пiсля третього повертається перiод 1 секунда i т.д. циклiчно.




\end{document}