\documentclass[a4paper,14pt]{extreport}
  \usepackage[left=1.5cm,right=1.5cm,
  top=1.5cm,bottom=2cm,bindingoffset=0cm]{geometry}
  \usepackage{scrextend}
  \usepackage[T1,T2A]{fontenc}
  \usepackage[utf8]{inputenc}
  \usepackage[english,russian,ukrainian]{babel}
  \usepackage{tabularx}
  \usepackage{amssymb}
  \usepackage{color}
  \usepackage{amsmath}
  \usepackage{mathrsfs}
  \usepackage{listings}
  \usepackage{graphicx}
  \graphicspath{ {./images/} }
  \usepackage{lipsum}
  \usepackage{xcolor}
  \usepackage{hyperref}
  \usepackage{tcolorbox}
  \usepackage{tikz}
  \usepackage[framemethod=TikZ]{mdframed}
  \usepackage{wrapfig,boxedminipage,lipsum}
  \mdfdefinestyle{MyFrame}{%
  linecolor=blue,outerlinewidth=2pt,roundcorner=20pt,innertopmargin=\baselineskip,innerbottommargin=\baselineskip,innerrightmargin=20pt,innerleftmargin=20pt,backgroundcolor=gray!50!white}
  \usepackage{csvsimple}
  \usepackage{supertabular}
  \usepackage{pdflscape}
  \usepackage{fancyvrb}
  %\usepackage{comment}
  \usepackage{array,tabularx}
  \usepackage{colortbl}

  \usepackage{varwidth}
  \tcbuselibrary{skins}
  \usepackage{fancybox}
  \usepackage{spreadtab}
  % Цвета для гиперссылок
  \definecolor{linkcolor}{HTML}{799B03} % цвет ссылок
  \definecolor{urlcolor}{HTML}{799B03} % цвет гиперссылок


  \usepackage{tikz}
  \usepackage[framemethod=TikZ]{mdframed}
  \usepackage{xcolor}
  \usetikzlibrary{calc}
  \makeatletter
  \newlength{\mylength}
  \xdef\CircleFactor{1.1}
  \setlength\mylength{\dimexpr\f@size pt}
  \newsavebox{\mybox}
  \newcommand*\circled[2][draw=blue]{\savebox\mybox{\vbox{\vphantom{WL1/}#1}}\setlength\mylength{\dimexpr\CircleFactor\dimexpr\ht\mybox+\dp\mybox\relax\relax}\tikzset{mystyle/.style={circle,#1,minimum height={\mylength}}}
  \tikz[baseline=(char.base)]
  \node[mystyle] (char) {#2};}
  \makeatother

  \definecolor{ggreen}{rgb}{0.4,1,0}
  \definecolor{rred}{rgb}{1,0.1,0.1}
  \definecolor{amber}{rgb}{1.0, 0.75, 0.0}
  \definecolor{babyblue}{rgb}{0.54, 0.81, 0.94}
  \definecolor{amethyst}{rgb}{0.6, 0.4, 0.8}

  \usepackage{float}
  \usepackage{wrapfig}
  \usepackage{framed}
  %for nice Code{
  \lstdefinestyle{customc}{
  belowcaptionskip=1\baselineskip,
  breaklines=true,
  frame=L,
  xleftmargin=\parindent,
  language=C,
  showstringspaces=false,
  basicstyle=\small\ttfamily,
  keywordstyle=\bfseries\color{green!40!black},
  commentstyle=\itshape\color{purple!40!black},
  identifierstyle=\color{blue},
  stringstyle=\color{orange},
  }
  \lstset{escapechar=@,style=customc}
  %}


\begin{document}
\pagecolor{white}

%----------------------------------------1
\newtcbox{\xmybox}[1][red]{on line, arc=7pt,colback=#1!10!white,colframe=#1!50!black,
 before upper={\rule[-3pt]{0pt}{10pt}},boxrule=1pt, boxsep=0pt,left=6pt,right=6pt,
 top=2pt,bottom=2pt}

\newpage
\setcounter{page}{2}


\begin{center}
\fbox{Класифікація потреб Мюррея}
\end{center}


Генрі Мюррей описав психогенні потреби, які призводять до певних вчинків. Ці потреби містять такі виміри як автономія (незалежність), захист (самозахист від критики), гра (заняття діяльністю, що приносить задоволення). Він запропонував класифікацію потреб споживача за чотирма ознаками:
\begin{itemize}
\item фізіологічне походження — первинні і вторинні потреби;
\item ступінь привабливості об'єкту задоволення потреби — позитивні потреби і негативні потреби;
\item ступінь прояву проблеми — явні і латентні потреби;
\item ступінь усвідомлення потреби — усвідомлені і неусвідомлені потреби;
\end{itemize}
На його думку, всі мають однакові потреби, але ступінь їх вираження для кожного відрізняється в залежності від особистих чинників і чинників зовнішнього середовища. Потреби можуть бути спровоковані як внутрішніми, так і зовнішніми стимулами та можуть проявлятися сильніше або слабше.

Список психогенних потреб Генрі Мюррея:

Потреби, пов'язані з неживими об'єктами:
\begin{itemize}
\item придбання;
\item порядок;
\item зберігання;
\item конструювання, будівництво;
\end{itemize}
Потреби, що відображають амбіції, владу, досягнення й престиж:
\begin{itemize}
\item перевага;
\item досягнення;
\item визнання;
\item прояв;
\item непорушність;
\item непохитність (для запобігання сорому, невдачі, приниження, висміювання);
\item захищеність (почуття захищеності, безпеки);
\item протидії (протидіюче ставлення). Потреби, пов'язані із владою людини:
\item вплив, домінування;
\item повага;
\item мораль.
\end{itemize}
Садо-мазохістські потреби:
\begin{itemize}
\item протидії (протидіюче ставлення).
\item агресія;
\item приниження.
\end{itemize}
Потреби, пов'язані зі стримуванням:
\begin{itemize}
\item ухиляння від відповідальності (бажання уникнути відповідальності). Потреби, що належать до прихильності людей один до одного:
\item причетність;
\item відкинутість;
\item опіка (годувати, допомагати або захищати безпомічного);
\item допомога (шукати допомогу, захист, співчуття);
\item гра.
\end{itemize}
Потреби в спілкування (бажання запитати й бути почутим):
\begin{itemize}
\item усвідомлення (запитальна позиція);
\item тлумачення (пояснювальна позиція).
\end{itemize}

\end{document}
