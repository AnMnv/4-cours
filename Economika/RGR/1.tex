\documentclass[a4paper,14pt]{extreport}
\usepackage[left=1.5cm,right=1.5cm,
    top=1.5cm,bottom=2cm,bindingoffset=0cm]{geometry}
    \usepackage[english,russian,ukrainian]{babel}
\input{preambula.tex}
\usepackage{multirow}
\usepackage{spreadtab}
\usepackage{pgf-pie}   
\usepackage{xcolor} % package used to implement colors  




 
\begin{document}


\begin{titlepage}
      \begin{center}
      \large
      Національний технічний університет України \\ "Київський політехнічний інститут імені Ігоря Сікорського"


      
      Факультет Електроніки

      Кафедра мікроелектроніки
      \vfill

      \textsc{РОЗРАХУНКОВО-ГРАФІЧНА РОБОТА}\\
    за темою: «Аналіз доцільності виходу на ринок моноблоку власного виробництва»
      %{\Large Про виконання лабораторної роботи №1\\
      з дисципліни: «Фізичні основи сенсорики»\\[1cm]

      


      %}
      \bigskip
      \end{center}
      \vfill

      \newlength{\ML}
      \settowidth{\ML}{«\underline{\hspace{0.4cm}}» \underline{\hspace{2cm}}}
      \hfill
      \begin{minipage}{1\textwidth}
      Виконав:\\
      Студент 4-го курсу \hspace{4cm} $\underset{\text{(підпис)}}{\underline{\hspace{0.2\textwidth}}}$  \hspace{1cm}Мнацаканов А. С.\\
      \vspace{1cm}

      Перевірила: \hspace{5.6cm} $\underset{\text{(підпис)}}{\underline{\hspace{0.2\textwidth}}}$  \hspace{1cm}доц. Павленко Т. В.\\

      \end{minipage}

      \vfill

      \begin{center}
      2021
      \end{center}
\end{titlepage}

\tableofcontents
\setcounter{page}{2}
\newpage

%%%%%%%%%%%%%%%%%%%%%%%%%%%%%%%%%%%%%%%%%%%%%%%%%%%%%%%%%%%%%%%%%%%%%%%%%%%%%%%
\chapter{Вступ}
%%%%%%%%%%%%%%%%%%%%%%%%%%%%%%%%%%%%%%%%%%%%%%%%%%%%%%%%%%%%%%%%%%%%%%%%%%%%%%%
    Метою цієї РГР є аналіз доцільності виходу на ринок виробу, згідно з яким
    необхідним є визначення оцінка рівня якості та конкурентоспроможності, а також
    розрахунок собівартості та ціни виробу.\\ 

    Необхідні розрахунки будемо проводити для кастомного моноблоку. 
    Потрібно буде обґрунтувати вибір договірної ціни, а також розрахувати кількість
    виробленої продукції, при якій виробництво буде беззбитковим та досягне
    запланованого рівня рентабельності.


%%%%%%%%%%%%%%%%%%%%%%%%%%%%%%%%%%%%%%%%%%%%%%%%%%%%%%%%%%%%%%%%%%%%%%%%%%%%%%%
\chapter{Аналіз ринку}
%%%%%%%%%%%%%%%%%%%%%%%%%%%%%%%%%%%%%%%%%%%%%%%%%%%%%%%%%%%%%%%%%%%%%%%%%%%%%%%
    Даний   пристрій відноситься до класу «All-in-one PC» -- представляє собою РК-монітор, ззаду якого знаходиться системний блок, він поєднує кілька пристроїв в одному корпусі, застосовується для зменшення займаної обладнанням площі, спрощення збірки кінцевим користувачем, надання естетичного вигляду.\\

    Продаж виробу буде організований на ринках України, Білорусі, Польщі, прибалтійських
    країн, Молдови, Румунії, оскільки в цих державах відсутні фірми, що спеціалізуються на
    виготовленні подібної продукції.  Виріб буде продаватись оптовим покупцям та в роздрібних фірмових магазинах.\\ 

    Основні технічні характеристики виробу наступні:
    \begin{itemize}
    \item Процесор \dotfill 2.2 ГГц 
    \item Вiдеокарта \dotfill 3 Тф
    \item ОЗУ \dotfill 16 Гб
    \item Яскравiсть дисплея \dotfill 500 кд
    \item Дiагональ \dotfill 19.5 inch
    \end{itemize}

    Візьмемо серійне виробництво з серією 5000 шт./рік.\\ 

    Найбільш наближеним за технічними характеристиками до розробленого
    пристрою  є  \href{https://www.itbox.ua/ua/product/Kompyuter_HP_24-df0054ua_IPS_Pentium_J5040_426F3EA-p702646/?utm=shopping&utm_content=shopping&gclid=CjwKCAiAtdGNBhAmEiwAWxGcUoFSjhq16_yyk_vusMuu-Nb1aAfwNXYS7bWDjVH86uK1hBbyGcG11BoCPasQAvD_BwE}{моноблок}  китайської  компанії  з  наступними  технічними
    характеристиками:
    \begin{itemize}
    \item Процесор \dotfill 2.8 ГГц 
    \item Вiдеокарта \dotfill 2.2 Тф
    \item ОЗУ \dotfill 12 Гб
    \item Яскравiсть дисплея \dotfill 400кд
    \item Дiагональ \dotfill 20 inch
    \end{itemize}
    Ціна за базовий виріб 22 000 грн.



%%%%%%%%%%%%%%%%%%%%%%%%%%%%%%%%%%%%%%%%%%%%%%%%%%%%%%%%%%%%%%%%%%%%%%%%%%%%%%%
\chapter{ОЦІНКА РІВНЯ ЯКОСТІ ВИРОБУ}
%%%%%%%%%%%%%%%%%%%%%%%%%%%%%%%%%%%%%%%%%%%%%%%%%%%%%%%%%%%%%%%%%%%%%%%%%%%%%%%

\section{Вихідні положення}
    Для оцінки рівня якості виробу використовується коефіціснт технічного рівня ( $K_{\text {T.P. }}$ ), який розраховується для кожного варіанту інженерного рішення:
    \begin{align}
    K_{T . P .}=\sum_{i=1}^{n} \varphi_{i j} B_{i j}
    \end{align}
    де $\varphi_{i j}$ - коефіціснт вагомості $i$-го параметра якості $j$-го варіанта в сукупності прийнятих для розгляду параметрів якості; $B_{i j}$ - оцінка $i$-го параметра якості $j$-го варіанта виробу; $^{n}-$ кількість параметрів виробу.

    При наявності кількісної характеристики виробу коефіцієнт технічного рівня можна визначити за формулою:
    \begin{align}
    K_{T. P.}=\sum_{i=1}^{n} \varphi_{i} q_{i}
    \end{align}
    де $q_{i}$ - відносний (одиничний)   i-й показник якості. 


\section{Обґрунтування системи параметрів виробу і визначення відносних показників якості}
    На основі даних про зміст основних функцій, які повинен реалізовувати виріб, вимог
    замовника, а також умов, які характеризують експлуатацію виробу, визначають основні
    параметри виробу, які будуть використані для розрахунку коефіцієнта технічного рівня виробу.\\ 

    Відносні показники якості по будь-якому параметру $q_i$, якщо вони находяться в лінійній
    залежності від якості, визначаються за формулами:
    \begin{align}
    q_i = \dfrac{P_{H_{i}}}{P_{\text{Б}_{i}}}
    \end{align}
    або
    \begin{align}
    \acute q_i = \dfrac{P_{H_{i}}}{P_{\text{Б}_{i}}}
    \end{align}
    Формула (3) використовується при розрахунку відносних показників якості,
    коли збільшення величини параметра веде до покращення якості виробу і формула
    (4) коли зі збільшенням величини параметра якість виробу погіршується.
    Параметри нового і базового виробів наведені в табл. \ref{t1}. В цій таблиці також
    наведемо розраховані показники якості.

    \begin{table}[h!]
    \caption{Параметри виробів.}
    \begin{center}
        \begin{spreadtab}{{tabular}{|l|c|c|c|}}
        \hline%     A                             B           С            D
        @Параметр                           &@ Варіант   & @Варіант   & @Показник якості $q_i$ \\ \hline
                                            & @базовий   & @новий     &      \\ \hline
        1) @Вага \hfill [кг]               &   4        &   3.5      &  round(b3/c3,2)    \\ \hline
        2) @Діагональ\hfill [inch]         &    20      &  19.5      &  round(c4/b4,2)    \\ \hline
        3) @Процесор \hfill [ГГц]          &    2.8     &   2.2      &  round(c5/b5,2)   \\ \hline
        4) @Відеокарта\hfill [Тф]          &    2.2     &   3        &  round(c6/b6,2)   \\ \hline
        5) @ОЗУ\hfill [Гб]                 &     12     &   16       &  round(c7/b7,2)   \\ \hline
        6) @Яскравість дисплея \hfill [кд]  & 400        &   500      &  round(c8/b8,2)     \\ \hline
        \end{spreadtab}
    \end{center} 
    \label{t1}
    \end{table}


\section{Визначення коефіцієнтів вагомості параметрів}
    Вагомість кожного параметра в загальній кількості розглянутих при оцінці
    параметрів визначається методом попарного порівняння. Оцінку проводить
    експертна комісія, кількість членів якої повинна дорівнювати непарному числу (не
    менше 7 осіб). Експерти повинні бути фахівцями у даній предметній галузі. Після
    детального обговорення та аналізу кожний експерт оцінює ступінь важливості
    шляхом присвоєння їм рангів. В даному випадку оцінку дають 7 експертів в галузі
    вимірювальних приладів. Результати рангування параметрів заносимо до табл. \ref{t2}.

     \begin{table}[ht]
     
    \caption{Результат оцінки параметрів}
     \begin{center}
        \begin{spreadtab}{{tabular}{|l|c|c|c|c|c|c|c|c|c|c|}}
        \hline%  A     B     С    D     e   f    g     h               i                j                              k
                      & 1  & 2  & 3  & 4 &  5 &   6 &  7 & @$\sum \text{рівнів}$& @$\triangle$           &  @$\triangle^2$\\ \hline
        @x1           & 5  & 4  & 4  & 5 &  6 & 5  &  6 & sum(b2:h2)               &   round( i2-(sum(i2:i7)/6),2) & j2*j2\\ \hline
        @x2           & 6  & 6  & 5  & 3 &  4 &  4 &  4 &sum(b3:h3)                &    round(i3-(sum(i2:i7)/6),2)  & j3*j3\\ \hline
        @x3           & 4  & 5  & 6  & 6 &  5 &  6 &  5 &sum(b4:h4)                &    round(i4-(sum(i2:i7)/6),2)  & j4*j4\\ \hline
        @x4           & 1  & 2  & 1  & 1 &  1 &  3 &  1 &sum(b5:h5)                &    round(i5-(sum(i2:i7)/6),2)  & j5*j5\\ \hline
        @x5           & 3  & 3  & 2  & 2 &  2 &  2 &  3 &sum(b6:h6)                &    round(i6-(sum(i2:i7)/6),2)  & j6*j6\\ \hline
        @x6           & 2  & 1  & 3  & 4 &  3 &  1 &  2 &sum(b7:h7)                &    round(i7-(sum(i2:i7)/6),2)  & j7*j7\\ \hline
        @$\sum $      &    &    &    &   &    &    &    &  sum(i2:i7)              &                                & sum(k2:k7) \\ \hline
        \end{spreadtab}
    \end{center} 
    \label{t2}
     \end{table}

     Коефіцієнт конкордації 
     \[ W = \dfrac{S\cdot 12}{N^2 (n^3-n)} = \dfrac{661.5\cdot 12}{7^2 (6^3-6)} \approx 0.77, \]
     де $n$ -- к-ть параметрів, $N$ -- к-ть експериментів.\\ 
     Нормативна величина для радіотехнічних виробів $W_H$ =  0,77 . Розрахункове
    значення W  = $W_H$ , отже визначені дані заслуговують довіри.
    Далі проводимо попарне порівняння всіх параметрів, результати занесемо до
    табл. \ref{t3}.

    \begin{table}[h!]
    \caption{Попарне зрівняння параметрів}
    \begin{center}
    \begin{tabular}{|c|ccccccc|c|c|}
    \hline
    \multirow{2}{*}{} & \multicolumn{7}{c|}{Експерт}                                                                                                                            & \multirow{2}{*}{Підсумкова оцінка} & \multirow{2}{*}{Числове значення} \\ \cline{2-8}
                      & \multicolumn{1}{c|}{1} & \multicolumn{1}{c|}{2} & \multicolumn{1}{c|}{3} & \multicolumn{1}{c|}{4} & \multicolumn{1}{c|}{5} & \multicolumn{1}{c|}{6} & 7 &                                    &                                   \\ \hline
    x1 vs  x2         & \multicolumn{1}{c|}{>} & \multicolumn{1}{c|}{>} & \multicolumn{1}{c|}{>} & \multicolumn{1}{c|}{<} & \multicolumn{1}{c|}{<} & \multicolumn{1}{c|}{<} & < & <                                  & 0.5                               \\ \hline
    x1 vs x3          & \multicolumn{1}{c|}{<} & \multicolumn{1}{c|}{>} & \multicolumn{1}{c|}{>} & \multicolumn{1}{c|}{>} & \multicolumn{1}{c|}{<} & \multicolumn{1}{c|}{>} & < & >                                  & 1.5                               \\ \hline
    x1 vs x4          & \multicolumn{1}{c|}{<} & \multicolumn{1}{c|}{<} & \multicolumn{1}{c|}{<} & \multicolumn{1}{c|}{<} & \multicolumn{1}{c|}{<} & \multicolumn{1}{c|}{<} & < & <                                  & 0.5                               \\ \hline
    x1 vs x5          & \multicolumn{1}{c|}{<} & \multicolumn{1}{c|}{<} & \multicolumn{1}{c|}{<} & \multicolumn{1}{c|}{<} & \multicolumn{1}{c|}{<} & \multicolumn{1}{c|}{<} & < & <                                  & 0.5                               \\ \hline
    x1 vs x6          & \multicolumn{1}{c|}{<} & \multicolumn{1}{c|}{<} & \multicolumn{1}{c|}{<} & \multicolumn{1}{c|}{<} & \multicolumn{1}{c|}{<} & \multicolumn{1}{c|}{<} & < & <                                  & 0,5                               \\ \hline
    x2 vs x3          & \multicolumn{1}{c|}{<} & \multicolumn{1}{c|}{<} & \multicolumn{1}{c|}{>} & \multicolumn{1}{c|}{>} & \multicolumn{1}{c|}{>} & \multicolumn{1}{c|}{>} & > & >                                  & 1.5                               \\ \hline
    x2 vs x4          & \multicolumn{1}{c|}{<} & \multicolumn{1}{c|}{<} & \multicolumn{1}{c|}{<} & \multicolumn{1}{c|}{<} & \multicolumn{1}{c|}{<} & \multicolumn{1}{c|}{<} & < & <                                  & 0.5                               \\ \hline
    x2 vs x5          & \multicolumn{1}{c|}{<} & \multicolumn{1}{c|}{<} & \multicolumn{1}{c|}{<} & \multicolumn{1}{c|}{<} & \multicolumn{1}{c|}{<} & \multicolumn{1}{c|}{<} & < & <                                  & 0.5                               \\ \hline
    x2 vs x6          & \multicolumn{1}{c|}{<} & \multicolumn{1}{c|}{<} & \multicolumn{1}{c|}{<} & \multicolumn{1}{c|}{>} & \multicolumn{1}{c|}{<} & \multicolumn{1}{c|}{<} & < & <                                  & 0.5                               \\ \hline
    x3 vs x4          & \multicolumn{1}{c|}{<} & \multicolumn{1}{c|}{<} & \multicolumn{1}{c|}{<} & \multicolumn{1}{c|}{<} & \multicolumn{1}{c|}{<} & \multicolumn{1}{c|}{<} & < & <                                  & 0.5                               \\ \hline
    x3 vs x5          & \multicolumn{1}{c|}{>} & \multicolumn{1}{c|}{>} & \multicolumn{1}{c|}{>} & \multicolumn{1}{c|}{>} & \multicolumn{1}{c|}{>} & \multicolumn{1}{c|}{<} & < & >                                  & 1.5                               \\ \hline
    x3 vs x6          & \multicolumn{1}{c|}{<} & \multicolumn{1}{c|}{<} & \multicolumn{1}{c|}{<} & \multicolumn{1}{c|}{<} & \multicolumn{1}{c|}{<} & \multicolumn{1}{c|}{<} & < & <                                  & 0.5                               \\ \hline
    x4 vs x5          & \multicolumn{1}{c|}{>} & \multicolumn{1}{c|}{>} & \multicolumn{1}{c|}{>} & \multicolumn{1}{c|}{>} & \multicolumn{1}{c|}{>} & \multicolumn{1}{c|}{>} & < & >                                  & 1.5                               \\ \hline
    x4 vs x6          & \multicolumn{1}{c|}{>} & \multicolumn{1}{c|}{<} & \multicolumn{1}{c|}{>} & \multicolumn{1}{c|}{>} & \multicolumn{1}{c|}{>} & \multicolumn{1}{c|}{<} & > & >                                  & 1.5                               \\ \hline
    x5 vs x6          & \multicolumn{1}{c|}{<} & \multicolumn{1}{c|}{<} & \multicolumn{1}{c|}{>} & \multicolumn{1}{c|}{>} & \multicolumn{1}{c|}{>} & \multicolumn{1}{c|}{<} & < & <                                  & 0.5                               \\ \hline
    \end{tabular}
    \end{center} 
    \label{t3}
    \end{table}

    \begin{table}[h!]
    \caption{Розрахунок вагомості параметрів.}
    \begin{center}
        \begin{spreadtab}{{tabular}{|l|c|c|c|c|c|c|c|c|}}
    \hline%A       B       С      D     e      f       g               h           i
                 & @x1  & @x2  & @x3  &@x4  &@x5   &   @x6    & @$b_i$      &  @   $\varphi_i$ \\ \hline
        @x1      & 1    & 1.5  & 1.5  & 0.5 &  0.5 &  0.5     &  sum(b2:g2) & round(h2/(sum(h2:h7)),2)\\ \hline
        @x2      & 0.5  & 1    & 1.5  & 0.5 &  0.5 &  0.5     &  sum(b3:g3) & round(h3/(sum(h2:h7)),2)\\ \hline
        @x3      & 0.5  & 0.5  & 1    & 0.5 &  0.5 &  0.5     &  sum(b4:g4) & round(h4/(sum(h2:h7)),2)\\ \hline
        @x4      & 1.5  & 1.5  & 1.5  & 1   &  1.5 &  1.5     &  sum(b5:g5) & round(h5/(sum(h2:h7)),2)\\ \hline
        @x5      & 1.5  & 1.5  & 1.5  & 0.5 &  1   &  1.5     &  sum(b6:g6) & round(h6/(sum(h2:h7)),2)\\ \hline
        @x6      & 1.5  & 1.5  & 1.5  & 0.5 &  0.5 &   1      &  sum(b7:g7)  & round(h7/(sum(h2:h7)),2)\\ \hline
        @$\sum$  &      &      &      &     &      &          &  sum(h2:h7)  & round(sum(i2:i7),0)\\ \hline
        \end{spreadtab}
    \end{center} 

    \label{t4}
    \end{table}



    Розрахунок вагомості (пріоритетності) кожного параметра $\varphi_{i}$ проводимо за наступними формулами:
    $$
    \varphi_{i}=\frac{b_{i}}{\sum_{i=1}^{n} b_{i}}=\sum_{i=1}^{n} a_{i j}
    $$
    де $b_{i}$-вагомість $i$-го параметра за результатами оцінок всіх експертів визначається як сума значень коефіцієнтів переваги $\left(a_{i j}\right)$ даних усіма експертами по $i$-му параметру.
    Результати розрахунків занесемо до табл. \ref{t4}


%%%%%%%%%%%%%%%%%%%%%%%%%%%%%%%%%%%%%%%%%%%%%%%%%%%%%%%%%%%%%%%%%%%%%%%%%%%%%%%
\chapter{РОЗРАХУНОК СОБІВАРТОСТІ ВИРОБУ}
%%%%%%%%%%%%%%%%%%%%%%%%%%%%%%%%%%%%%%%%%%%%%%%%%%%%%%%%%%%%%%%%%%%%%%%%%%%%%%%
\section{Калькуляція собівартості}
\subsection{Сировина та матеріали}
    \begin{table}[h!]
        \caption{Витрати на матеріали.}
        \begin{center}
            \begin{spreadtab}{{tabular}{|p{4.2cm}|p{2.2cm}|p{2.2cm}|p{2.0cm}|p{2.0cm}|c|}}
    \hline% A                B                       С                   D              E                F 
    @Матеріал       &@ Стандарт або марка   & @Одиниця виміру   & @Норма витрат & @Ціна одиниці, грн &@ Сума, грн \\ \hline
    @ Екран         &@ Sasung IPS           & @ inch  &     & 4000              & e2 \\ \hline
    @  Процесор     &@ core i7 5th gen      & @ ГГц  &      & 2000              & e3 \\ \hline
    @  Відеокарта   &@ GeForce GT 1030      & @ Тф  &       & 4500              & e4 \\ \hline
    @  ОЗУ          &@ DDR4                 & @ МГц  &      & 960               & e5 \\ \hline
    @ Разом         &@                      & @   &   &                         & round(sum(e2:e5),2) \\ \hline
    @ Невраховані матеріали 10\%            & @   & @   &   &                   & round(f6/11,2) \\ \hline
    @  Всього з урахуванням транспортно-заготівельних витрат (К = 1,1)& @ & & && round((f7+f6)*1.1,0) \\ \hline
    \end{spreadtab}
        \end{center} 
        \label{t5}
        \end{table}


\subsection{ Покупні комплектуючі виробу, напівфабрикати, роботи і послуги
    виробничого характеру сторонніх підприємств та організацій}
    \begin{table}[h!]
    \caption{Розрахунки по витратам на покупні вироби та напівфабрикати.}
    \begin{center}
    \begin{spreadtab}{{tabular}{|p{4.2cm}|p{2.2cm}|p{2.2cm}|p{2.0cm}|p{2.0cm}|}}
    \hline% A                B                       С                   D              E                 
    @Вироби, напівфабрикати&@ Стандарт або марка   & @К-сть, одиниць    & @Ціна одиниці, грн &@ Сума, грн \\ \hline
    @ дроти        &@ -    & @ 10      & 52              & d2 \\ \hline
    @  клеми       &@ -    & @ 8      & 19             & d3 \\ \hline
    @  корпус      &@ -    & @ 1      & 300              & d4 \\ \hline
    @  охолоджння  &@ -    & @ 1      &120               & d5 \\ \hline
    @ Разом         &@      & @        &         & round(sum(d2:d5),2) \\ \hline
    @ Невраховані матеріали 10\% & @   & @       &                 &round(e6*1.1,0)- round(sum(d2:d5),2) \\ \hline
    @  Всього з урахуванням транспортно-заготівельних витрат (К = 1,1)& @ & & & e6+e7 \\ \hline
    \end{spreadtab}
    \end{center} 
    \label{t6}
    \end{table}




\subsection{ Основна заробітна плата}
    \begin{table}[h!]
    \caption{Основна заробітна плата.}
    \begin{center}
    \begin{spreadtab}{{tabular}{|p{4.2cm}|p{2.2cm}|p{2.2cm}|p{2.0cm}|p{2.0cm}|}}
    \hline% A                B                              С                              D           E                 
    @Найменування робіт &@ Сер. погод. ставка   & @К-сть операцій, одиниць  & @Норма часу, год &@ Сума, грн \\ \hline
    @ підготовка елементів     &@ 20             & @ 1                      & 52              & 0.25 \\ \hline
    @ нанесення термопасти     &@ 20             & @ 1                      & 52              & 1.82 \\ \hline
    @ встановлення охолодження &@ 20             & @ 1                      & 52              & 2.7 \\ \hline
    @ попередня діагностика    &@ 22             & @ 1                      & 52              & 1.3 \\ \hline
    @ калібровка               &@ 25             & @ 1                      & 52              & 1.5 \\ \hline
    @  Всього                  & @ & & & sum(e2:e6) \\ \hline
    \end{spreadtab}
    \end{center} 
    \label{t7}
    \end{table}



\subsection{ Додаткова заробітна плата}
    Витрати за цією статтею визначаються у відсотках до основної заробітної
    плати:\\ 

    \FPset\k{0.35}
    \FPset\c{7.57}
    \FPmul\cc\k\c
    \FPeval\cc{round(\cc:2)}
    \begin{equation}
    C_{\text{з.д}}. = k_{\text{з.д}} C_{\text{з.o}} = \k \cdot \c = \cc \text{ грн},
    \end{equation}
    де к$_{\text{з.д}}$ = 0,3...0,4 – коефіцієнт, який враховує додаткову зарплату.


\subsection{ Відрахування на соціальне страхування}
    За діючими нормативами, відрахування на соціальне страхування (ЄСВ)
    складає 22\% від суми основної та додаткової заробітної плати:\\ 

    \FPset\dd{0.22}
     \FPmul\esv\c\dd
     \FPeval\esv{round(\esv:2)}
    \begin{equation}
     \text{ЄСВ} =  \c \cdot \dd = \esv \text{ грн}
    \end{equation}
 
\subsection{Загальновиробничі витрати}
    Враховуючи, що собівартість виробу визначається на ранніх стадіях його
    проектування в умовах обмеженої інформації щодо технології виробництва та
    витрат на його підготовку у загальновиробничі витрати включаються, крім власне
    цих витрат, витрати на: освоєння основного виробництва, відшкодування зносу
    спеціальних інструментів і пристроїв цільового призначення, утримання та
    експлуатацію устаткування. При цьому загальновиробничі витрати визначаються у
    відсотках до основної заробітної плати. При такому комплексному складі
    загальновиробничих витрат їх норматив ( n$_{\text{з.в}}$ ) досягає 200-300\%, тобто
    \FPadd\czv\c\c
    \FPeval\czv{round(\czv:2)}
     \begin{equation}
        C_{\text{з.д}}. = n_{\text{з.в}}\cdot C_{\text{з.o}} =  \czv  \text{ грн}
    \end{equation}

\subsection{Адміністративні витрати}
    Ці витрати відносяться на собівартість виробу пропорційно основній
    заробітній платі і на приладобудівних підприємствах вони становлять ( n$_{\text{з.г}}$ ) 100-
    200\%, нехай
    \FPset\nzg{1.3}
    \FPadd\czg\c\nzg
    \FPeval\czg{round(\czg:2)}
    \begin{equation}
    C_{\text{з.г}}. = n_{\text{з.г}}\cdot C_{\text{з.o}} =  \czg  \text{ грн}
    \end{equation}



\subsection{Комерційні витрати}
    Витрати за цією статтею визначаються у відсотках до виробничої
    собівартості ( $n_{\text{з.в}}$ = 2,5 ... 5\% ), С$_{\text{вир}}$ – сума за усіма наведеними вище статтями
    калькуляції, являє повну собівартість продукції.

 
    \begin{equation}
    C_{\text{з.г}}. = n_{\text{п.в}}\cdot C_{\text{з.o}} =  0.03\cdot  14319.03  = 429.57 \text{ грн}
    \end{equation}

    \begin{figure}
     \centering
     \begin{tikzpicture}
     
    \pie[
        color = {
            yellow!90!black, 
            green!60!black, 
            blue!60, 
            red!70,
            gray!70,
            teal!20,
            orange!50,
            pink!90},
        text = legend
    ]
    {93.0/Bитрати на матеріали (93\%),
        3.6/Bитрати на покупні вироби(3.6\%),
        0.05/Основна зп (0.05\%),
        0.018/Додаткова зп (0.018\%),
        0.011/Відрахування на соціальне страхування (0.011\%),
        0.01/Загальновиробничі витрати (0.01\%),
        0.06/Адміністративні витрати (0.06\%),
        2.3/Комерційні витрати (2.3\%)}
     
    \end{tikzpicture}
    \caption{Калькуляція собівартості виробу.}
    \end{figure}


    \begin{itemize}
    \item Виробнича собівартість = 14 319.03
    \item Повна собівартість = 14 343.04
    \end{itemize}


 
%%%%%%%%%%%%%%%%%%%%%%%%%%%%%%%%%%%%%%%%%%%%%%%%%%%%%%%%%%%%%%%%%%%%%%%%%%%%%%%
\chapter{ВИЗНАЧЕННЯ ЦІНИ ВИРОБУ}
%%%%%%%%%%%%%%%%%%%%%%%%%%%%%%%%%%%%%%%%%%%%%%%%%%%%%%%%%%%%%%%%%%%%%%%%%%%%%%%
 Серед різних методів ціноутворення на ранніх стадіях проектування досить поширений метод лімітних цін. При цьому визначається верхня і нижня межа ціни.

 \section{Визначення нижньої межі ціни}
 Нижня межа ціни ( Ц$_{H.M.}$ ) захищає інтереси виробника продукції і
передбачає, що ціна повинна покрити витрати виробника, пов’язані з
виробництвом і реалізацією продукції, і забезпечити рівень рентабельності не нижче тієї, що має підприємство при виробництві вже освоєної продукції:

\[\text{Ц}_{\text{H. M.}}=\text{Ц}_{\text{ОПТ. П.}} \cdot\left(1+\frac{\alpha_{\text{ПДВ}}}{100}\right)\]
\[\text{Ц}_{\text{ОПТ. П.}}=\text{С}_{\text{ПОВ.}} \cdot\left(1+\frac{P_{\text {Н}}}{100}\right)\]
де $\text{Ц}_{\text{ОПТ. П.}}$ - оптова ціна підприємства, грн; $C_{\text {пов }}$ - повна собівартість виробу, грн; $P_{H}$ - нормативний рівень рентабельності, $\%\left(P_{H}=20 \%\right) ; \alpha_{\text {пдв }}$-податок на додану вартість, \% $\left(\alpha_{\text {Пдв }}=20 \%\right) ;$ Тоді маємо:
 
 \FPset\cpovna{14 343.04}
 \FPset\koef{1.2}

 \FPmul\copt\cpovna\koef
 \FPeval\copt{round(\copt:2)}

 \FPmul\cnm\copt\koef
 \FPeval\cnm{round(\cnm:2)}
\begin{align*}
&\text{Ц}_{\text{H. M.}}= 14 343.04 \cdot\left(1+\frac{20}{100}\right)=\copt  \text{ грн} \\
&\text{Ц}_{\text{ОПТ. П.}}=\copt \cdot\left(1+\frac{20}{100}\right)= \cnm \text{ грн}
\end{align*}


%%%%%%%%%%%%%%%%%%%%%%%%%%%%%%%%%%%%%%%%%%%%%%%%%%%%%%%%%%%%%%%%%%%%%%%%%%%%%%%
\section{Визначення нижньої межі ціни}
Верхня межа ціни ($\text{Ц}_{\text{B. M.}}$) захищає інтереси споживача і визначається тією
ціною, що споживач готовий заплатити за продукцію з кращою споживчою якістю.
%%%%%%%%%%%%%%%%%%%%%%%%%%%%%%%%%%%%%%%%%%%%%%%%%%%%%%%%%%%%%%%%%%%%%%%%%%%%%%%
\section{Визначення договірної ціни}
Договірну ціну ( $\text{Ц}_{\text{ДОГ}}$  ) встановлюємо за домовленістю між виробником та
споживачем в інтервалі між нижньою та верхньою лімітними цінами.

\[\text{Ц}_{\text{HM}}  < \text{Ц}_{\text{ДОГ}} < \text{Ц}_{\text{ВМ}} \]

В нашому випадку: 20653 < $\text{Ц}_{\text{ДОГ}}$ < 22000 . Приймаємо договірну ціну нового
виробу $\text{Ц}_{\text{ДОГ}}$ =  21300грн.
 
%%%%%%%%%%%%%%%%%%%%%%%%%%%%%%%%%%%%%%%%%%%%%%%%%%%%%%%%%%%%%%%%%%%%%%%%%%%%%%%
\section{Визначення мінімального обсягу виробництва продукції}
Собівартість річного випуску продукції:
\[ C_P = a\cdot \text{C}_{\text{ПOB.}} \cdot Q + b\cdot \text{C}_{\text{ПOB.}} \cdot \chi\]
де $\text{C}_{\text{ПOB.}}$ – повна собівартість одиниці продукції, грн; a, b – відповідно змінні та
умовно-постійні витрати у склад собівартості одиниці продукції ( а = 0,91; b = 0, 09 );
$\chi$ – розрахункова виробнича потужність підприємства з випуску продукції шт./рік
( $\chi = $ 7000шт./рік ); Q – річний обсяг випуску продукції, шт./рік ( Q = 5000шт./рік ).\\ 
Тоді маємо:

\[ C_P = a\cdot \text{C}_{\text{ПOB.}} \cdot Q + b\cdot \text{C}_{\text{ПOB.}} \cdot \chi = 74 296 740.0 \text{ грн}\]

Вартість річного випуску продукції:

\[ Q_P = \text{Ц}_{\text{ДОГ}} \cdot Q =  106 500 000\]

Визначимо при якому обсязі продукції ( $Q_1$ ) виторг від реалізації продукції та
її собівартість співпадають (прибуток дорівнює 0), що відповідає точці
беззбитковості виробництва:
\[ Q_1 = \dfrac{b\cdot \text{C}_{\text{ПOB.}} \cdot \chi}{\text{Ц}_{\text{ДОГ}}-a\cdot \text{C}_{\text{ПOB.}}}  = 1300 \text{ шт }\]

Визначимо при якому обсязі продукції ( $Q_2$ ) буде досягнуто запланований
рівень рентабельності:

\[ Q_2 = \dfrac{b\cdot \text{C}_{\text{ПOB.}} \cdot \chi \cdot \left(1+\dfrac{P_H}{100}\right)}{\text{Ц}_{\text{ДОГ}   }-a\cdot \text{C}_{\text{ПOB.}} \cdot \left(1+\dfrac{P_H}{100}\right)}  = 2652 \text{ шт }\]\

Річний прибуток при досягненні запланованого рівня рентабельності
складає:

\[ \text{П} = (\text{Ц}_{\text{ДОГ}} - \text{C}_{\text{ПOB.}})\cdot Q_{2} = 18 455 554.32 \text{грн}\]

\begin{figure}[h!]
\centering
\img{0.9}{1111.png}{Визначення мінімального обсягу виробництва.}{}
\end{figure}


%%%%%%%%%%%%%%%%%%%%%%%%%%%%%%%%%%%%%%%%%%%%%%%%%%%%%%%%%%%%%%%%%%%%%%%%%%%%%%%
%%%%%%%%%%%%%%%%%%%%%%%%%%%%%%%%%%%%%%%%%%%%%%%%%%%%%%%%%%%%%%%%%%%%%%%%%%%%%%%
%%%%%%%%%%%%%%%%%%%%%%%%%%%%%%%%%%%%%%%%%%%%%%%%%%%%%%%%%%%%%%%%%%%%%%%%%%%%%%%
%%%%%%%%%%%%%%%%%%%%%%%%%%%%%%%%%%%%%%%%%%%%%%%%%%%%%%%%%%%%%%%%%%%%%%%%%%%%%%%
%%%%%%%%%%%%%%%%%%%%%%%%%%%%%%%%%%%%%%%%%%%%%%%%%%%%%%%%%%%%%%%%%%%%%%%%%%%%%%%
%%%%%%%%%%%%%%%%%%%%%%%%%%%%%%%%%%%%%%%%%%%%%%%%%%%%%%%%%%%%%%%%%%%%%%%%%%%%%%%
%%%%%%%%%%%%%%%%%%%%%%%%%%%%%%%%%%%%%%%%%%%%%%%%%%%%%%%%%%%%%%%%%%%%%%%%%%%%%%%
%%%%%%%%%%%%%%%%%%%%%%%%%%%%%%%%%%%%%%%%%%%%%%%%%%%%%%%%%%%%%%%%%%%%%%%%%%%%%%%
%%%%%%%%%%%%%%%%%%%%%%%%%%%%%%%%%%%%%%%%%%%%%%%%%%%%%%%%%%%%%%%%%%%%%%%%%%%%%%%
%%%%%%%%%%%%%%%%%%%%%%%%%%%%%%%%%%%%%%%%%%%%%%%%%%%%%%%%%%%%%%%%%%%%%%%%%%%%%%%
%%%%%%%%%%%%%%%%%%%%%%%%%%%%%%%%%%%%%%%%%%%%%%%%%%%%%%%%%%%%%%%%%%%%%%%%%%%%%%%
%%%%%%%%%%%%%%%%%%%%%%%%%%%%%%%%%%%%%%%%%%%%%%%%%%%%%%%%%%%%%%%%%%%%%%%%%%%%%%%
%%%%%%%%%%%%%%%%%%%%%%%%%%%%%%%%%%%%%%%%%%%%%%%%%%%%%%%%%%%%%%%%%%%%%%%%%%%%%%%
%%%%%%%%%%%%%%%%%%%%%%%%%%%%%%%%%%%%%%%%%%%%%%%%%%%%%%%%%%%%%%%%%%%%%%%%%%%%%%%
%%%%%%%%%%%%%%%%%%%%%%%%%%%%%%%%%%%%%%%%%%%%%%%%%%%%%%%%%%%%%%%%%%%%%%%%%%%%%%%
%%%%%%%%%%%%%%%%%%%%%%%%%%%%%%%%%%%%%%%%%%%%%%%%%%%%%%%%%%%%%%%%%%%%%%%%%%%%%%%
\clearpage
\newpage












\begin{center}
\xmybox[green]{ ВИСНОВОК }
\end{center}
В даній розрахунково-графічній роботі обґрунтовано доцільність
виробництва та виходу на ринок цифрового вимірювача температури, розраховано
його собівартість, що складає 14 343 грн. З калькуляції видно, що 93\%
собівартості становлять витрати на матеріали, покупні вироби та напівфабрикати.
Їх можна скоротити за рахунок вибору оптимальних за ціною постачальників з
оптовими цінами. А 3,6\% собівартості становлять витрати на покупнi вироби. Щоб оптимізувати витрати за цією
складовою, варто автоматизувати виробництво, що скоротить витрати на ЗП та
соціальні внески.\\ 

Була встановлена ціна кінцевого пристрою, а саме 22 000 грн за шт., що є досить розумною
ціною на ринку. Якщо порівнювати з базовою ціною подібного пристрою, то як
бачимо, різниця становить 700 грн. Згідно з законом попиту, менша ціна сприяє
більшому попиту, тому наш виріб буде користуватися попитом та складатиме
значну конкуренцію на ринку.




\end{document}